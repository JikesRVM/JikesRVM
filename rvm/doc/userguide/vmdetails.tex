This section provides some information on various
implementation details for the Jikes\TMweb{} RVM runtime system.

\subsection{Object Model}\label{sssec:objects}
\index{object model}

An {\em object model} dictates how to represent objects in storage;
the best object model will maximize efficiency of frequent language
operations while minimizing storage overhead. \jrvm's
object model is defined by 
\xlink{{\tt VM\_ObjectModel.java}}{\VMObjectModelURL}.
The \jrvm object model underwent major changes between versions 2.3.3
and 2.3.4 of the system.  In particular, in 2.3.3 and before a
``reverse scalar'' object model was used in which scalar objects and
arrays were layed out in different directions in memory to optimize
null pointer checks on AIX.

\paragraph{Overview}
Values in the Java\TMweb{} programming language are either {\em
primitive} ({\it e.g.}\ {\tt int}, {\tt double}, etc.)  or they are
{\em references} (that is, pointers) to objects.  Objects are either
{\em arrays} having elements or {\em scalar objects} having fields.
Objects are logically composed of two primary sections: an object
header (described in more detail below) and the object's instance
fields (or array elements).

The following non-functional requirements govern the Jikes RVM object model:
\begin{itemize}
\item
instance field and array accesses should be as fast as possible,
\item
null-pointer checks should be performed by the hardware if possible, 
\item
method dispatch and other frequent runtime services should be fast,
\item
other (less frequent) Java operations should not be prohibitively
slow, and
\item
per-object storage overhead (ie object header size) should be as small
as possible.
\end{itemize}

\index{field access}
\index{array access}
Assuming the reference to an object resides in a register, compiled
code can access the object's fields at a fixed displacement in a
single instruction.  To facilitate array access, the reference to an
array points to the first (zeroth) element of an array and the
remaining elements are laid out in ascending order.  The number of
elements in an array, its {\em length}, resides just before its first
element. Thus, compiled code can access array elements via base +
scaled index addressing.

The Java programming language requires that an attempt to access an
object through a {\tt null} object reference generates a
\IndexTexttt{NullPointerException}.  In Jikes RVM, references are
machine addresses, and {\tt null} is represented by address $0$.  
On Linux, accesses to both very low and very high memory can be
trapped by the hardware, thus all null checks can be made implicit.
However, the AIX\TMweb{} operating system permits loads from low
memory, but accesses to very high memory (at small {\em negative}
offsets from a null pointer) normally cause hardware 
interrupts. Therefore on AIX only a subset of pointer dereferences can
be protected by an implicit null check. 

\paragraph{Object Header}
\index{object header}
Logically, every object header contains the following components:
\begin{description}
%
\index{TIB}%
\index{superclass}%
\index{interfaces}%
\index{virtual methods}%
\index{instance methods}%
\index{interface methods}%
\begin{Label}{TIB}
\item[TIB Pointer] The TIB (Type Information Block) holds information that
applies to all objects of a type.  The structure of the TIB is defined by 
\xlink{{\tt VM\_TIBLayoutConstants.java}}{\VMTIBLayoutConstantsURL}.
A TIB includes the virtual method table, a pointer to an object
representing the type, and pointers to a few data structures to
facilitate efficient interface invocation and dynamic type checking.
\end{Label}
%
\index{hashing}
\item[Default Hash Code] Each Java object has a default hash code.
%
\index{locking}
\item[Lock] Each Java object has an associated lock state.  This could be a
pointer to a lock object or a direct representation of the lock.
%
\item[Array Length] Every array object provides a {\em length} field
that contains the length (number of elements) of the array.
%
\item[Garbage Collection Information] Each Java object has associated
information used by the memory management system.  Usually this consists of one
or two mark bits, but this could also include some combination of a reference
count, forwarding pointer, etc.
%
\item[Misc Fields] In experimental configurations, the object header
can be expanded to add additional fields to every object, typically to
support profiling. 
\end{description}

An implementation of this abstract header is defined by three files: 
\xlink{{\tt VM\_JavaHeader.java}}{\VMJavaHeaderURL}, which supports
\link{TIB}{TIB} access, default hash codes, and locking; 
\xlink{{\tt VM\_AllocatorHeader.java}}{\VMAllocatorHeaderURL}, which
supports garbage collection information; 
and 
\xlink{{\tt VM\_MiscHeader.java}}{\VMMiscHeaderURL}, which supports
adding additional fields to all objects. 

As of Jikes RVM 2.2.1, the system supports only one implementation
\xlink{{\tt VM\_JavaHeader.java}}{\VMJavaHeaderURL}; 
defining a two-word header for scalar objects.  The source tree includes
three other object models which are currently unsupported, but may be
resurrected in a future release.

\subsection{Methods and Fields}\label{sssec:methods}
\index{methods}
\index{JTOC}
\index{TIB}
\begin{Label}{JTOC}
A compiled method body is an array of machine instructions (stored as
{\tt int}s on PowerPC\TMweb{} and {\tt byte}s on x86-32). 
The {\em Jikes RVM Table of Contents} (JTOC),
stores pointers to static fields and methods.  However, 
pointers for instance fields and instance methods are stored in the receiver 
class's \link{TIB}{TIB}.  Consequently, the dispatch mechanism differs
between static methods and instance methods.
\end{Label}

\paragraph{The JTOC}
\index{JTOC}
\begin{figure}[htb]
\begin{gif}{jtoc}
\vbox{
\hbox{\psfig{file=jtoc.ps,height=3.5in}}
}\hfil
\end{gif}
\caption{The Jikes RVM Table Of Contents and other objects.}
\label{fig:jtoc}
\end{figure}
\index{literals}%
\index{constants}%
\index{dynamic type checking}%
The JTOC holds pointers to 
each of Jikes\TMweb{} RVM's global data structures, as well as
literals, numeric constants and references to \texttt{String} constants.
The JTOC also
contains references to the \link{TIB}{TIB} for each class in the system.  
Since these 
structures can have many types and the JTOC is declared to be an array of 
{\tt int}s,  
Jikes RVM uses a descriptor array, co-indexed with the JTOC, 
to identify the entries containing references.
The JTOC
is depicted \link*{in this figure}[in figure~\Ref, on page~\Pageref{}]{fig:jtoc}.  

\paragraph{Virtual Methods}%
\index{virtual methods}%
\index{instance methods}
A \link{TIB}{TIB} contains pointers to the compiled method 
bodies (executable code) for the virtual methods and other instance
methods of its class. 
Thus, the \link{TIB}{TIB} serves as Jikes RVM's virtual method table.
A virtual method dispatch entails loading the \link{TIB}{TIB} pointer from 
the object reference, loading the address of the method
body at a given offset off the \link{TIB}{TIB} pointer, and making an indirect
branch and link to it.  A virtual method is dispatched to with the
{\instruction invokevirtual} bytecode; other instance methods are
invoked by the {\instruction invokespecial} bytecode.

\paragraph{Static Fields and Methods}%
\index{static methods}
Static fields and pointers to static method bodies are stored in the \link{JTOC}{JTOC}. 
Static method dispatch is simpler than virtual dispatch, since 
a well-known \link{JTOC}{JTOC} entry method holds the address of the
compiled method body.  

\paragraph{Instance Initialization Methods}%
\index{instance initialization methods}
Pointers to the bodies of instance initialization
methods, \texttt{<init>}, are stored in the \link{JTOC}{JTOC}.
(They are always dispatched to with the {\instruction invokespecial} bytecode.)

\paragraph{Lazy Method Compilation}\label{par:lazy}%
\index{lazy method compilation}%
\index{deferred compilation}%
\index{lazy method invocation stub}
Method slots in a \link{TIB}{TIB} or the \link{JTOC}{JTOC} may hold either
a pointer to the compiled code, 
or a pointer to the compiled code of the {\em lazy method invocation stub}.
When invoked, the lazy method invocation stub compiles the 
method, installs a pointer to the compiled code in the appropriate
\link{TIB}{TIB} or the \link{JTOC}{JTOC} slot, then 
jumps to the start of the compiled code. 

\paragraph{Interface Methods}%
\index{interface methods}%
\index{IMT}%
\index{conflict resolution stub}
Regardless of whether or not a virtual method is overridden,
virtual method dispatch is still simple since the method will occupy 
the same \link{TIB}{TIB} offset its defining class and in every sub-class.
However, a method invoked through an {\tt invokeinterface} call rather than
an {\tt invokevirtual call}, will not occupy the same \link{TIB}{TIB} offset in every class that 
implements its interface.  This complicates dispatch for {\tt
invokeinterface}.

\index{Interface Method Table (IMT)}%
\index{IMT: Interface Method Table}%
The simplest, and least efficient way, of locating an interface method 
is to search all the virtual method entries in the \link{TIB}{TIB} finding a match.
Instead, Jikes RVM uses an {\em Interface Method Table} (IMT) which resembles
a virtual method table for interface methods. Any method that could be an interface method has 
a fixed offset into the IMT just as with the \link{TIB}{TIB}. However, unlike in the \link{TIB}{TIB}, two different methods may
share the same offset into the IMT.\@ In this case, a
\index{conflict resolution stub}{\em conflict resolution
stub} is inserted in the IMT.\@ Conflict resolution stubs are
custom-generated machine code sequences that test the value of a
hidden parameter to dispatch to the desired interface method.
For more details, see
\xlink{{\tt VM\-\_In\-ter\-face\-In\-vo\-ca\-tion\-.java}}{\VMInterfaceInvocationURL}.

\subsection{VM Conventions}

\subsubsection{AIX/PowerPC VM Conventions}
\label{aix-conventions}

\index{stack conventions: AIX/PowerPC}
\index{register conventions: AIX/PowerPC}
\index{calling conventions: AIX/PowerPC}

This section describes register, stack, and calling conventions that apply to 
Jikes RVM on AIX\-/\-Pow\-er\-PC\TMweb{}.

Stackframe layout and calling conventions may evolve as our understanding
of Jikes RVM's performance improves.  Where possible, API's should be used
to protect code against such changes.  In particular, we may move to
the AIX\TMweb{} conventions at a later date.  Where code differs from the AIX
conventions, it should be marked with a comment to that effect containing
the string ``\texttt{AIX}''.

%\noindent{\bf Register conventions}
\paragraph{Register conventions}

Registers (general purpose, gp, and floating point, fp) can be roughly
categorized into four types:

\begin{description}
\item[Scratch]
     Needed for method prologue/epilogue.  Can be used by compiler between
     calls.

\item[Dedicated]
     Reserved registers with known contents:
\begin{description}
\item[\link{JTOC}{JTOC} - Jikes RVM Table Of Contents]
        Globally accessible data: constants, static fields and methods.

\item[FP - Frame Pointer]
        Current stack frame (thread specific).

\item[PR - Processor register]
        An object representing the current virtual processor (the one
        executing on the CPU containing these registers).  A field in
        this object contains a reference to the object representing
        the {\tt VM\_Thread} being executed.
\end{description}

\item[Volatile (``caller save'', or ``parameter'')]
     Like scratch registers, these can be used by the compiler as
     temporaries, but they are not preserved across calls.  Volatile
     registers differ from scratch registers in that volatiles
     can be used to pass parameters and result(s) to and from
     methods.

\item[Nonvolatile (``callee save'', or ``preserved'')]
     These can be used (and are preserved across calls), but they must be
     saved on method entry and restored at method exit.  Highest numbered
     registers are to be used first.  (At least initially, nonvolatile
     registers will not be used to pass parameters.)

\item[Condition Register's 4-bit fields]
We follow the AIX conventions to minimize cost in JNI transitions
between C and Java code. 
\begin{description}
\item[CR0, CR1, CR5, CR6, CR7] volatile
\item[CR2, CR3, CR4] non-volatile
\end{description}
The baseline compiler only uses CR0.  The opt compiler allocates CR0,
CR1, CR5 and CR6 and reserves CR7 for use in yieldpoints.  None of the
compilers use CR2, CR3, or CR4 to avoid saving/restoring condition
registers when doing a JNI transition from C to Java code. 
\end{description}


%\noindent{\bf Stack conventions}
\paragraph{Stack conventions}

Stacks grow from high memory to low memory.
The layout of the stackframe appears in a block comment in
\texorhtml{\texttt{\$RVM\_\-ROOT/\-rvm/\-src/\-vm/\-arch/\-powerpc/\-VM\_\-StackframeLay\-outCon\-stants.java}}{\xlink{\texttt{\$RVM\_\-ROOT/\-rvm/\-src/\-vm/\-arch/\-powerpc/\-VM\_\-StackframeLay\-outCon\-stants.java}}{\PPCStackframeLayoutURL}}.

%\noindent{\bf Calling Conventions}
\paragraph{Calling Conventions}

\begin{description}
\item[Parameters]

    All parameters (that fit) are passed in VOLATILE registers.  Object
    reference and int parameters (or results) consume one GP register; long
    parameters, two gp registers (low-order half in the first);  float and
    double parameters, one fp register.  Parameters are 
    assigned to registers
    starting with the lowest volatile register through the highest volatile
    register of the required kind (gp or fp).

    Any additional parameters are passed on the stack in a parameter spill
    area of the caller's stack frame.  The first spilled parameter occupies
    the lowest memory slot.  Slots are filled in the order that parameters
    are spilled.

    An int, or object reference, result is returned in the first volatile
    gp register; a float or double result is returned in the first volatile
    fp register; a long result is returned in the first two volatile gp
    registers (low-order half in the first);

\item[Method prologue responsibilities] (some of these can be omitted for leaf
  methods):

\begin{enumerate}
\item Execute a stackoverflow check, and grow the thread stack if necessary.

\item Save the caller's next instruction pointer (callee's return address,
       from the Link Register).

\item Save any nonvolatile floating-point registers used by callee.

\item Save any nonvolatile general-purpose registers used by callee.

\item Store and update the frame pointer FP.\@

\item Store callee's compiled method ID 

\item Check to see if the Java\TMweb{} thread must yield the {\tt VM\_Processor}
(and yield if threadswitch was requested). 
\end{enumerate}

\item[Method epilogue responsibilities] (some of these can be
ommitted for leaf methods):

\begin{enumerate}
\item Restore FP to point to caller's stack frame.

\item Restore any nonvolatile general-purpose registers used by callee.

\item Restore any nonvolatile floating-point registers used by callee.

\item Branch to the return address in caller.
\end{enumerate}
\end{description}

\subsubsection{Linux/x86-32 VM Conventions} \label{lintel-conventions}
\index{stack conventions: Linux/x86-32}
\index{register conventions: Linux/x86-32}
\index{calling conventions: Linux/x86-32}

This section describes register, stack, and calling conventions that
apply to Jikes RVM on Li\-nux\Rweb{}\-/x86-32.  

%\noindent{\bf Register conventions}
\paragraph{Register conventions}

\begin{description}
\item[EAX]
    First GPR parameter register, first GPR result value (high-order part
    of a long result), otherwise volatile (caller-save).

\item[ECX]
    Scratch.

\item[EDX]
    Second GPR parameter register, second GPR result value (low-order part
    of a long result), otherwise volatile (caller-save).

\item[EBX]
    Nonvolatile.

\item[ESP]
    Stack pointer.

\item[EBP]
    Nonvolatile.

\item[ESI]
    Processor register, reference to the VM\_Processor object for the current
    virtual processor.

\item[EDI]
    Nonvolatile.  (used to hold \link{JTOC}{JTOC} in baseline compiled code)

\end{description}

%\noindent{\bf Stack conventions}
\paragraph{Stack conventions}

Stacks grow from high memory to low memory.
The layout of the stackframe appears in a block comment in
\xlink{{\tt
\$RVM\_\-ROOT/\-rvm/\-src/\-vm/\-arch/\-in\-tel/\-VM\_\-Stack\-frame\-Lay\-out\-Con\-stants.java}}
{\LintelStackframeLayoutURL}.

%\noindent{\bf Calling Conventions}
\paragraph{Calling Conventions}

\begin{description}
\item[At the beginning of callee's prologue]
The first two areas of the callee's stackframe (see above) have been
     established.  ESP points to caller's return address.
     Parameters from caller to callee are as mandated by 
\xlink{{\tt
\$RVM\_\-ROOT/\-rvm/\-src/\-vm/\-arch/\-in\-tel/\-VM\_\-Re\-gis\-terCon\-stants.java}}
{\LintelRegisterConstantsURL}.

\item[After callee's epilogue]
     Callee's stackframe has been removed.  ESP points to the word above where
     callee's frame was.  The framePointer field
     of the VM\_Processor object pointed to by ESI points to A's
     frame.  If B returns a floating-point result, this is at
     the top of the fp register stack.  If B returns a long, the
     low-order word is in EAX and the high-order word is in EDX.\@
     Otherwise, if B has a result, it is in EAX.\@

\end{description}

\subsubsection{OS~X VM Conventions}
\index{calling conventions: OS X/PowerPC}

\paragraph{Calling Conventions}

\newcommand{\OSXCallingConventionsURL}{http:\-//\-de\-vel\-op\-er.ap\-ple.com/\-do\-cu\-men\-ta\-tion\-/\-De\-ve\-lo\-per\-Tools\-/\-Con\-cep\-tu\-al\-/\-Mach\-O\-Run\-time\-/\-Mach\-O\-Run\-time\-.pdf}
The calling conventions we use for OS~X are the same as those listed
at:
\begin{example}
\xlink{\texttt{\OSXCallingConventionsURL}}{\OSXCallingConventionsURL}
\end{example}
They're similar to the Linux PowerPC calling conventions.  One
major difference is how the two operating systems handle the case of
a \texttt{long} parameter when you only have a single parameter register left.


\subsection{Class Loading} \label{sssec:classLoading}
\index{class loading}

Jikes\TMweb{} RVM implements the Java\TMweb{} programming
language's dynamic class loading.  While a class is being loaded it can
be in one of six states. These are:
\begin{description}
\item[vacant] The class is in the process of being loaded.
\item[loaded] the class's bytecode file has been read and parsed successfully.
\item[resolved] the superclass of this class has been loaded and resolved and
the offsets (whether in the object itself, the \link{JTOC}{JTOC}, or the class's
\link{TIB}{TIB}) of its fields and methods have been calculated.
\item[instantiated] the superclass has been instantiated and pointers to the
compiled methods have been inserted into the \link{JTOC}{JTOC} (for static methods) and the
\link{TIB}{TIB} (for virtual methods).
\item[initializing] the superclass has been initialized and the class
initializer is being run.
\item[initialized] the superclass has been initialized and the class
initializer has been run.
\end{description}

The class passes through these states in the following fashion.

\paragraph{Vacant}
The 
\xlink{{\tt VM\_Class}}{\VMClassURL} 
object for this class has been created and registered and is in the
process of being loaded.

\paragraph{Loaded} 
\index{constant pool}
In this state the class file has been read and parsed.  The constant
pool has been constructed. The declared methods and fields of the
class have been loaded.  Loading a method or field consists of reading
its modifiers and attributes. The class's superclass (if any) and
superinterfaces have been loaded as well.

\paragraph{Resolved}
In this state the superclass and superinterfaces of this class have
been resolved as well.  A list of the virtual methods and instance fields
of this class, including the methods and fields inherited from its
superclass has been constructed and the offsets for the instance
fields have been calculated.  Space has been allocated in the \link{JTOC}{JTOC} for
all static fields of the class and for static method pointers and the
appropriate offsets calculated.  The \link{TIB}{TIB} has been initialized and
offsets for the virtual methods have been calculated.

\paragraph{Instantiated}
In this state the superclass has been instantiated.  The
slots in the \link{TIB}{TIB} are filled in with pointers to compiled code or \link{lazy
compilation stubs for
the virtual methods}[.  (See Paragraph~\Ref{} on page~\Pageref.)]{par:lazy}  The slots in the \link{JTOC}{JTOC} are filled in with
pointers to compiled code or lazy compilation stubs for the static methods.

\paragraph{Initializing} 
\index{class initializer}
In this state the superclass has been initialized. The class
initializer is being run. 

\paragraph{Initialized} 
\index{class initializer}
In this state the superclass has been initialized.  The class initializer has 
been run. 

\subsection{Thread System}\label{sec:threads}

This section provides some explanation of how Java\trademark threads are
scheduled and synchronized by the RVM.

\index{threads}
\index{scheduling}
\index{locking}

All Java threads (application threads, garbage collector threads, {\em
etc.})  derive from 
\xlink{{\tt VM\_Thread}}{\VMThreadURL}.  
These threads are multiplexed by
one or more virtual processors (see 
\xlink{{\tt VM\_Processor}}{\VMProcessorURL}
).  Normally, the
number of RVM virtual processors to use is a command line argument
(e.g. {\tt -X:processors=4}) Generally, there should be one RVM
virtual processor for each CPU on an SMP.  Additional virtual
processors may be created to handle threads executing non Java code
through the Java JNI.  Multiple virtual processors require a working
pThread library, each virtual processor being bound to a pThread.  It
is possible to build a system that only uses one virtual processor by
setting the preprocessor directive {\tt
JVM\_WITH\_SINGLE\_VIRTUAL\_PROCESSOR} to 1.  This may give a minor
performance benefit on uniprocessors.

Threads that are not executing are either placed on thead queues
(deriving from 
\xlink{{\tt VM\_AbstractThreadQueue}}{\VMAbstractThreadQueueURL}
) or are proxied (see below).
Thread queues are either global or (virtual) processor local.  The
latter do not require synchronized access but global queues do.
Unfortunately, we did not see how to use Java monitors to provide
this synchronization.  (In part, because it is needed to implement
monitors, see below.)  Instead this low-level synchronization is
provided by 
\xlink{{\tt VM\_ProcessorLock}}{\VMProcessorLockURL}s.

Transferring execution from one thread (A) to another (B) is a complex
operation negotiated by the {\tt yield} and {\tt morph} methods of
VM\_Thread and the {\tt dispatch} method of VM\_Processor.  {\tt
yield} places A on an indicated queue (releasing the lock on the
queue, if it is global).  {\tt morph} uses 
\xlink{{\tt VM\_Magic}}{\VMMagicURL} 
to capture the
state of the running thread and transfers control to {\tt dispatch}.
At this point, the virtual processor is executing in {\em phantom
mode}, it is using A's stack, but not in a way that will be visible to
A when it next gets executed.  {\tt dispatch} removes B from a queue
of executable threads and, using more VM\_Magic, transfers control
to B's stack.  (To B, it looks as if {\em its} call to {\tt dispatch}
has just returned.)  To prevent a different processor from dispatching
A while it is still executing in phantom mode, {\tt yield} sets the
{\tt beingDispatched} field of A, which is only reset by the magic
that transfers control to B.

Currently, RVM has no priority mechanism, that is, all threads run
at the same priority.

Similarly, it has only the most rudimentary load balancing mechanism.
Each virtual processor has a (local) idle queue.  Normally, the
processor's \xlink{{\tt VM\_IdleThread}}{\VMIdleThreadURL} inhabits
this queue.  When no other thread is executable, this thread is
executed.  This thread requests work by setting the static {\tt
idleProcessor} field to its virtual processor.  (When {\tt dispatch}
is looking for a runnable thread it checks that this field is null.
If not, and it has an spare runnable thread, it places the spare
thread on the idle processor's transfer queue.)  The idle thread then
spins for a short period of time waiting for work to materialize.  If
it does, the idle thread returns to its processor's idle queue and
the processor resumes execution of the newly transferred thread.
Otherwise, the idle thread surrenders the remainder of its virtual
processor's time slice back to the operating system.

More sophisticated priority and load-balancing mechanisms are in
order.

RVM uses a light-weight locking scheme to implement Java monitors (see
\xlink{{\tt VM\_Lock}}{\VMLockURL}).  Twenty bits of the status word of 
the object header are used for locking.  If the top bit is set, the
bottom nineteen are an index into an array of heavy-weight locks.
Otherwise, if the object is locked, these bits contain the id of the
thread that holds the lock and a count of how many times it is held.
If a thread tries to lock an object locked with a light-weight lock by
another thread, it can spin, yield, or inflate the lock.  Spinning is
probably a bad idea.  The number of times to yield before inflating is
a matter open for investigation (as are a number of locking
issues, see {\tt VM\_Lock}).  Heavy-weight locks contain an {\tt
enteringQueue} for threads trying to acquire the lock.

A similar mechanism is used to implement Java wait/notification
semantics.  Heavy-weight locks contain a {\tt waitingQueue} for
threads blocked at a Java {\tt wait}.  When a {\tt notify} is
received, a thread is taken from this queue and transferred to a ready
queue.  Priority {\tt wakeupQueue}s are used to implement Java sleep
semantics.  Logically, Java timed-wait semantics entail placing a
thread on both a {\tt waitingQueue} and a {\tt wakeupQueue}.  However, our
implementation only allows a thread to be on one thread queue at
a time.  To accommodate timed-waits, both {\tt waitingQueue}s and
{\tt wakeupQueue}s are queues of {\em proxies} rather than threads.
A \xlink{{\tt VM\_Proxy}}{\VMProxyURL} can represent the same thread
on more than one proxy queue.

\JavaTMFooter

\subsection{VM Callbacks}\label{sssec:callbacks}

\index{callbacks}

The RVM provides callbacks for many runtime events of interest to the RVM
programmer, such as classloading, VM bootimage creation, and VM exit.  The
callbacks allow arbitrary code to be executed on any of the supported events.

The callbacks are accessed through the nested interfaces defined in the 
\xlink{{\tt VM\_Callbacks}}{\VMCallbacksURL} 
class.  There is one interface per event type.  To be notified
of an event, register an instance of a class that implements the corresponding
interface with {\tt VM\_Callbacks} by calling the corresponding {\tt add...()}
method.  For example, to be notified when a class is instantiated (see section
\ref{sssec:classLoading}), first implement the {\tt
VM\_Callbacks.ClassInstantiatedMonitor} interface, and then call {\tt
VM\_Callbacks.addClassInstantiatedMonitor()} with an instance of your class.
When any class is instantiated, the {\tt notifyClassInstantiated} method in
your instance will be invoked.

The RVM currently supports callbacks for the following events.
\begin{itemize}
\item {\tt ClassLoaded}, which happens when a class is {\em loaded}.
\item {\tt ClassResolved}, which happens when a class is {\em resolved}.
\item {\tt ClassInstantiated}, which happens when a class is {\em
instantiated}.
\item {\tt ClassInitialized}, which happens when a class is {\em initialized}.
\item {\tt MethodOverride}, which happens when a method in a newly loaded class
overrides a method in an existing class.
\item {\tt ForName}, which happens when java.lang.Class.forName() is invoked.
\item {\tt BootImageWriting}, which happens when boot image writing is started.
\item {\tt Exit}, which happens when the RVM is about to exit.
\end{itemize}
The appropriate interface names can be obtained by appending ``Monitor'' to the
event names (e.g. the interface to implement for the {\tt MethodOverride} event
is {\tt VM\_Callbacks.MethodOverrideMonitor}).  Likewise, the method to
register the callback is ``add'', followed by the name of the interface (e.g.
the register method for the above interface is {\tt
VM\_Callbacks.addMethodOverrideMonitor()}).

NOTE: there is currently no {\tt Startup} event.  It is in the process of being
implemented.

Since the events for which callbacks are available are internal to the RVM,
there are naturally some limitations on the behavior of the callback code.  For
example, as soon as the exit callback is invoked, all threads are considered
daemon threads (i.e. the VM will not wait for any new threads created in the
callbacks to complete before exiting).  Thus, if the exit callback creates any
threads, it has to {\tt join()} with them before returning.  These limitations
may also produce some unexpected behavior.  For example, while there is an
elementary safeguard on any classloading callback that prevents recursive
invocation (i.e. if the callback code itself causes classloading), there is no
such safeguard across events, so, if there are callbacks registered for both
{\tt ClassLoaded} and {\tt ClassInstantiated} events, and the {\tt
ClassInstantiated} callback code causes dynamic class loading, the {\tt
ClassLoaded} callback will be invoked for the new class, but not the {\tt
ClassInstantiated} callback.

Examples of callback use can be seen in the {\tt VM\_Controller} class in the
adaptive system and most {\tt VM\_Allocator} classes.





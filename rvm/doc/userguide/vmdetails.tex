This section provides some information on various
implementation details for the Jikes RVM runtime system.

\subsection{Object Model} \label{sssec:objects}
\index{object model}
An {\em object model} dictates how to represent objects in storage;
the best object model will maximize efficiency of frequent language
operations while minimizing storage overhead. Jikes RVM's object model
is defined by \xlink{{\tt VM\_ObjectModel.java}}{\VMObjectModelURL}.

Values in the Java\trademark programming language are either {\em
primitive} (e.g. {\tt int}, {\tt double}, etc.)  or they are {\em
references} (that is, pointers) to objects.  Objects are either {\em
arrays} having elements or {\em scalar objects} having fields.  RVM's
object model is governed by the following criteria:
\begin{itemize}
\item{}
field and array accesses should be as fast as possible,
\item{}
null-pointer checks should be performed by the hardware if possible, 
\item{}
virtual method dispatch should be fast, and 
\item{}
other (less frequent) Java operations should not be prohibitively slow.
\item{}
per object storage overhead should be as small as possible
\end{itemize}

\index{field access}
\index{array access}
Assuming the reference to an object is in a register, the object's
fields can be accessed at a fixed displacement in a single
instruction.  To facilitate array access, the reference to an array
points to the first (zeroth) element of an array and the remaining
elements are laid out in ascending order.  The number of elements in
an array, its {\em length}, is kept just before its first
element. Elements of the array can be easily accessed via base +
scaled index addressing.

\index{NullPointerException}
The Java programming language requires that an attempt to access an
object through a {\tt null} 
object reference generate a NullPointerException.  In the RVM, references
are machine addresses, and {\tt null} is represented by address $0$.
The AIX\AIXTMFootnote operating system permits loads from low memory,
but accesses 
to very high memory (at small {\em negative} offsets from a null
pointer) normally cause hardware interrupts.\footnote{In AIX, it is at
least theoretically possible for another process to cause a shared
system library to get loaded into very high memory.  This remote
possibility is not a concern in a research project, but would need to
be addressed by a commercial JVM.  It would be sufficient to forbid
read and write access to the last page of addressable memory.
(Accesses to some of the fields of objects bigger than a page could be
checked explicitly without having a major impact on performance.)}
Thus, attempts to index off a null array reference are trapped by
the hardware, because array accesses require loading the array length
which is $-4$ bytes off the array reference.  A hardware null-pointer
check for field accesses is effected by locating fields at negative
offsets from the object reference.

\JavaTMFooter

\AIXTMFooter 

\subsection{Object headers} \label{sssec:headers}
\index{object header}
A two-word object header is associated with each object.  This header
supports virtual method dispatch, dynamic type checking, memory
management, synchronization, and hashing.  It is located $12$ bytes
below the value of a reference to the object.  (This leaves room for
the length field in case the object is an array, see
figure~\ref{fig:objects}.)

\index{locking}
\index{hashing}
One word of the header is a {\em status} word.  The status word is
divided into three {\em bit-fields}.  The first bit-field is used for
locking.  The second bit-field holds
the default hash value of hashed objects.  The third bit-field is used
by the memory management subsystem.  (The size of these bit-fields is
determined by build-time constants.)

\index{TIB}
\index{superclass}
\index{interfaces}
\index{virtual methods}
The other word of an object header is a reference to the {\em Type
Information Block} (TIB) for the object's class. This structure describes
the object's class including its superclass and the interfaces it implements
as well as has pointers to the class' virtual methods.

\subsection{Methods and Fields}\label{sssec:methods}
\index{methods}
\index{JTOC}
\index{TIB}
A compiled method body is an array of machine instructions (stored as
{\tt int}s). 
While the pointers for static fields and methods are stored in the 
{\em JikesRVM Table of Contents} (JTOC),
pointers for instance fields and virtual methods are stored in the class's TIB.
Consequently the dispatch mechanism is different for static and virtual 
methods.

\paragraph{The JTOC}
\index{JTOC}
\begin{figure}[htb]
\begin{gif}{jtoc}
\vbox{
\hbox{\psfig{file=jtoc.ps,height=3.5in}}
}\hfil
\end{gif}
\caption{The RVM Table Of Contents and other objects.}
\label{fig:jtoc}
\end{figure}
\index{literals}
\index{constants}
\index{dynamic type checking}
All of RVM's global data structures are stored in the JTOC. 
Literals, numeric
constants and references to String constants, are also stored there.
To enable fast common-case dynamic type checking, the JTOC also
contains references to the TIB for each class in the system.  
Since these 
structures can have many types and the JTOC is declared to be an array of 
{\tt int}s  
RVM uses a descriptor array, co-indexed with the JTOC, 
to identify the entries containing references.
A reference to the JTOC is maintained in a dedicated machine register 
(the JTOC register).
The JTOC
is depicted in figure~\ref{fig:jtoc}.  

\paragraph{Virtual Methods}
\index{virtual methods}
A TIB contains pointers to the compiled method 
bodies (executable code) for the virtual methods of its class. 
Thus, the TIB serves as RVM's virtual method table.
A
virtual method dispatch entails loading the TIB pointer at a fixed
offset off the object reference, loading the address of the method
body at a given offset off the TIB pointer, moving this address to the
PowerPC\PowerPCTMFootnote ``link-register'', and executing a
branch-and-link --- four 
instructions.

\paragraph{Static Fields and Methods} 
\index{static methods}
Static fields and methods are stored in the JTOC. Static method dispatch is 
simpler than virtual dispatch requiring only that the offset of method in the 
JTOC be read to find the address of the method. 

\paragraph{Lazy Method Compilation}
\index{lazy method compilation}
\index{deferred compilation}
\index{lazy method invocation stub}
The slots in the TIB or the JTOC may be filled in with 
a pointer to the compiled code for the method itself or if lazy method 
compilation is enabled and the method has not yet been compiled 
it may be filled in with
a pointer to the compiled code of the {\em lazy method invocation stub}.
If the lazy method invocation stub is invoked its action is to compile the 
method, substitute a pointer to the compiled code of the method in the slot in
the TIB or the JTOC from which it was invoked and then 
cause execution to jump to the start of the compiled method. 

\paragraph{Interface Methods}
\index{interface methods}
\index{IMT}
\index{conflict resolution stub}
Regardless of whether or not a method is overridden in a class that inherits it
virtual method dispatch is still very simple since the method body will be at
the same offset in the TIB in its defining class and in every class that 
inherits from it. 
However, where the method is an interface method, 
that is where it is invoked through an {\tt invoke\_interface} call rather than
an {\tt invoke\_virtual call}, its offset is not the same for every class that 
implements its interface and dispatch is more difficult.
The simplest, and least efficient way, of locating an interface method 
is to search all the virtual method entries in the TIB until a match is found.
Another way uses an {\em Interface Method Table} (IMT) which is much like the 
TIB. Any method that could be an interface method has a fixed offset into the 
IMT just as with the TIB. However, unlike in the TIB, two different methods may
share the same offset into the IMT. In this case, a {\em conflict resolution
stub} is inserted in the IMT. Conflict resolution stubs are
custom-generated machine code sequences that test the value of a
hidden parameter to dispatch to the desired interface method.

\PowerPCTMFooter

\subsection{VM Conventions}

%% footnotes not allowed in section headings, so we specialize
\htmlonly{\subsubsection{AIX\trademark VM Conventions}} \label{aix-conventions}
\texonly{\subsubsection{AIX\trademark VM Conventions}} \label{aix-conventions}

\index{stack conventions}
\index{register conventions}
\index{calling conventions}

This section describes register, stack, and calling conventions that apply to 
RVM on PowerPC\PowerPCTMFootnote.

Stackframe layout and calling conventions may evolve as our understanding
of the RVM's performance improves.  Where possible API's should be used
to protect code against such changes.  In particular, we may move to
the AIX conventions at a later date.  Where code differs from the AIX
conventions, it should be marked with a comment to that effect containing
the string "AIX".

\noindent{\bf Register conventions}

Registers (general purpose, gp, and floating point, fp) can be roughly
categorized into four types:

\begin{description}
\item [Scratch]
     Needed for method prologue/epilogue.  Can be used by compiler between
     calls.

\item[Dedicated]
     Reserved registers with known contents:
\begin{description}
\item [JTOC - JikesRVM Table Of Contents]
        Globally accessible data: constants, static fields and methods.

\item [FP - Frame Pointer]
        Current stack frame (thread specific).

\item [TI - Thread (locking) Id]
        Used to set (and test) the locking field of light weight object
        locks.  Can also be shifted to get the index of an object
        representing the current thread in into a global array.

\item [PR - Processor register]
        An object representing the current virtual processor (the one
        executing on the CPU containing these registers).  A field in
        this object contains a reference to the object representing
        the VM\_Thread being executed.
\end{description}

\item [volatile ("caller save", or "parameter")]
     Like scratch registers these can be used by the compiler as
     temporaries, but they are not preserved across calls.  (Volatile
     registers differ from scratch registers in that volatiles
     can be used to pass parameters and result(s) to and from
     methods.)

\item [Nonvolatile ("callee save", or "preserved")]
     These can be used (and are preserved across calls), but they must be
     saved on method entry and restored at method exit.  Highest numbered
     registers are to be used first.  (At least initially, nonvolatile
     registers will not be used to pass parameters.)

\item[Condition Register's 4-bit fields]
\begin{description}
\item    [CR0 - CR1] scratch

\item    [CR2 - TSCR] dedicated (thread switching, bit 8 TSCRB) (this
     convention is being phased out, and in fact is not being used in RVM 2.0)

\item    [CR3 - CR7] scratch
\end{description}
\end{description}


\noindent{\bf Stack conventions}

Stacks grow from high memory to low memory.
The layout of the stackframe appears in a block comment in
\xlink{{\tt \$RVM\_ROOT/rvm/src/vm/arch/powerpc/VM\_StackframeLayoutConstants.java}}
{\PPCStackframeLayoutURL}.

\noindent{\bf Calling Conventions}

\begin{description}
\item[Parameters]

    All parameters (that fit) are passed in VOLATILE registers.  Object
    reference and int parameters (or results) consume one GP register; long
    parameters, two gp registers (low-order half in the first);  float and
    double parameters, one fp registers.  Parameters are 
    assigned to registers
    starting with the lowest volatile register through the highest volatile
    register to the highest nonvolatile of the required kind (gp or fp).

    Any additional parameters are passed on the stack in an parameter spill
    area of the caller's stack frame.  The first spilled parameter occupies
    the lowest memory slot.  Slots are filled in the order that parameters
    are spilled.

    An int, or object reference, result is returned in the first volatile
    gp register; a float or double result is returned in the first volatile
    fp register; a long result is returned in the first two volatile gp
    registers (low-order half in the first);

\item [Method prologue responsibilities] (some of these can be omitted for leaf
  methods):

\begin{enumerate}
\item Save the caller's next instruction pointer (callee's return address,
       from the Link Register).

\item Save any nonvolatile floating-point registers used by callee.

\item Save any nonvolatile general-purpose registers used by callee.

\item Store and update the frame pointer FP.

\item Store callee's compiled method ID (eventually, this may not be needed).

\item Check to see if the Java\trademark thread must yield the VM\_Processor
(and yield if threadswitch was requested). 
\end{enumerate}

\item [Method epilogue responsibilities]

\begin{enumerate}
\item Restore FP to point to caller's stack frame.

\item Restore any nonvolatile general-purpose registers used by callee.

\item Restore any nonvolatile floating-point registers used by callee.

\item Branch to the return address in caller.
\end{enumerate}
\end{description}

\subsubsection{Linux/IA32 VM Conventions} \label{lintel-conventions}
\index{stack conventions}
\index{register conventions}
\index{calling conventions}

This section describes register, stack, and calling conventions that
apply to RVM on Linux/IA32.  {\em Linux/IA32 conventions are still
changing; be sure to check the relevant files for the most accurate
information!}

\noindent{\bf Register conventions}

\begin{description}
\item [EAX]
    First GPR parameter register, first GPR result value (high-order part
    of a long result), otherwise volatile (caller-save).

\item[ECX]
    Scratch.

\item[EDX]
    Second GPR parameter register, second GPR result value (low-order part
    of a long result), otherwise volatile (caller-save).

\item[EBX]
    Nonvolatile.

\item[ESP]
    Stack pointer.

\item[EBP]
    Nonvolatile.

\item[ESI]
    Processor register, reference to the VM\_Processor object for the current
    virtual processor.

\item[EDI]
    Nonvolatile.  (used to hold JTOC in baseline compiled code)

\end{description}

\noindent{\bf Stack conventions}

Stacks grow from high memory to low memory.
The layout of the stackframe appears in a block comment in
\xlink{{\tt
\$RVM\_ROOT/rvm/src/vm/arch/intel/VM\_StackframeLayoutConstants.java}}
{\LintelStackframeLayoutURL}.

\noindent{\bf Calling Conventions}

\begin{description}
\item[At the beginning of callee's prologue]
The first two areas of the callee's stackframe (see above) have been
     established.  ESP points to caller's return address.
     Parameters from caller to callee are as mandated by 
\xlink{{\tt
\$RVM\_ROOT/rvm/src/vm/arch/intel/VM\_RegisterConstants.java}}
{\LintelRegisterConstantsURL}.
\item[After callee's epilogue]
     Callee's stackframe has been removed.  ESP points to the word above where
     callee's frame was.  The framePointer field
     of the VM\_Processor object pointed to by ESI points to A's
     frame.  If B returns a floating-point result, this is at
     the top of the fp register stack.  If B returns a long, the
     low-order word is in EAX and the high-order word is in EDX.
     Otherwise, if B has a result, it is in EAX.

\end{description}

\JavaTMFooter

\AIXTMFooter

\PowerPCTMFooter

\subsection{Class Loading} \label{sssec:classLoading}
\index{class loading}

RVM implements the Java\trademark programming language's dynamic class
loading. While a class is being loaded it 
can be in one of five states. These are
\begin{description}
\item[vacant] a forward reference exists to the class but loading has not yet 
begun.
\item[loaded] the class's bytecode file has been read and parsed successfully.
\item[resolved] the superclass of this class has been loaded and resolved and
the offsets (whether in the object itself, the JTOC, or the class's TIB) of its 
fields and methods have been calculated.
\item[instantiated] the superclass has been instantiated and pointers to the
compiled methods have been inserted into the JTOC(for static methods) and the
TIB (for virtual methods).
\item[initializing] the superclass has been initialized and the class
initializer is being run.
\item[initialized] the superclass has been initialized and the class
initializer has been run.
\end{description}

The class passes through these states in the following fashion.

\paragraph{Vacant}
The 
\xlink{{\tt VM\_Class}}{\VMClassURL} 
object for this class has been created and registered. 
A class can be in this state if a reference to the class exists in a constant
pool of some other class.

\paragraph{Loaded} 
\index{constant pool}
In this state the class file has been read and parsed.  The constant pool has 
been constructed. The declared methods and fields of the class have been loaded.
Loading a method or field consists of reading its modifiers and attributes.

\paragraph{Resolved}
In this state the superclass of this class has been loaded and resolved. 
A list of the virtual methods and instance fields of this class, including the 
methods and fields
inherited from its superclass has been constructed and the offsets for the 
instance fields have been calculated.  
Space has been allocated in the JTOC for all static fields of the class and for
static method pointers and the appropriate offsets calculated.
The TIB has been initialized and offsets for the virtual methods have been
calculated.

\paragraph{Instantiated}
In this state the superclass of this state has been instantiated. 
The slots in the TIB are filled in with pointers to the compiled code for the 
virtual methods. 
The slots in the JTOC are filled in with pointers to the compiled code for the 
static methods.

\paragraph{Initializing} 
\index{class initializer}
In this state the superclass has been initialized. The class
initializer is being run. 

\paragraph{Initialized} 
\index{class initializer}
In this state the superclass has been initialized. The class initializer has 
been run. 

\JavaTMFooter

\subsection{Thread System}\label{sec:threads}

This section provides some explanation of how Java\trademark threads are
scheduled and synchronized by the RVM.

\index{threads}
\index{scheduling}
\index{locking}

All Java threads (application threads, garbage collector threads, {\em
etc.})  derive from 
\xlink{{\tt VM\_Thread}}{\VMThreadURL}.  
These threads are multiplexed by
one or more virtual processors (see 
\xlink{{\tt VM\_Processor}}{\VMProcessorURL}
).  Normally, the
number of RVM virtual processors to use is a command line argument
(e.g. {\tt -X:processors=4}) Generally, there should be one RVM
virtual processor for each CPU on an SMP.  Additional virtual
processors may be created to handle threads executing non Java code
through the Java JNI.  Multiple virtual processors require a working
pThread library, each virtual processor being bound to a pThread.  It
is possible to build a system that only uses one virtual processor by
setting the preprocessor directive {\tt
JVM\_WITH\_SINGLE\_VIRTUAL\_PROCESSOR} to 1.  This may give a minor
performance benefit on uniprocessors.

Threads that are not executing are either placed on thead queues
(deriving from 
\xlink{{\tt VM\_AbstractThreadQueue}}{\VMAbstractThreadQueueURL}
) or are proxied (see below).
Thread queues are either global or (virtual) processor local.  The
latter do not require synchronized access but global queues do.
Unfortunately, we did not see how to use Java monitors to provide
this synchronization.  (In part, because it is needed to implement
monitors, see below.)  Instead this low-level synchronization is
provided by 
\xlink{{\tt VM\_ProcessorLock}}{\VMProcessorLockURL}s.

Transferring execution from one thread (A) to another (B) is a complex
operation negotiated by the {\tt yield} and {\tt morph} methods of
VM\_Thread and the {\tt dispatch} method of VM\_Processor.  {\tt
yield} places A on an indicated queue (releasing the lock on the
queue, if it is global).  {\tt morph} uses 
\xlink{{\tt VM\_Magic}}{\VMMagicURL} 
to capture the
state of the running thread and transfers control to {\tt dispatch}.
At this point, the virtual processor is executing in {\em phantom
mode}, it is using A's stack, but not in a way that will be visible to
A when it next gets executed.  {\tt dispatch} removes B from a queue
of executable threads and, using more VM\_Magic, transfers control
to B's stack.  (To B, it looks as if {\em its} call to {\tt dispatch}
has just returned.)  To prevent a different processor from dispatching
A while it is still executing in phantom mode, {\tt yield} sets the
{\tt beingDispatched} field of A, which is only reset by the magic
that transfers control to B.

Currently, RVM has no priority mechanism, that is, all threads run
at the same priority.

Similarly, it has only the most rudimentary load balancing mechanism.
Each virtual processor has a (local) idle queue.  Normally, the
processor's \xlink{{\tt VM\_IdleThread}}{\VMIdleThreadURL} inhabits
this queue.  When no other thread is executable, this thread is
executed.  This thread requests work by setting the static {\tt
idleProcessor} field to its virtual processor.  (When {\tt dispatch}
is looking for a runnable thread it checks that this field is null.
If not, and it has an spare runnable thread, it places the spare
thread on the idle processor's transfer queue.)  The idle thread then
spins for a short period of time waiting for work to materialize.  If
it does, the idle thread returns to its processor's idle queue and
the processor resumes execution of the newly transferred thread.
Otherwise, the idle thread surrenders the remainder of its virtual
processor's time slice back to the operating system.

More sophisticated priority and load-balancing mechanisms are in
order.

RVM uses a light-weight locking scheme to implement Java monitors (see
\xlink{{\tt VM\_Lock}}{\VMLockURL}).  Twenty bits of the status word of 
the object header are used for locking.  If the top bit is set, the
bottom nineteen are an index into an array of heavy-weight locks.
Otherwise, if the object is locked, these bits contain the id of the
thread that holds the lock and a count of how many times it is held.
If a thread tries to lock an object locked with a light-weight lock by
another thread, it can spin, yield, or inflate the lock.  Spinning is
probably a bad idea.  The number of times to yield before inflating is
a matter open for investigation (as are a number of locking
issues, see {\tt VM\_Lock}).  Heavy-weight locks contain an {\tt
enteringQueue} for threads trying to acquire the lock.

A similar mechanism is used to implement Java wait/notification
semantics.  Heavy-weight locks contain a {\tt waitingQueue} for
threads blocked at a Java {\tt wait}.  When a {\tt notify} is
received, a thread is taken from this queue and transferred to a ready
queue.  Priority {\tt wakeupQueue}s are used to implement Java sleep
semantics.  Logically, Java timed-wait semantics entail placing a
thread on both a {\tt waitingQueue} and a {\tt wakeupQueue}.  However, our
implementation only allows a thread to be on one thread queue at
a time.  To accommodate timed-waits, both {\tt waitingQueue}s and
{\tt wakeupQueue}s are queues of {\em proxies} rather than threads.
A \xlink{{\tt VM\_Proxy}}{\VMProxyURL} can represent the same thread
on more than one proxy queue.

\JavaTMFooter

\subsection{VM Callbacks}\label{sssec:callbacks}

\index{callbacks}

The RVM provides callbacks for many runtime events of interest to the RVM
programmer, such as classloading, VM bootimage creation, and VM exit.  The
callbacks allow arbitrary code to be executed on any of the supported events.

The callbacks are accessed through the nested interfaces defined in the 
\xlink{{\tt VM\_Callbacks}}{\VMCallbacksURL} 
class.  There is one interface per event type.  To be notified
of an event, register an instance of a class that implements the corresponding
interface with {\tt VM\_Callbacks} by calling the corresponding {\tt add...()}
method.  For example, to be notified when a class is instantiated (see section
\ref{sssec:classLoading}), first implement the {\tt
VM\_Callbacks.ClassInstantiatedMonitor} interface, and then call {\tt
VM\_Callbacks.addClassInstantiatedMonitor()} with an instance of your class.
When any class is instantiated, the {\tt notifyClassInstantiated} method in
your instance will be invoked.

The RVM currently supports callbacks for the following events.
\begin{itemize}
\item {\tt ClassLoaded}, which happens when a class is {\em loaded}.
\item {\tt ClassResolved}, which happens when a class is {\em resolved}.
\item {\tt ClassInstantiated}, which happens when a class is {\em
instantiated}.
\item {\tt ClassInitialized}, which happens when a class is {\em initialized}.
\item {\tt MethodOverride}, which happens when a method in a newly loaded class
overrides a method in an existing class.
\item {\tt ForName}, which happens when java.lang.Class.forName() is invoked.
\item {\tt BootImageWriting}, which happens when boot image writing is started.
\item {\tt Exit}, which happens when the RVM is about to exit.
\end{itemize}
The appropriate interface names can be obtained by appending ``Monitor'' to the
event names (e.g. the interface to implement for the {\tt MethodOverride} event
is {\tt VM\_Callbacks.MethodOverrideMonitor}).  Likewise, the method to
register the callback is ``add'', followed by the name of the interface (e.g.
the register method for the above interface is {\tt
VM\_Callbacks.addMethodOverrideMonitor()}).

NOTE: there is currently no {\tt Startup} event.  It is in the process of being
implemented.

Since the events for which callbacks are available are internal to the RVM,
there are naturally some limitations on the behavior of the callback code.  For
example, as soon as the exit callback is invoked, all threads are considered
daemon threads (i.e. the VM will not wait for any new threads created in the
callbacks to complete before exiting).  Thus, if the exit callback creates any
threads, it has to {\tt join()} with them before returning.  These limitations
may also produce some unexpected behavior.  For example, while there is an
elementary safeguard on any classloading callback that prevents recursive
invocation (i.e. if the callback code itself causes classloading), there is no
such safeguard across events, so, if there are callbacks registered for both
{\tt ClassLoaded} and {\tt ClassInstantiated} events, and the {\tt
ClassInstantiated} callback code causes dynamic class loading, the {\tt
ClassLoaded} callback will be invoked for the new class, but not the {\tt
ClassInstantiated} callback.

Examples of callback use can be seen in the {\tt VM\_Controller} class in the
adaptive system and most {\tt VM\_Allocator} classes.



\subsection{Support for Soot-style Annotations}\label{sssec:annotations}
Jikes RVM optionally supports reading and using Soot-style class file
annotations.  Such annotations can specify that a null check or bounds
check is redundant for a particular byte code instruction.  These
annotations are produced by the {\bf Soot} class file optimizer
available from
\xlink{\SOOTURL}{\SOOTURL}.  

When the annotations command-line option is true, class file
annotations are processed by \xlink{{\tt
VM\_Method.java}}{\VMMethodURL} and stored internally in Jikes RVM.
During compilation (either baseline or optimizing), the compiler 
queries the {\tt VM\_Method} class for a particular bytecode to
determine if the generation of a null or bounds check can be
suppressed. 

Processing Soot-style annotations is not enabled, by default.  To
enable this support specify {\tt ``annotations=true''} to the
appropriate compiler.  For example, use {\tt -X:irc:annotations=true}
for non-adaptive images and {\tt -X:aos:irc:annotations=true} for
adaptive images.




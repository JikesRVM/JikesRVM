As of this writing, about twelve percent (128 out of 1045) of the Java
source code in Jikes RVM contains preprocessor constructs.  The
preprocessor syntax has not been documented until this writing (July,
2003).  Here is the help message the preprocessor now displays; this
is a placeholder until someone writes more attractive documentation
for it.

\subsection{Usage}

\begin{verbatim}
Usage: preprocessModifiedFiles [--help]
        [ --disable-modification-exit-status ] [--trace]
        [ -D<name>[ =1 | =0 | =<string-value> ] ]... 
        [ -- ] <output directory> [ <input file> ]...
   Preprocess source files that are new or have changed.

   The timestamp of each input file is compared with that
   of the corresponding file in the <output directory>.  If the
   output file doesn't exist, or is older than the input file,
   then the input file is copied to the <output directory>, 
   with preprocessing.

   Invocation parameters:
      - zero or more preprocessor directives of the form "-D<name>=1", of the
        equivalent shorthand form "-D<name>", of the form "-D<name>=0",
         and/or of the form "-D<name>=<string-value>".
      - name of directory to receive output files
      - names of zero or more input files
      - other flags

   Process exit status means:
           0 - no files changed
           1 - some files changed
       other - trouble

   With --disable-modification-exit-status, the process will
   exit wtih status 0 even when some files changed.  Under
   --disable-modification-exit-status, non-zero exit status
   always means trouble.


   --trace  The preprocessor prints a '.' for each file that did
          not need to be changed and a '+' for each file that needed
          preprocessing again.

   --verbose, -v  The preprocessor prints a message for each file
         examined, and prints a summary at the end 

   --help, -h  Show this long help message and exit with status 0.

   -D<name>=0 is a no-op; equivalent to never defining <name>.

   -D<name>=1 and -D<name> are equivalent.

   -D<name>=<any-string-value-but-0-or-1> will define a constant that is
       usable in a //-#value dirctive.

   The following preprocessor directives are recognized
   in source files.  They must be the first non-whitespace characters
   on a line of input.

      //-#if    <name>
            It is not an error for <name> to be undefined.  Only checks
            whether <name> is defined.

            "//-#if" also supports the constructs '!' (invert the sense of 
            the next test), '&&', and '||'.  '!' binds more tightly 
            than '&&' and '||' do.   '&&' and '||' are at the same precedence.
            The preprocessor does not support parentheses in //-#if constructs
            If you don't mix '&&' and '||' in the same line, you'll be OK.

      //-#elif  <name>
            Takes the same arguments that //-#if does. 

      //-#else  <optional-comment>

      //-#endif <optional-comment>

      //-#value <preprocessor-symbol>
            <-preprocessor-symbol> is the name of a constant defined on the
            command line with -D; it will be replaced with the defined value.

            It is an error for <preprocessor-symbol> not to be defined.

            It is an error for <preprocessor-symbol> to have been defined with
            -D<name>=1 or with -D<name>

           (This is an odd restriction, but is the way the code was written
           when I found it.  You're free to rewrite it if you want it to act
           just like the C preprocessor does.)

     There is no equivalent to the C preprocessor's "#define" construct;
     all constants are defined on the command line with "-D".
\end{verbatim}

\subsection{Where It Is}

The source code for the preprocessor is in {\tt
rvm/src/tools/preprocessor}.  The Jikes RVM build process installs it
as RVM\_BUILD{\tt /jbuild.prep}.  The preprocessing is done automatically
during the build process.



\documentclass{article}
\usepackage{hyperlatex}
\usepackage{epsfig}
\usepackage{xspace}
\usepackage{verbatim}
\usepackage{makeidx}
\W\usepackage{longtable}
\W\usepackage{jframes}

%%%% These are grammatical commands.
\newcommand{\Mor}{{\bf \(|\)}}  %meta-Or
\newcommand{\Mlbr}{{\bf \{}}    %left-brace
\newcommand{\Mrbr}{{\bf \}}}    %right-brace
\newcommand{\Mlsq}{{\bf [}}     %left-square
\newcommand{\Mrsq}{{\bf ]}}     %right-brace
\newcommand{\Mmeta}[1]{{\it \(<\)#1\(>\)}} %meta-token
%%% Hyperlatex doesn't seem to know about \rm.  
\W \newcommand{\rm}{}
\W \newcommand{\raggedright}{}
\T \newcommand{\OneCMStrut}{\hspace*{1cm}}
\W \newcommand{\OneCMStrut}{}
\newcommand{\Mlitch}[1]{{ \bf ``{\tt #1}''}} %literal char or txt
\newcommand{\MZeroOrMore}[1]{{\(\Mlsq\mbox{#1}\Mrsq_+\)}}
\newcommand{\MZeroOrOne}[1]{{\(\Mlsq\mbox{#1}\Mrsq\)}}

%%%
%%% End of Meta-grammar symbols.
%%% 
% $Id$
%
% trademark-macros.tex
% Two trademark styles we use in the Jikes RVM User's Guide.
%%
%% Rationale: As of October 16th and 17th, 2003, I have changed the
% style of trademark attribution that we use.  I do not know whether
%there may be legal reasons why we were using the old style.
%
%% I took the footnote off from the title page.  That's what Sun does with
% their Java books.  If you think otherwise, put it back.  You should
%probably be able to do it by changing the boolean variable below, TMOnOwnPage.

%% Changed back pending dicussion with Mike or IBM attorneys.

%% \T \usepackage{ifthen}
%% \T \newboolean{TMOnOwnPage}
%% \T \setboolean{TMOnOwnPage}{true}
% An earlier attempt to get a nice pretty TM symbol.
% \T \renewcommand{\texttrademark}{\ensuremath{^{\mbox{\tiny \texttrademark}}}}
% \T \newcommand{\trademark}{\ }

% From the SPEC page (HTML ISO Latin 1):
% [ <b>&reg;</b> = Registered Trademark, <b>&#153;</b> = Trademark
% Unicode: Registered Trademark is 0x00AE,
% Unicode: Trademark sign is 0x2122 or <super> 0x0054 0x004D
% TS1 (the Cork encoding) uses 0227 (octal)

%% %% \ifthenelse{\boolean{TMOnOwnPage}}{%
%%   \W \newcommand{\trademark}{\xmlsym{##8482}}
%%   \W \newcommand{\registeredTrademark}{\xml{small}\xml{sup}\xmlsym{reg}\xml{/sup}\xml{/small}}
%%   %  \T \newcommand{\registeredTrademark}{\}
%%   %% TM notices on a page of trademark statements.
%%   % Earlier attempts:
%%   % \title{\texonly{\vfill} {\huge The $\mbox{Jikes}^{\mbox{\tiny TM}}$ 
  
%%   %% The \*TMheading{} forms of the macros are for inside
%%   %% LaTeX \section{} commands.
\T \renewcommand{\texttrademark}{\ensuremath{%
\T               \protect\raisebox{.41\baselineskip}{\tiny TM}}}
\T \newcommand{\trademark}{\texttrademark}
\T \newcommand{\registeredTrademark}{\trademark}

%% The spaces do not hurt the HTML and may help it.
\W \newcommand{\trademark}{\xmlsym{##8482}\ }
\W \newcommand{\registeredTrademark}{\xml{sup}\xmlsym{reg}\xml{/sup}\ }



  \W \begin{iftex}
  %% This looks quite nice on the title page --augart, 2003 Oct 16
  \newcommand{\AIXTM}{AIX\ensuremath{%
     \protect\raisebox{.41\baselineskip}{\tiny TM}}}
  \newcommand{\AIXTMheading}{\AIXTM}

  \newcommand{\AIXTMweb}{AIX}
  \newcommand{\AIXTMwebheading}{AIX}

  \newcommand{\IBMR}{IBM\ensuremath{%\!
      \protect\raisebox{.41\baselineskip}{\tiny TM}}}
  \newcommand{\IBMRweb}{IBM}

  \newcommand{\IntelR}{Intel\ensuremath{\protect\raisebox{.41\baselineskip}{\tiny TM}}}
  \newcommand{\IntelRheading}{\IntelR}

  \newcommand{\IntelRweb}{Intel}
  \newcommand{\IntelRwebheading}{Intel}

  \newcommand{\JikesTM}{Jikes\ensuremath{\!%
      \protect\raisebox{.41\baselineskip}{\tiny TM}}}
  \newcommand{\JikesTMweb}{Jikes}

  \newcommand{\JavaTM}{Java\ensuremath{\!%\!\!\!
      \protect\raisebox{.41\baselineskip}{\tiny TM}}}

  \newcommand{\JavaTMweb}{Java}

  \newcommand{\JavadocTM}{Javadoc\ensuremath{\!%\!\!\!
      \protect\raisebox{.41\baselineskip}{\tiny TM}}}
  \newcommand{\JavadocTMweb}{Javadoc}

  \newcommand{\LinuxR}{Linux\ensuremath{\!%\!
      \protect\raisebox{.41\baselineskip}{\tiny TM}}}
  \newcommand{\LinuxRheading}{\LinuxR}

  \newcommand{\LinuxRweb}{Linux}
  \newcommand{\LinuxRwebheading}{Linux}

  \newcommand{\PostScriptR}{PostScript\ensuremath{\!%\!
      \protect\raisebox{.41\baselineskip}{\tiny TM}}}
  \newcommand{\PostScriptRweb}{PostScript}


  \newcommand{\PowerPCTM}{PowerPC\ensuremath{%\!
      \protect\raisebox{.41\baselineskip}{\tiny TM}}}
  \newcommand{\PowerPCTMweb}{PowerPC}

  \newcommand{\RedHatTM}{Red Hat\ensuremath{%
      \!\protect\raisebox{.41\baselineskip}{\tiny TM}}}
  \newcommand{\RedHatTMheading}{\RedHatTM}

  \newcommand{\RedHatTMweb}{Red Hat}
  \newcommand{\RedHatTMwebheading}{Red Hat}

  \newcommand{\SPECR}{SPEC\ensuremath{%\!
      \protect\raisebox{.41\baselineskip}{\tiny TM}}}
  \newcommand{\SPECRheading}{\SPECR}
  \newcommand{\SPECjvmR}{SPECjvm\ensuremath{\!%\!\!
      \protect\raisebox{.41\baselineskip}{\tiny TM}}}
  \newcommand{\SPECjvmRheading}{\SuSER}
  \newcommand{\SPECjbbR}{SPECjbb\ensuremath{\!%\!
      \protect\raisebox{.41\baselineskip}{\tiny TM}}}
  \newcommand{\SPECjbbRheading}{\SPECjbbR}

  \newcommand{\SunR}{Sun\ensuremath{%
      \protect\raisebox{.41\baselineskip}{\tiny TM}}}
  \newcommand{\SunRweb}{Sun}

  \newcommand{\SuSER}{SuSE\ensuremath{%
      \protect\raisebox{.41\baselineskip}{\tiny TM}}}
  \newcommand{\SuSERheading}{\SuSER}

  \newcommand{\SuSERweb}{SuSE}
  \newcommand{\SuSERwebheading}{SuSE}


  \newcommand{\UnixR}{Unix\ensuremath{\!%\!\!
      \protect\raisebox{.41\baselineskip}{\tiny TM}}}
  \W \end{iftex}

  \T \begin{ifhtml}

  \W \newcommand{\AIXTM}{AIX\link{\trademark}{trademarks}}
  \W \newcommand{\AIXTMheading}{AIX\trademark}

  \newcommand{\AIXTMweb}{\AIXTM}
  \newcommand{\AIXTMwebheading}{\AIXTMheading}

  \W \newcommand{\IBMR}{IBM\link{\registeredTrademark}{trademarks}}
  \W \newcommand{\IBMRweb}{\IBMR}

  \W \newcommand{\IntelR}{Intel\link{\registeredTrademark}{trademarks}}
  \W \newcommand{\IntelRheading}{Intel\registeredTrademark}
  \W \newcommand{\IntelRweb}{\IntelR}
  \W \newcommand{\IntelRwebheading}{\IntelRheading}

  \W \newcommand{\JavaTM}{Java\link{\trademark}{trademarks}}
  \W \newcommand{\JavaTMweb}{\JavaTM}

  \W \newcommand{\JavadocTM}{Javadoc\link{\trademark}{trademarks}}
  \W \newcommand{\JavadocTMweb}{\JavadocTM}

  \W \newcommand{\JikesTM}{Jikes\link{\trademark}{trademarks}}
  \W \newcommand{\JikesTMweb}{\JikesTM}

  \W \newcommand{\LinuxR}{Linux\link{\registeredTrademark}{trademarks}}
  \W \newcommand{\LinuxRheading}{Linux\registeredTrademark}
  \W \newcommand{\LinuxRweb}{\LinuxR}
  \W \newcommand{\LinuxRwebheading}{\LinuxRheading}

  \W \newcommand{\PostScriptR}{PostScript\link{\registeredTrademark}{trademarks}}
  \W \newcommand{\PostScriptRweb}{\PostScriptR}


  \W \newcommand{\PowerPCTM}{PowerPC\link{\trademark}{trademarks}}
  \W \newcommand{\PowerPCTMweb}{\PowerPCTM}

  \W \newcommand{\SPECR}{SPEC\link{\registeredTrademark}{trademarks}}

  \W \newcommand{\SPECjvmR}{SPECjvm\link{\registeredTrademark}{trademarks}}
  \W \newcommand{\SPECjvmRheading}{SPECjvm\registeredTrademark}

  \W \newcommand{\SPECjbbR}{SPECjbb\link{\registeredTrademark}{trademarks}}
  \W \newcommand{\SPECjbbRheading}{SPECjbb\registeredTrademark}

  \W \newcommand{\RedHatTM}{Red Hat\link{\trademark}{trademarks}}
  \W \newcommand{\RedHatTMheading}{Red Hat\trademark}

  \W \newcommand{\RedHatTMweb}{\RedHatTM}
  \W \newcommand{\RedHatTMwebheading}{\RedHatTMheading}

  \W \newcommand{\SunR}{Sun\link{\registeredTrademark}{trademarks}}

  \W \newcommand{\SunRweb}{\SunR}

  \W \newcommand{\SuSER}{SuSE\link{\registeredTrademark}{trademarks}}
  \W \newcommand{\SuSERheading}{SuSE\registeredTrademark}

  \W \newcommand{\SuSERweb}{\SuSER}
  \W \newcommand{\SuSERwebheading}{\SuSERheading}

  \W \newcommand{\UnixR}{Unix\link{\registeredTrademark}{trademarks}}
  \T \end{ifhtml}

%Unnecessary:
%%   \newcommand{\JikesTMFootnote}{}
%%   \newcommand{\AIXTMFootnote}{}
%%   \newcommand{\PowerPCTMFootnote}{}
%%   \newcommand{\JavaTMFootnote}{}
%%   \newcommand{\LinuxRegisteredTMFootnote}{}

  %% Footers (Blank for this version.)
  \newcommand{\JikesTMFooter}{}
  \newcommand{\AIXTMFooter}{}
  \newcommand{\PowerPCTMFooter}{}
  \newcommand{\JavaTMFooter}{}
  \newcommand{\AIXPPCTMFooter}{}
  \newcommand{\JikesAIXTMFooter}{}
  \newcommand{\AIXPPCJikesTMFooter}{}

  
%% }{%% TM notices sprinkled everywhere. 
%%   % You could also use this compatibility definition:
%%   \newcommand{\JikesTM}{Jikes\JikesTMFootnote}
%%   \newcommand{\AIXTM}{AIX\AIXTMFootnote}
%%   \newcommand{\AIXTMheading}{AIX\trademark}
%%   \newcommand{\PowerPCTM}{PowerPC\PowerPCTMFootnote}
%%   \newcommand{\JavaTM}{Java\JavaTMFootnote}
%%   \newcommand{\LinuxR}{Linux\LinuxRFootnote}
%%   \newcommand{\LinuxRheading}{Linux\registeredTrademark}
%%   \newcommand{\SPECR}{SPEC\registeredTrademark}
%%   \newcommand{\SPECjvmR}{SPECjvm\registeredTrademark}
%%   \newcommand{\SPECjvmRheading}{SPECjvm\registeredTrademark}
%%   \newcommand{\SPECjbbR}{SPECjbb\registeredTrademark}
%%   \newcommand{\SPECjbbRheading}{SPECjbb\registeredTrademark}
%%   \newcommand{\RedHatTM}{Red Hat}
%%   \newcommand{\RedHatTMheading}{Red Hat}
%%   \newcommand{\SuSER}{SuSE}
%%   \newcommand{\SuSERheading}{SuSE}
%%   \newcommand{\IntelR}{Intel}
%%   \newcommand{\IntelRheading}{Intel}
%%   \newcommand{\IBMR}{IBM}
%%   \newcommand{\UnixR}{Unix}

%%   \newcommand{\JikesTMFootnote}{{\texonly \texttrademark}{\htmlonly
%%       \trademark}\texonly{\footnote{{\bf Jikes} is a
%%         trademark or registered trademark of International Business Machines
%%         Corporation in the United States, other countries, or both.}}}
  
%%   \newcommand{\JikesTMFooter}{\W \hline {\small {\bf Jikes} is a
%%       trademark or registered trademark of International Business Machines
%%       Corporation in the United States, other countries, or both.}}

%%   \newcommand{\AIXTMFootnote}{{\texonly \texttrademark}{\htmlonly
%%       \trademark}\texonly{\footnote{{\bf AIX} is a
%%         trademark or registered trademark of International Business Machines
%%         Corporation in the United States, other countries, or both.}}}
  
%%   \newcommand{\AIXTMFooter}{\W \hline {\small {\bf AIX} is a
%%       trademark or registered trademark of International Business Machines
%%       Corporation in the United States, other countries, or both.}}
  
%%   \newcommand{\PowerPCTMFootnote}{{\texonly \texttrademark}{\htmlonly
%%       \trademark}\texonly{\footnote{{\bf PowerPC} is a
%%         trademark or registered trademark of International Business Machines
%%         Corporation in the United States, other countries, or both.}}}
  
%%   \newcommand{\PowerPCTMFooter}{\W \hline {\small {\bf PowerPC} is a
%%       trademark or registered trademark of International Business Machines
%%       Corporation in the United States, other countries, or both.}}
  
%%   \newcommand{\JavaTMFootnote}{{\texonly {\texttrademark}}{\htmlonly
%%       \trademark}\texonly{\footnote{\small {\bf Java} and all Java-based
%%         trademarks and 
%%         logos are trademarks or registered trademarks of Sun Microsystems,
%%         Inc.\ in the United States, other countries, or both.}}}
  
%%   \newcommand{\JavaTMFooter}{\W \hline {\small {\bf Java} and all Java-based trademarks and
%%       logos are trademarks or registered trademarks of Sun Microsystems,
%%       Inc.\ in the United States, other countries, or both.}}
  
%%   \newcommand{\AIXPPCTMFooter}{\W \hline {\small {\bf AIX} and {\bf
%%         PowerPC} are
%%       trademarks or registered trademarks of International Business Machines
%%       Corporation in the United States, other countries, or both.}}
  
%%   \newcommand{\JikesAIXTMFooter}{\W \hline {\small {\bf Jikes} and {\bf AIX} are
%%       trademarks or registered trademarks of International Business Machines
%%       Corporation in the United States, other countries, or both.}}
  
%%   \newcommand{\AIXPPCJikesTMFooter}{\W \hline {\small {\bf AIX}, {\bf
%%         PowerPC}, and {\bf Jikes} are
%%       trademarks or registered trademarks of International Business Machines
%%       Corporation in the United States, other countries, or both.}}
  
%%   %% I do not know how to get the (R) symbol, so we'll substitute with
%%   %% TM for now.
%%   \newcommand{\LinuxRFootnote}{\registeredTrademark\texonly{\footnote{\small {\bf Linux} is a registered
%%         trademark of Linux Torvalds.}}}
%% }




%Define macros
%%  Releases
%%     Jan, 2001: 1.0, 1.0a
%%     Apr, 2001: 1.1
%%     Oct, 2001: 2.0.0
%%     Nov, 2001: 2.0.1
%%     Jan, 2002: 2.0.2
%%     Mar, 2002: 2.0.3
%%     Jun, 2002: 2.1.0
%%     Jul, 2002: 2.1.1
%%     Dec, 2002: 2.2.0
%%     Apr, 2003: 2.2.1
%%     Jun, 2003: 2.2.2
%%     Aug, 2003: 2.3.0
%%     Sep, 2003: 2.3.0.1
%%     Dec, 2003: 2.3.1
%%     Apr, 2004: 2.3.2
%%     Jul, 2004: 2.3.3
%%     Dec, 2004  2.3.4
%%     Apr, 2005  2.3.5

\newcommand{\version}{2.3.5}
\newcommand{\classpathversion}{0.14}
\newcommand{\jp}{Jalape\~{n}o}
\newcommand{\jrvm}{Jikes RVM}

\newcommand{\Indextt}[1]{\cindex[#1]{\texttt{#1}}}
\newcommand{\IndexTexttt}[1]{{\protect\texttt{#1}\cindex[#1]{\protect\texttt{#1}}}}
\newcommand{\IndexttClass}[1]{\cindex[#1]{\protect\texttt{#1} class}}
\newcommand{\ignore}[1]{{}}
\newcommand{\instruction}{\it}
\T\newcommand{\SectionName}[1]{{\sc #1}}
\W\newcommand{\SectionName}[1]{\textbf{#1}}
\newcommand{\varName}[1]{\texttt{#1}}

\newcommand{\RVMTarFile}{jikes\-r\-v\-m-{\version}\-.tar\-.gz}

%Define URLs
% definitions of all the URLs we want to include in the userguide.
% TODO: develop some tool for purging stuff that is not being used.

%\newcommand{\RVMHomeURL}{http://www.ibm.com/developerworks/oss/jikesrvm}
\newcommand{\dWossURL}{http://\-www.ibm.com/\-de\-vel\-o\-per\-works/\-oss}
\newcommand{\sFossURL}{http://\-jikesrvm.\-sourceforge.\-net}
\newcommand{\RVMHomeURL}{\sFossURL}
\newcommand{\RVMBugURL}{{\RVMHomeURL}/so\-lu\-tions}


\newcommand{\RVMBugTrackerURL}{{http://\-sourceforge.\-net/\-tracker/\-index.php}
\newcommand{\RVMBugURL}{\RVMHomeURL}/so\-lu\-tions}

\newcommand{\JalapenoHomeURL}{http://\-www.research.ibm.com/\-ja\-la\-pe\-no}
\newcommand{\QandAURL}{{\RVMHomeURL}/\-info/\-overview.shtml}
\newcommand{\RVMPubsURL}{{\RVMHomeURL}/\-info/\-pubs.shtml}
\newcommand{\RVMUsersPubsURL}{{\RVMHomeURL}/\-info/\-users-pubs.shtml}
\newcommand{\RVMUsersURL}{{\RVMHomeURL}/\-info/\-users.shtml}
\newcommand{\RVMSlidesURL}{{\RVMHomeURL}/\-info/\-pre\-sen\-ta\-tions.shtml}
\newcommand{\RVMUserGuideURL}{{\RVMHomeURL}/userguide/HTML/userguide.html}
\newcommand{\RVMJavadocURL}{{\RVMHomeURL}/api}

\newcommand{\RVMDownloadURL}{{\RVMHomeURL}/download}
\newcommand{\KaffeURL}{http://www.kaffe.org}
\newcommand{\RVMCVSURL}{{\RVMHomeURL}/cvsweb.html}
%\newcommand{\RVMUserListURL}{{\RVMHomeURL}/info/users.shtml}

\newcommand{\RVMMailingListURL}{\sFossURL/\-mail/?group_id=128805}
\newcommand{\RVMmailArchiveURL}{http:\\sourceforge.net/\-mailarchive}
\newcommand{\RVMAnnounceMailingListURL}{\RVMmailArchiveURL/\-forum.php/\-forum_id\-=43942}
\newcommand{\RVMCoreMailingListURL}{\RVMmailArchiveURL/\-forum.php/\-forum_id\-=43940}
\newcommand{\RVMCommitMailingListURL}{\RVMmailArchiveURL/\-forum.php/\-forum_id\-=44096}
\newcommand{\RVMRegressionMailingListURL}{\RVMmailArchiveURL/\-forum.php/\-forum_id\-=43938}
\newcommand{\RVMResearcherMailingListURL}{\RVMmailArchiveURL/\-forum.php/\-forum_id\-=43937}
% Ugliness to remove _ in latex output
\newcommand{\RVMMailingListURLtext}{\sFossURL/\-mail/?group\_id=128805}
\newcommand{\RVMAnnounceMailingListURLtext}{\RVMmailArchiveURL/\-forum.php/\-forum\_id\-=43942}
\newcommand{\RVMCoreMailingListURLtext}{\RVMmailArchiveURL/\-forum.php/\-forum\_id\-=43940}
\newcommand{\RVMCommitMailingListURLtext}{\RVMmailArchiveURL/\-forum.php/\-forum\_id\-=44096}
\newcommand{\RVMRegressionMailingListURLtext}{\RVMmailArchiveURL/\-forum.php/\-forum\_id\-=43938}
\newcommand{\RVMResearcherMailingListURLtext}{\RVMmailArchiveURL/\-forum.php/\-forum\_id\-=43937}

\newcommand{\RVMResearcherMailingListArchiveURL}{\RVMResearcherMailingListURL}

\newcommand{\RVMTeachingResourcesURL}{{\RVMHomeURL}\-/\-in\-fo\-/\-course-in\-fo\-.shtml}
\newcommand{\RVMContribURL}{{\RVMHomeURL}\-/\-in\-fo\-/\-con\-tri\-bu\-tions\-.shtml}


%\newcommand{\CPLURL}{http://oss.software.ibm.com/developerworks/opensource/license-cpl.html}
\newcommand{\CPLURL}{{\sFossURL}/licenses/CPLv1.0.htm}
% I don't know what to map the following to. 
%\newcommand{\CPLFAQURL}{http://www-106.ibm.com/developerworks/library/os-cplfaq.html}

\newcommand{\IBMURL}{http://www.ibm.com}
\newcommand{\WatsonURL}{http://www.watson.ibm.com}
\newcommand{\SPECURL}{http://www.spec.org}
\newcommand{\SOOTURL}{http://www.sable.mcgill.ca/soot}
\newcommand{\classpathURL}{http://www.gnu.org/software/classpath}
\newcommand{\classpathLicenseURL}{http://www.gnu.org/software/classpath/license.html}
\newcommand{\classpathftp}{ftp://ftp.gnu.org/pub/gnu/classpath}
\newcommand{\xalanURL}{http://xml.apache.org/xalan-j/}
\newcommand{\xalanFile}{xalan-j\_2\_6\_0-bin.tar.gz}
\newcommand{\osiURL}{http://www.opensource.org}
\newcommand{\JDWPURL}{http://java.sun.com/j2se/1.4.2/docs/guide/jpda/jdwp-spec.html}
\newcommand{\OSXURL}{http://directory.google.com/Top/Computers/Software/Operating\_Systems/Mac\_OS/System\_Software/Mac\_OS\_X/?tc=1}
\newcommand{\DebianFreeSoftwareGuidelinesURL}{http://www.debian.org/social_contract\#guidelines}
\newcommand{\DebianURL}{http://www.debian.org}
\newcommand{\DebianWhatIsFreeSoftwareURL}{http://www.debian.org/intro/free}


% package com.ibm.JikesRVM
\newcommand{\JikesRVMpackage}{com\-.ibm\-.Jikes\-RVM}
\newcommand{\JikesRVMJavadocURL}{{\RVMJavadocURL}/com/ibm/JikesRVM}
\newcommand{\VMURL}{\JikesRVMJavadocURL/VM.html}
\newcommand{\VMAbstractThreadQueueURL}{{\JikesRVMJavadocURL}/VM\_AbstractThreadQueue.html}
\newcommand{\VMCallbacksURL}{{\JikesRVMJavadocURL}/VM\_Callbacks.html}
\newcommand{\VMIdleThreadURL}{{\JikesRVMJavadocURL}/VM\_IdleThread.html}
\newcommand{\VMJavaHeaderURL}{{\JikesRVMJavadocURL}/VM\_JavaHeader.html}
\newcommand{\VMLockURL}{{\JikesRVMJavadocURL}/VM\_Lock.html}
\newcommand{\VMMagicURL}{{\JikesRVMJavadocURL}/VM\_Magic.html}
\newcommand{\VMCompilerURL}{{\JikesRVMJavadocURL}/VM\_Compiler.html}
\newcommand{\VMMiscHeaderURL}{{\JikesRVMJavadocURL}/VM\_MiscHeader.html}
\newcommand{\VMObjectModelURL}{{\JikesRVMJavadocURL}/VM\_ObjectModel.html}
\newcommand{\VMProcessorURL}{{\JikesRVMJavadocURL}/VM\_Processor.html}
\newcommand{\VMProcessorLockURL}{{\JikesRVMJavadocURL}/VM\_ProcessorLock.html}
\newcommand{\VMProxyURL}{{\JikesRVMJavadocURL}/VM\_Proxy.html}
\newcommand{\VMRuntimeCompilerURL}{{\JikesRVMJavadocURL}/VM\_RuntimeCompiler.html}
\newcommand{\VMSynchronizationURL}{{\JikesRVMJavadocURL}/VM\_Synchronization.html}
\newcommand{\VMSysCallURL}{{\JikesRVMJavadocURL}/VM\_SysCall.html}
\newcommand{\VMThinLockURL}{{\JikesRVMJavadocURL}/VM\_ThinLock.html}
\newcommand{\VMThreadURL}{{\JikesRVMJavadocURL}/VM\_Thread.html}
\newcommand{\VMThreadthreadSwitchURL}{{\JikesRVMJavadocURL}/VM\_Thread.html\#threadSwitch(int)}
\newcommand{\VMTIBLayoutConstantsURL}{{\JikesRVMJavadocURL}/VM\_TIBLayoutConstants.html}
\newcommand{\VMUninterruptibleURL}{{\JikesRVMJavadocURL}/VM\_Uninterruptible.html}

% package com.ibm.JikesRVM.adaptive
\newcommand{\AdaptiveJavadocURL}{{\JikesRVMJavadocURL}/adaptive}
\newcommand{\VMAIByEdgeOrganizerURL}{{\AdaptiveJavadocURL}/VM\_AIByEdgeOrganizer.html}
\newcommand{\VMAOSDatabaseURL}{{\AdaptiveJavadocURL}/VM\_AOSDatabase.html}
\newcommand{\VMAOSLoggingURL}{{\AdaptiveJavadocURL}/VM\_AOSLogging.html}
\newcommand{\VMCounterArrayManagerURL}{{\AdaptiveJavadocURL}/VM\_CounterArrayManager.html}
\newcommand{\VMMethodSampleOrganizerURL}{{\AdaptiveJavadocURL}/VM\_MethodSampleOrganizer.html}
\newcommand{\VMControllerURL}{{\AdaptiveJavadocURL}/VM\_Controller.html}
\newcommand{\VMControllerThreadURL}{{\AdaptiveJavadocURL}/VM\_ControllerThread.html}
\newcommand{\VMCompilationThreadURL}{{\AdaptiveJavadocURL}/VM\_Com\-pi\-la\-tion\-Thread.html}
\newcommand{\VMCompilerDNAURL}{{\AdaptiveJavadocURL}/VM\_CompilerDNA.html}
\newcommand{\VMEdgeListenerURL}{{\AdaptiveJavadocURL}/VM\_EdgeListener.html}
\newcommand{\VMInstrumentationURL}{{\AdaptiveJavadocURL}/VM\_Instrumentation.html}
\newcommand{\VMManagedCounterDataURL}{{\AdaptiveJavadocURL}/VM\_ManagedCounterData.html}
\newcommand{\VMMethodInvocationCounterDataURL}{{\AdaptiveJavadocURL}/VM\_MethodInvocationCounterData.html}
\newcommand{\VMMethodListenerURL}{{\AdaptiveJavadocURL}/VM\_MethodListener.html}
\newcommand{\VMStringEventCounterDataURL}{{\AdaptiveJavadocURL}/VM\_StringEventCounterData.html}
%\newcommand{\VMStringCounterEventDataURL}{{\AdaptiveJavadocURL}/VM\_StringCounterEventData.html}
\newcommand{\VMYieldpointCounterDataURL}{{\AdaptiveJavadocURL}/VM\_YieldpointCounterData.html}

% package com.ibm.JikesRVM.classloader
\newcommand{\ClassloaderJavadocURL}{{\JikesRVMJavadocURL}/classloader}
\newcommand{\VMClassURL}{{\ClassloaderJavadocURL}/VM\_Class.html}
\newcommand{\VMInterfaceInvocationURL}{{\ClassloaderJavadocURL}/VM\_InterfaceInvocation.html}
\newcommand{\VMMethodURL}{{\ClassloaderJavadocURL}/VM\_Method.html}

% package com.ibm.JikesRVM.memoryManagers
\newcommand{\MMpackage}{\JikesRVMpackage{}.me\-mo\-ry\-Ma\-na\-gers}
\newcommand{\MMJavadocURL}{{\JikesRVMJavadocURL}/memoryManagers}

% package com.ibm.JikesRVM.memoryManagers.JMTk
\newcommand{\JMTkJavadocURL}{{\MMJavadocURL}/JMTk}
\newcommand{\JMTkPackageURL}{{\JMTkJavadocURL}/package-frame.html}
\newcommand{\AddressDequeURL}{{\JMTkJavadocURL}/AddressDeque.html}
\newcommand{\SegregatedFreeListURL}{{\JMTkJavadocURL}/SegregatedFreeList.html}
\newcommand{\BasePlanURL}{{\JMTkJavadocURL}/BasePlan.html}
\newcommand{\BumpPointerURL}{{\JMTkJavadocURL}/BumpPointer.html}
\newcommand{\CopyURL}{{\JMTkJavadocURL}/Copy.html}
\newcommand{\FreeListVMResourceURL}{{\JMTkJavadocURL}/FreeListVMResource.html}
\newcommand{\GenerationalURL}{{\JMTkJavadocURL}/Generational.html}
\newcommand{\LazyMmapperURL}{{\JMTkJavadocURL}/LazyMmapper.html}
\newcommand{\LocalSSBURL}{{\JMTkJavadocURL}/LocalSSB.html}
\newcommand{\MemoryResourceURL}{{\JMTkJavadocURL}/MemoryResource.html}
\newcommand{\MonotoneVMResourceURL}{{\JMTkJavadocURL}/MonotoneVMResource.html}
\newcommand{\MarkSweepLocalURL}{{\JMTkJavadocURL}/MarkSweepLocal.html}
\newcommand{\MarkSweepSpaceURL}{{\JMTkJavadocURL}/MarkSweepSpace.html}
\newcommand{\RawPageAllocatorURL}{{\JMTkJavadocURL}/RawPageAllocator.html}
\newcommand{\SharedDequeURL}{{\JMTkJavadocURL}/SharedDeque.html}
\newcommand{\StopTheWorldGCURL}{{\JMTkJavadocURL}/StopTheWorldGC.html}
\newcommand{\VMResourceURL}{{\JMTkJavadocURL}/VMResource.html}

% package com.ibm.JikesRVM.memoryManagers.mmInterface
\newcommand{\mmInterface}{m\-m\-In\-ter\-face}
\newcommand{\VMInterfaceJavadocURL}{{\MMJavadocURL}/\mmInterface}
\newcommand{\mmInterfacePackageURL}{{\VMInterfaceJavadocURL}/package-frame.html}
\newcommand{\ScanObjectURL}{{\VMInterfaceJavadocURL}/ScanObject.html}
\newcommand{\ScanStaticsURL}{{\VMInterfaceJavadocURL}/ScanStatics.html}
\newcommand{\ScanThreadURL}{{\VMInterfaceJavadocURL}/ScanThread.html}
\newcommand{\VMAllocatorHeaderURL}{{\VMInterfaceJavadocURL}/VM\_AllocatorHeader.html}
\newcommand{\VMCollectorThreadURL}{{\VMInterfaceJavadocURL}/VM\_CollectorThread.html}
\newcommand{\VMGCMapIteratorURL}{{\VMInterfaceJavadocURL}/VM\_GCMapIterator.html}
\newcommand{\VMHandshakeURL}{{\VMInterfaceJavadocURL}/VM\_Handshake.html}
\newcommand{\VMInterfaceURL}{{\VMInterfaceJavadocURL}/VM\_Interface.html}

% package com.ibm.JikesRVM.opt
\newcommand{\OptJavadocURL}{{\JikesRVMJavadocURL}/opt}
\newcommand{\OPTCompilationPlanURL}{{\OptJavadocURL}/OPT\_CompilationPlan.html}
\newcommand{\OPTCompilerPhaseURL}{{\OptJavadocURL}/OPT\_CompilerPhase.html}
\newcommand{\OPTDefUseURL}{{\OptJavadocURL}/OPT\_DefUse.html}
\newcommand{\OPTInsertInstructionCountersURL}{{\OptJavadocURL}/OPT\_InsertInstructionCounters.html}
\newcommand{\OPTInsertYieldpointCountersURL}{{\OptJavadocURL}/OPT\_InsertYieldpointCounters.html}
\newcommand{\OPTInsertMethodInvocationCounterURL}{{\OptJavadocURL}/OPT\_InsertMethodInvocationCounter.html}
\newcommand{\OPTInstrumentedEventCounterManagerURL}{{\OptJavadocURL}/OPT\_InstrumentedEventCounterManager.html}
\newcommand{\OPTOptimizationPlanElementURL}{{\OptJavadocURL}/OPT\_OptimizationPlanElement.html}
\newcommand{\OPTOptimizationPlannerURL}{{\OptJavadocURL}/OPT\_OptimizationPlanner.html}
\newcommand{\OPTSSAURL}{{\OptJavadocURL}/OPT\_SSA.html}
\newcommand{\OPTSimpleEscapeURL}{{\OptJavadocURL}/OPT\_SimpleEscape.html}

% package com.ibm.JikesRVM.opt.ir
\newcommand{\IRJavadocURL}{{\OptJavadocURL}/ir}
\newcommand{\OPTGenerateMagicURL}{{\IRJavadocURL}/OPT\_GenerateMagic.html}
\newcommand{\OPTGenerateMachineSpecificMagicURL}{{\IRJavadocURL}/OPT\_GenerateMachineSpecificMagic.html}
\newcommand{\OPTInlinerURL}{{\IRJavadocURL}/OPT\_Inliner.html}
\newcommand{\OPTInstructionURL}{{\IRJavadocURL}/OPT\_Instruction.html}
\newcommand{\OPTRegisterURL}{{\IRJavadocURL}/OPT\_Register.html}

% incomplete URLs or stray files
\newcommand{\PPCStackframeLayoutURL}{{\RVMJavadocURL}/finishURL}
\newcommand{\LintelStackframeLayoutURL}{{\RVMJavadocURL}/finishURL}
\newcommand{\PPCRegisterConstantsURL}{{\RVMJavadocURL}/finishURL}
\newcommand{\LintelRegisterConstantsURL}{{\RVMJavadocURL}/finishURL}



\T \hyphenation{Class-path}
\T \hyphenation{a-vai-la-ble}
\T \hyphenation{Post-Script}
\T \hyphenation{li-bra-ries}
\T \hyphenation{di-rec-to-ry}
\T \hyphenation{SPEC-jvm}
\T \hyphenation{SPEC-jbb}
% \T \hyphenation{RVM\_-ROOT}
\T \hyphenation{CLASS-PATH}
\T \hyphenation{j-con-fi-gure}
\T \hyphenation{Run-Sa-ni-ty-Tests}
\T \hyphenation{Pow-er-PC}
\newcommand{\gnuMakeURL}{http:\-//\-www\-.gnu\-.org\-/\-software\-/\-make}
\newcommand{\gnuTarURL}{http:\-//\-www\-.gnu\-.org\-/\-soft\-ware\-/\-tar\-/\-tar\-.html}
%\newcommand{\jikesURL}{http://oss.software.ibm.com/developerworks/opensource/jikes}
\newcommand{\jikesURL}{http:\-//\-oss\-.soft\-ware\-.ibm\-.com\-/\-de\-ve\-l\-o\-per\-works\-/\-op\-en\-source\-/\-jikes}
\newcommand{\HyperlatexURL}{http:\-//www\-.cs\-.ruu\-.nl\-/\-\~{}ot\-fried\-/\-Hy\-per\-la\-tex/}
\newcommand{\AIXJdkURL}{http://www.ibm.com/\-java/\-jdk/\-aix/\-in\-dex.html}
\newcommand{\LinuxJdkURL}{http://www.ibm.com/\-java/\-jdk/\-aix/\-in\-dex.html}
\newcommand{\BlackdownURL}{http:\-//\-www\-.black\-down\-.org}
\newcommand{\SystemJournalPaperURL}{{\RVMPubsURL}\#ibmsj00}
\newcommand{\JavaGrandePaperURL}{{\RVMPubsURL}\#grande99}
\newcommand{\OOPSLAPaperURL}{{\RVMPubsURL}\#oopsla00\_aos}
\newcommand{\EscapeAnalysisPaperURL}{{\RVMPubsURL}\#oopsla99\_escape}
\newcommand{\SASPaperURL}{{\RVMPubsURL}\#sas00}
\newcommand{\ABCDPaperURL}{{\RVMHomeURL}\#ABCD}
\newcommand{\SunCodeConventionURL}{http://java.sun.com/docs/codeconv}
\newcommand{\JavadocURL}{http://\-ja\-va.sun.com/\-j2se/\-java\-doc/\-in\-dex.html}
\newcommand{\VimURL}{http://\-www.vim.org}
\newcommand{\DejavuURL}{{\JalapenoHomeURL}/\-dejavu/\-index.html}
\newcommand{\QuicksilverURL}{http://www.research.ibm.com/people/g/gupta/oopsla00.ps}
\newcommand{\unzipURL}{http://www.info-zip.org/\-pub/\-infozip/\-UnZip.html}

%% Html declarations: Output directory and filenames, node title
%\htmlcss{http://www.w3.org/StyleSheets/Core/Steely}
%\htmltitle{The Jikes\trademark Research Virtual Machine User's Guide}
\htmltitle{Jikes RVM User's Guide}
%\htmltitle{Jikes RVM User's Guide}
\W \htmldirectory{HTML}
\W \renewcommand{\HlxIcons}{}
\W \renewcommand{\htmlpanelfield}[3][\link]{\HlxAppend{\HlxPanelFields}%
  {\xml*{td bgcolor="###99ccff" align="center"}%
    #1{#2}{#3}\xml*{/td}}}
%\W \htmlpanelfield[\xlink]{Home}{http://www-124.ibm.com/developerworks/oss/jikesrvm" target="\_top}
%\W \htmlpanelfield[\xlink]{Project}{http://www-124.ibm.com/developerworks/projects/jikesrvm" target="\_top}
\W \htmlpanelfield[\xlink]{Home}{http://sourceforge.jikesrvm.net" target="\_top}
\W \htmlpanelfield[\xlink]{Project}{http://sourceforge.jikesrvm.net" target="\_top}
\W \htmlpanelfield{\contentsname}{hlxtoc}
\W \htmlpanelfield{\indexname}{hlxindex}
%\setcounter{secnumdepth}{3}
\setcounter{htmldepth}{3}
\xmlattributes{table}{border}
%\xmlattributes{ul}{compact}
\xmlattributes{body}{bgcolor="#eeeeee"}
\xmlattributes{a}{style="text-decoration: none"}

\W\begin{iftex}
% \sloppy
%% These definitions work reasonably for A4 and letter paper
\oddsidemargin 0mm
\evensidemargin 0mm
\topmargin 0mm
%\textwidth 10cm
%\textheight 8in
\textwidth 15cm
\textheight 22cm
\advance\textheight by -\topskip
\count255=\textheight\divide\count255 by \baselineskip
\textheight=\the\count255\baselineskip
\advance\textheight by \topskip
\W\end{iftex}


% Use \remark{...text...} to add a remark (todo item).
\T \newif\ifremark
\long\def\remark#1{
\ifremark%
        \begingroup%
        \dimen0=\columnwidth
        \advance\dimen0 by -1in%
        \setbox0=\hbox{\parbox[b]{\dimen0}{\protect\em #1}}
        \dimen1=\ht0\advance\dimen1 by 2pt%
        \dimen2=\dp0\advance\dimen2 by 2pt%
        \vskip 0.25pt%
        \hbox to \columnwidth{%
                \vrule height\dimen1 width 3pt depth\dimen2%
                \hss\copy0\hss%
                \vrule height\dimen1 width 3pt depth\dimen2%
        }%
        \endgroup%
\fi}

\T \remarktrue
\W \newcommand{\remark}[1]{}

\title{\texonly{\vfill} {\huge The Jikes\TMboth{} 
Research Virtual Machine \\
User's Guide} 
% \\ {\huge Post \version{} (CVS Head)}\\ { } \texonly{\vfill} }
\\ {\huge Version \version{}}\\ { } \texonly{\vfill} }

\makeindex

% \newcommand{\Xname}[1]{\basename{#1}\xname{#1}}
%% Failed experiment:
% \newcommand{\InputSec}[2]{\section{#1}\xname{#2}\htmlname{#2}\input{#2}\htmlname{userguide}}

\begin{document}

\maketitle
\date{}

\T \newpage
%% \T \ifthenelse{\boolean{TMOnOwnPage}}{
   \T \section*{Trademarks:}
\label{trademarks}

{\bf \JavaTM} and all \JavaTM{}-based trademarks and logos are trademarks or
registered trademarks of Sun Microsystems, Inc.

{\bf \LinuxR} is a registered trademark of Linus Torvalds.

{\bf \AIXTM}, {\bf \PowerPCTM}, {\bf \IBMR}, and {\bf \JikesTM} are
trademarks or 
registered trademarks of International Business Machines Corporation in the
United States, other countries, or both.

{\bf \UnixR} is a registered trademark in the United States and other
countries, exclusively licensed through X/Open Company, Ltd.

{\bf \SPECjvmR}, {\bf \SPECjbbR}, and {\bf \SPECR} are registered
trademarks of The Standard Performance Evaluation Corporation (SPEC),
a non-profit corporation. 

{\bf \RedHatTM} is a trademark or registered trademark of Red Hat, Inc.

{\bf \SuSER} is a registered trademark of SuSE Linux.

{\bf \IntelR} is a registered trademark of Intel Corporation.

Other product names mentioned herein may be the trademarks or
registered trademarks of their respective owners.


   \T \cleardoublepage
%% \T }{}

\label{hlxtoc}
\T \tableofcontents
\T \listoffigures
\T \listoftables
% This does nothing.
% \xname{contents}
\W \htmlmenu{6}

\T \newpage
%% Failed experiment:
%%\InputSec{Introduction}{intro}
\xname{intro}
\section{Introduction}
% \W \htmldirectory{HTML/intro}
%\htmlname{intro}
This section provides an overview of Jikes\TMweb{} RVM as well as
information about how best to use this document.

\subsection{Welcome to Jikes RVM}

Jikes\TMboth{} RVM is a Research Virtual Machine 
 developed at the 
\xlink{IBM}{\IBMURL}\Rboth{}
\xlink{T.J.\ Watson Research Center}{\WatsonURL}.  Key
features of the system include
\begin{itemize}
\item the entire virtual machine (VM) is implemented in the
  Java\TMboth{} programming language,
\item the VM utilizes two compilers and no interpreter,\footnote{The
Quick Compiler is an experimental third compiler}
\item a family of parallel, type-exact garbage collectors,
\item a lightweight thread package with compiler-supported preemption,
\item an aggressive optimizing compiler, and 
\item a flexible online adaptive compilation infrastructure.
\end{itemize}

A significant body of information about Jikes RVM 
(formerly known as 
\xlink{\jp}{\JalapenoHomeURL}) appears 
in our published
papers.  For overviews of the system's structure, including the runtime system,
optimizing compiler, and adaptive systems, \xlink{see the list of published papers}[, available from the \jrvm{} web page:
\begin{quote}
\texttt{\RVMPubsURL}
\end{quote}
]{\RVMPubsURL}.

The best paper for a general introduction to \jrvm{} is the 
\xlink{IBM Systems Journal, January 2000
paper
\T~\cite{jalapeno-ibmsj-00}
}{\SystemJournalPaperURL}.  
For introductions to the
optimizing compiler and adaptive system, see the 
\xlink{1999 ACM Java Grande\begin{iftex}~\cite{jalapeno-opt-grande-99}\end{iftex}}
{\JavaGrandePaperURL}
 and 
 \xlink{2000 OOPSLA\begin{iftex}~\cite{jalapeno-adaptive-00}\end{iftex}}
{\OOPSLAPaperURL}  
papers, respectively.

We have given several tutorials on Jikes RVM that you may find
useful. The PACT'01 tutorial covers all of Jikes RVM.\@  The PLDI'02 and
OOPSLA'02 tutorials focus on the optimizing compiler.  The tutorial
slides are available at
\begin{quote}
\xlink{{\tt \RVMSlidesURL}}{\RVMSlidesURL}
\end{quote}

Jikes RVM is a bleeding-edge research project.  You will find that
some of the code does not live up to product quality standards.  Don't
hesitate to help rectify this by contributing clean-ups, bug fixes,
and missing documentation to the project.

Many academic groups have adopted Jikes RVM as their primary research
infrastructure, resulting in
\xlink{publications}{\RVMUsersPubsURL}
in leading
conferences, such as PLDI, POPL, OOPSLA, and SIGMETRICS.\@  In the first year
of its open source release, the VM was downloaded by over
1500 unique IP addresses, including over ninety
universities around the world. A list
of some of the \xlink{users}{\RVMUsersURL} is available.
\xlink{Teaching resources}{\RVMTeachingResourcesURL} using
Jikes RVM are also available.

\subsection{About this document}

This document provides Jikes\TMweb{} RVM information that is not
covered in our published papers.  For high-level overviews,
algorithms, and structures, you will find the published papers to be
the best starting place. This document supplements the Jikes RVM
papers, focusing on implementation details of how to build, run,
and add functionality to the system.

This document is available as both PostScript\Rboth{} and HTML.\@  \xlink{You will find the
HTML version more useful, as it includes hyperlinks}[.  The HTML version is
available at:
\begin{example}
\RVMUserGuideURL
\end{example}
].

Each released Jikes RVM tarball holds a PostScript version of this
document in its top directory, as ``{\tt userguide.ps}''.


The \jrvm{} web page includes Javadoc\TMboth{}
\xlink{API documentation}{\RVMJavadocURL}. 
This HTML should be the primary reference for individual classes. The
level of detail provided in the Javadoc is highly variable, but most
Jikes RVM classes have at least a minimal description.

You may find sections of this user's guide missing, incomplete or
otherwise confusing. We intend this document to live as a continual
work-in-progress, hopefully growing and maturing as community members
edit and add to the guide.  Please accept this invitation to
contribute.

Please send feedback, bug fixes, and text contributions to the 
\xlink{{\tt JikesRVM-researchers} mailing list}{\RVMResearcherMailingListURL}.  
Constructive criticism will be cheerfully accepted. 



\htmlname{userguide}
% \W \htmldirectory{HTML}
\T \newpage
\xname{installation}
\section{Installation Guide} \label{section:installation}
% \W \htmldirectory{HTML/installation}
So you've gotten the Jikes\trademark RVM distribution.  Now what?
This section gives 
instructions on how to install and run the system.

\subsection{System Prerequisites}
To build on any platform you will need the following:
\begin{itemize}
\item GNU make. You can download this from
\xlink{{\tt \gnuMakeURL}} {\gnuMakeURL}.

\item The Jikes Java\trademark compiler 
You can download this from
\xlink{{\tt \jikesURL}} {\jikesURL}.
You can use any of the pre-compiled binaries, or build it yourself from the
source. We recommend version 1.13; users have reported difficulties with 
version 1.14.  We have not yet tried version 1.15.

\item GNU tar is needed to extract the distribution tar file.  
You can download this from
\xlink{{\tt \gnuTarURL}} {\gnuTarURL}.
We have experienced problems with the AIX\AIXTMFootnote tar program
truncating file names. 

\item unzip. 
You can download an {\tt unzip} implementation from
\xlink{{\tt \unzipURL}} {\unzipURL}.

\end{itemize}

%% footnotes not allowed in section headings, so we specialize
\htmlonly{\subsubsection{AIX\trademark/PowerPC\trademark}}
\texonly {\subsubsection{AIX\trademark/PowerPC\trademark}}

In addition to the software mentioned above, to install, build, and
run the Jikes RVM on a AIX/PowerPC environment, 
you will need 
\begin{itemize}
\item a PowerPC processor
\item AIX 4.3 or later, and
\item (recommended) 512MB of memory.
\item the IBM AIX Developer Kit for Java version 1.3.0.  You can
download this from \xlink{{\tt \AIXJdkURL}} {\AIXJdkURL}. 
\item a C/C++ compiler.  We recommend {\tt gcc v2.95.3}; it's been
reported that other gcc versions on AIX don't work.  It's also been
reported that IBM's {\tt xlC} works.
\end{itemize}

\subsubsection{Linux/IA32}
In addition to the software mentioned above, to install, build, and
run the Jikes RVM on a Linux Intel environment, you will need 
\begin{itemize}
\item an Intel Architecture 32 bit processor
\item Linux  (we have run on RedHat 6.0 and 7.0)
\item The system has run in 386MB; we have not established a lower bound.
\item the IBM Developer Kit (at least version 2.1.3).  You can
download this from 
\xlink{{\tt \LinuxJdkURL}} {\LinuxJdkURL}.  We did have problems on
bleeding edge 2.4 kernels running on SMP machines with versions of the IBM
Developer Kit built prior to May 2001.  We are currently using IBM
build cx130-20010502.  We have seen problems using the IBM 1.3.1 JDK;
avoid it until further notice.
\item {\tt ksh}. This must be installed in {\tt /bin/ksh}.
\end{itemize}

\subsubsection{Win32 and Cygwin/IA32}
Currently this option for running the Jikes RVM is barely, if at all,
functional. 
It is not currently being maintained or tested. You are encouraged to 
skip this section and don't even think about running the Jike RVM on
Win32 with  
cygwin.

If you're still reading, you are probably an enterprising hacker that is
interested in getting this configuration up an running.  More power to
you! We'd like to hear your progress.

In addition to the software mentioned above, to install, build, and
run the Jikes RVM on a Win32/Cygwin Intel environment, you will need 
\begin{itemize}
\item an Intel Architecture 32 bit processor
\item a Win32 operating system (we have only tried it on Windows 2000).
\item Cygwin. You can download this from 
\xlink{{\tt \CygwinURL}} {\CygwinURL}. (We used dll version 1.1.8).
\item The system has run in 256MB on Windows 2000; we have not established
a lower bound.
\item The IBM Developer Kit (at least version 2.1.3).  You can download this from
\xlink{{\tt \WinJdkURL}} {\WinJdkURL}.
\item {\tt ksh}.  One alternative is {\tt pdksh} from 
\xlink{{\tt \pdkshURL}} {\pdkshURL}.
\end{itemize}

Some issues with this platform include:

\begin{itemize}
\item We haven't managed to successfully install a handler for hardware traps,
so the first hardware exception (e.g., a {\tt NullPointerException}) will cause
an ugly core dump.
\item Spaces in directory names aren't handled by jconfigure and the
other scripts.  Use cygwin's mount command to avoid them (in
particular the config file assumes that
C:$\backslash$Program~Files$\backslash$IBM is mounted as /IBM).
\end{itemize}

\W \AIXPPCJikesTMFooter

\W \JavaTMFooter

\subsection{Installation Overview}

To install and build the Jikes\trademark RVM, you will need to acquire
the following 
items from the Jikes RVM
\xlink{download}{\RVMDownloadURL} page.
\begin{itemize}
\item The Jikes RVM source distribution.  This is available as a compressed tar
file {\tt \RVMTarFile}.  You can also work with the contents of this repository with CVS from the
\xlink{public repository}{\RVMCVSURL}.
\item The Jikes RVM standard libraries.  This is a file {\tt \LibTarFile}
available from the download page. 
\end{itemize}

After downloading these files, you will set up 
a working directory holding the Jikes RVM source files, standard
library jar, and  
tools needed to build Jikes RVM. 

JIkes RVM can be configured in various ways. Multiple versions of the system,
corresponding to different configurations, can be generated from 
one working directory. See Section~\ref{configs} for information about the 
various 
configurations.
\index{configurations}
The Jikes RVM  {\em boot image} and other files generated during the 
configuration process
\index{boot image}
are stored in a {\em build directory}, which is logically separate from 
the working directory. 
\index{build directory}

To install Jikes RVM  you must do the following:
\begin{enumerate}
\item Set up a working directory.
\item Set various environment variables.
\item Edit Jikes RVM environment scripts.
\item Choose a configuration and run the configuration script to write
the appropriate directory and configuration specific files to the
build directory.
\item Build an executable version of Jikes RVM.
\end{enumerate}

The remainder of this section describes the process in greater detail.

\W \JikesTMFooter

\subsection{Installation Steps}

\begin{enumerate}
\item {\bf Set up a working directory.}

First extract the RVM source distribution into a directory such as 
{\tt \$HOME/rvmRoot}.
\begin{verbatim}
% cd $HOME
% mkdir rvmRoot
% cd rvmRoot
% zcat jikesrvm-[version].tar.gz | tar xvf - 
\end{verbatim}

Next extract the RVM standard libraries.  The following installs the
standard library file {\tt rvmrt.jar} in {\tt \$RVM\_ROOT/support/lib}.
\begin{verbatim}
% cd $HOME/rvmRoot
% zcat jlibraries-[version].tar.gz | tar xvf - 
\end{verbatim}

\index{environment variables}
\index{RVM\_ROOT}
\index{RVM\_BUILD}
\index{PATH}
\item {\bf Set up environment variables.}

You need to set up the following shell environment variables:

\begin{description}
\item [{\tt RVM\_ROOT}] the directory that contains the extracted
distribution 
\item [{\tt RVM\_BUILD}] the directory where you would like the build
process to generate an executable RVM configuration

\item [{\tt RVM\_HOST\_CONFIG}] the configuration file used to specify
the software environment on which the system is generated; i.e., where the
boot image is generated.

\item [{\tt RVM\_TARGET\_CONFIG}] the configuration file used to specify
the software environment where the system support is generated; i.e., where
the ``booter'' and ``C runtime'' will be generated.

\item[{\tt PATH}] your path should contain {\tt \$RVM\_ROOT/rvm/bin} in
order to pick up various scripts and utilities
\end{description}

We recommend you set up these variables in your shell configuration
file.  For example, for {\tt csh}, you might insert the
following into your {\tt .cshrc} file:

\begin{verbatim}
setenv RVM_ROOT $HOME/rvmRoot       # <--define your working directory 
setenv RVM_BUILD $HOME/rvmBuild     # <--define your current build directory 
setenv PATH $RVM_ROOT/rvm/bin:$PATH
setenv RVM_HOST_CONFIG $RVM_ROOT/rvm/config/powerpc-ibm-aix4.3.3.0
setenv RVM_TARGET_CONFIG $RVM_ROOT/rvm/config/powerpc-ibm-aix4.3.3.0
\end{verbatim}

{\em Note:} You should define each of these environment variables as an
{\em absolute} path.  The builder template expansion process will crash
and burn if you use a {\tt ..} in these paths.

For a Linux-Intel environment, the exports
would be replaced with the following:

\begin{verbatim}
setenv RVM_HOST_CONFIG $RVM_ROOT/rvm/config/i686-pc-linux-gnu
setenv RVM_TARGET_CONFIG $RVM_ROOT/rvm/config/i686-pc-linux-gnu
\end{verbatim}

For a Cygwin-Intel environment, the  exports
would be replaced with the following:

\begin{verbatim}
setenv RVM_HOST_CONFIG $RVM_ROOT/rvm/config/i686-pc-cygwin
setenv RVM_TARGET_CONFIG $RVM_ROOT/rvm/config/i686-pc-cygwin
\end{verbatim}

These two variables point to the same file when the type of system  
doing the build is the same as where  you are going 
the execute the RVM. To cross build a system
e.g., build on AIX\AIXTMFootnote/PowerPC\PowerPCTMFootnote for a
Linux/IA32 platform, see the section on Cross 
Platform Building.

\item {\bf Edit configuration scripts.}

You must edit a script in the {\tt \$RVM\_ROOT/rvm/config}  directory to set 
up variables used by the installation process.  
If someone else at your site has already installed RVM, they have
probably already done this step for you.  Consult your local RVM guru.

You must edit the file(s) that define the host and target configuration
environments in the {\tt \$RVM\_ROOT/rvm/config} directory.  
You do not need to {\em source} these variables in your working shell; 
variables in this file will be picked up by the installation scripts.  

The host and target configuration files have two sections.  In the
first section, you specify the operating system, architecture, and
whether or not the platform will support SMP-builds of JikesRVM. 
For operating system, define one of RVM\_FOR\_LINUX, RVM\_FOR\_AIX, or
RVM\_FOR\_CYGWIN to be 1.  For architecture define either
RVM\_FOR\_IA32 or RVM\_FOR\_POWERPC to be 1.  For SMP status, set
RVM\_FOR\_SINGLE\_VIRTUAL\_PROCESSOR to 0 (SMP supported) or 1 (SMP not
supported).  The following are the typical settings for
RVM\_FOR\_SINGLE\_VIRTUAL\_PROCESSOR 
\begin{description}
\item {\tt AIX/PowerPC} 0 
\item {\tt Linux/PowerPC} 1 
\item {\tt Linux/IA32} 1 with a 2.2 kernel; 
0 with a 2.4 kernel and glibc compiled to use the GS
segment register to access pthread-specific state.
\end{description}                
The second section in the configuration file is used to define how to
find tools that RVM needs. You must set the following variables:
\begin{description}
\item {\tt HOST\_JAVA\_HOME} the base directory for JDK JVM
\item {\tt HOST\_JAVA} the executable command for the JDK JVM
\item {\tt HOST\_JAVAC} the {\tt javac} executable for the JDK  
\item {\tt HOST\_JAR} the {\tt jar} executable for the JDK  
\item {\tt HOST\_REPOSITORIES} the {\tt rt.jar} archive for the JDK  
\item {\tt HOST\_TOOLS} the {\tt tools.jar} archive for the JDK  
\item {\tt GNU\_MAKE} the GNU {\tt make} executable
\item {\tt JIKES} the Jikes\trademark compiler executable ({\tt jikes}).
\item {\tt CC} how to invoke the C compiler.
\item {\tt CPLUS} how to invoke the C++ compiler.
\item {\tt LDSHARED} how to link a shared C++ library.
\item {\em various basic Unix utilities} e.g., {\tt grep}, {\tt xargs}, etc.
\end{description}
The remaining variables in the config file are not required for basic RVM
builds.

Someday we should consider setting up an autoconf to automate this
step.

\index{configurations}
\index{jconfigure script}
\item {\bf Choose configuration and populate your build directory.}

You will use the {\tt jconfigure} script (in {\tt \$RVM\_ROOT/rvm/bin}) to
populate your build ({\tt \$RVM\_BUILD}) directory with files.  You must
first choose a RVM configuration.

For novice users, two configurations are recommended.  (A discussion
of RVM configurations appears in Section~\ref{configs}.)

\begin{itemize}
\item {\tt BaseBaseSemispace}: a non-adaptive system that uses the
baseline compiler everywhere, with the semispace copying collector
\item {\tt OptOptSemispace}: a non-adaptive system that uses the
optimizing compiler everywhere, with the semispace copying collector
\end{itemize}

Depending on your purposes (See Section~\ref{ssec:choosinggc}.) you
may want to choose another configuration, e.g.,
\begin{itemize}
\item {\tt OptOptMarkSweep}: a non-adaptive system that uses the
optimizing  compiler everywhere, with the mark-sweep (noncopying) collector
\end{itemize}

Run the {\tt jconfigure} script to set up the {\tt \$RVM\_BUILD}
directory for the configuration you desire.  This step creates
build scripts for your configuration and otherwise formats your
{\tt \$RVM\_BUILD} directory.
The {\tt jconfigure} script takes one argument, the name of the
configuration desired: 

\begin{verbatim}
% jconfigure <configuration>
\end{verbatim}

For example, to configure a build 
directory for the {\tt OptOptSemispace} configuration, type
the following command:

\begin{verbatim}
% jconfigure OptOptSemispace
\end{verbatim}

\index{jbuild script}
\index{boot image}
\index{RVM\_BUILD}
\item {\bf Build an executable version of RVM.}  

Use the {\tt jbuild} script, located in the {\tt \$RVM\_BUILD} directory,
to build an executable system.  This script copies source files into
{\tt \$RVM\_BUILD/RVM.classes}, preprocesses these files, generates
some code with template expansions, builds an executable C program to
start the RVM, and writes the RVM boot image.  The boot image is the
binary image of a ready-to-go instance of the RVM.

The {\tt jbuild} script must be run from the {\tt \$RVM\_BUILD}
directory. It prints a copious report of its operation which you may
save for future reference by redirecting standard out and err.

\begin{verbatim}
% cd $RVM_BUILD
% jbuild
\end{verbatim}


After the {\tt jbuild} script has completed successfully you should be able 
to run RVM.  (See Section~\ref{section:running}.)

Note: The jbuild process may produce warning messages; these should not
affect system viability.

\end{enumerate}

\JikesTMFooter

\AIXTMFooter

\PowerPCTMFooter

\subsection{RVM Configurations}\label{configs}
\index{configuration names}

This section describes the RVM build configurations.  The various RVM
build configurations are defined by files in {\tt
\$RVM\_ROOT/rvm/config/build}.

Most standard RVM configuration files loosely follow the following naming scheme
\begin{verbatim}
       <boot image compiler> <runtime compiler> <garbage collector>
\end{verbatim}

\index{boot image compiler}
\index{runtime compiler}
where
\begin{itemize}
\item the {\em boot image compiler} is the compiler used to compile the RVM boot image.
\item the {\em runtime compiler} is the ``compiler'' used to compile
the classes loaded at runtime.  
\item the {\em garbage collector} is the garbage collection scheme used.
\end{itemize}

The types of compilers -- the baseline compiler and 
the optimizing compiler -- are designated by the names {\em Base}
and {\em Opt} respectively.  In these configurations,
all classes loaded at runtime are compiled once, by the specified
compiler.  This is different than the adaptive configurations,
discussed in Section~\ref{adaptive-configs}.

A garbage collector may have any of the following types:

\begin{description}
\item[NoGC] no garbage collection is performed
\item[Semispace] a copying semi-space collector
\item[MarkSweep] a mark-and-sweep (non copying) collector
\item[CopyGen] a copying generational collector with a
fixed-size nursery
\item[CopyGenVariable] a copying generational collector with a
variable-size nursery
\item[Hybrid] a hybrid generational collector, semi-space for the
nursery and mark-and-sweep for the mature space
\item[Concurrent] a concurrent reference counting collector
\end{description}

For example, to specify a compiler with a baseline-compiled boot image
that will 
compile classes loaded at runtime using the optimizing compiler and that uses
a non-generational semi-space copying garbage collector use the name 
{\em BaseOptSemispace}.

Some files augment the standard configurations as follows:
\begin{itemize}
\item The word 
{\em Full} at the beginning of the configuration name identifies a 
configuration
such that all the RVM classes are included in the boot image (by default
a small subset of the RVM classes are included in the boot image). 
\item The word
{\em Fast} at the beginning of the configuration name identifies a Full
configuration where all assertion checking has been turned off. 
\end{itemize}
A boot image with
either of these modifications is likely to run faster than without
(the opt compiler will be opt compiled),
but take longer to build.  

\subsubsection{Adaptive Configurations} \label{adaptive-configs}
\index{adaptive configurations}
In the non-adaptive configurations, all classes
loaded at runtime are compiled once by the specified
compiler: base or opt.  Another option is to build one configuration,
an adaptive configuration,
where the runtime compiler is either
\begin{itemize}
\item specified on the command line (base or opt), or
\item selected automatically as the application runs
\end{itemize}

The first choice allows an adaptive configuration to provide the same
functionality as a non-adaptive configuration.  One image is built,
and the runtime compiler can be specified at the command line as
follows:
\begin{verbatim}
  rvm -X:aos:primary_strategy=baseonly
            or
  rvm -X:aos:primary_strategy=optonly
\end{verbatim}

The second choice initially compiles all methods with the
baseline compiler and then automatically selects hot methods for
recompilation by the opt compiler at an appropriate optimization
level. Further details are provided in Section~\ref{section:aosdetails}.

The adaptive configurations follow the following naming scheme
\begin{verbatim}
           [boot image compiler] Adaptive  <garbage collector>
\end{verbatim}

For example, to configure a build 
directory for an adaptive configuration, where the Opt compiler is 
used to compile the boot image (but is not included in the boot image
and assertions are turned on), and the semi-space garbage collector is
used, use the following command:

\begin{verbatim}
% jconfigure OptAdaptiveSemispace
\end{verbatim}

Section~\ref{section:running} describes how this image can be used in
the manner mentioned above.

To view a list of configurations see 
{\tt \$RVM\_ROOT/rvm/config/build}.  Follow the examples in this
directory to define your own configurations with different options.  See
the {\tt jconfigure} file for a list of all options the builder
understands.

\subsection{Cross Platform Building}

The RVM build process consists of two major phases: the building of a
RVM {\em boot image}, and the building of a RVM {\em boot loader}.
The boot image is built using a Java\trademark program executed within a host
JVM and is therefore platform-neutral.  By contrast, the boot loader
is written in C, and must be compiled on the target platform.

Because the building of the boot image can be a relatively lengthy
process, it can be advantageous to perform that task somewhere other
than the target platform.  To cross build, simply set your
RVM\_HOST\_CONFIG and RVM\_TARGET\_CONFIG environment variables to
be different files.

For example, to build a BaseBaseSemispace system for AIX\AIXTMFootnote
{\em on a Linux host}:
\begin{verbatim}
% setenv RVM_ROOT $HOME/rvmRoot
% setenv RVM_BUILD $HOME/rvmBuild
% setenv PATH $RVM_ROOT/rvm/bin:$PATH
% setenv RVM_TARGET_CONFIG=$RVM_ROOT/rvm/config/powerpc-ibm-aix4.3.3.0
% setenv RVM_HOST_CONFIG=$RVM_ROOT/rvm/config/i686-pc-linux-gnu
% jconfigure BaseBaseSemispace
% cd $RVM_BUILD
% jbuild
\end{verbatim}

This phase of the build process will complete with the words ``{\tt
  please run me on AIX}''.


The build process is then completed by building just the boot loader {\em
  on an AIX host}:

\begin{verbatim}
% setenv RVM_ROOT $HOME/rvmRoot
% setenv RVM_BUILD $HOME/rvmBuild
% setenv PATH $RVM_ROOT/rvm/bin:$PATH
% jbuild -booter
\end{verbatim}

After the {\tt jbuild -booter} script has completed successfully you should be able 
to run RVM. 

The building of the boot loader must occur in the same directory as
the rest of the build.  This can either be done transparently via a
network file system, or by copying the build directory from the first
host to the target.  Of course {\tt RVM\_ROOT}, {\tt RVM\_BUILD }
and {\tt PATH} need not be explicitly set each time: they could have
been set in your {\tt .cshrc}.

More advanced users can experiment with the {\tt RVM\_BUILD\_COPY}
environment variable.  If this is set, then the {\tt
  jbuild.linkBooter} phase of the build process is replaced by the
execution of {\tt `\$RVM\_BUILD\_COPY`}.  This opens up a lot of
possibilities, including: copying the build directory to a target
machine and executing {\tt jbuild.linkBooter} remotely on the target
via {\tt rsh} or {\tt ssh}, etc.  By setting {\tt RVM\_BUILD\_COPY}
appropriately on the host platform, cross-platform building can become
a stream-lined process.

\JikesTMFooter

\JavaTMFooter

\AIXTMFooter

\subsection{Building the libraries}

Most RVM users will not need to rebuild the libraries; thus the default
process described above provides a ``binary'' {\tt rvmrt.jar}.  However,
the library source for the RVM is available at
\xlink{{\tt \RVMDownloadURL}}{\RVMDownloadURL}.  
Please consult the 
\xlink{license}{\RVMLibSourceLicenseURL} for restrictions.  

Should you decide to modify the library, you will need to rebuild the 
{\tt rvmrt.jar}.  The script to do this is
{\tt \$RVM\_ROOT/rvm/bin/jBuildLibs}.  Running this script will compile the
library sources in {\tt \$RVM\_ROOT/support/lib}, and build a new {\tt
rvmrt.jar} in {\tt \$RVM\_BUILD}.

You will find that some versions (including 1.13) of 
\xlink{{\tt Jikes\trademark}}{ \jikesURL } fail to compile the
libraries, dying with myriad errors.  You need to apply Jikes 
patch 62 to
the Jikes build to fix a problem with variable shadowing by inner classes. 
We have
successfully applied this patch to Jikes version 1.13 on both
AIX\AIXTMFootnote and Linux/IA32.

The {\tt jBuildLibs} script will prompt you, asking whether to install 
the new {\tt rvmrt.jar} in {\tt \$RVM\_ROOT/support/lib}.  
If you answer {\tt 'y'}, future invocations of {\tt jbuild} will pick up
the new library build.  However, note that by installing it, you will
{\em overwrite} the original version of {\tt rvmrt.jar}, losing it.  Be
careful!

\JikesTMFooter

\AIXTMFooter

\subsection{Building documentation}

The {\tt \RVMTarFile} file contains a postscript version of this userguide
in {\tt \$RVM\_ROOT/rvm/doc}.  Additionally, the 
\xlink{developerWorks web page}{\RVMHomeURL} keeps an on-line version of
the userguide and javadoc API, corresponding to the latest HEAD of the CVS
repository.

If you would like to recover the userguide or javadoc for an older release
of RVM, you can rebuild the documentation locally.  See the Makefile in
{\tt \$RVM\_ROOT/rvm/doc/userguide} for rules on how to build the
HTML userguide using
\xlink{{\tt hyperlatex}}{\HyperlatexURL}.  To build the javadoc pages, use
the {\tt jdoc.sh} script in {\tt \$RVM\_ROOT/rvm/bin}; this script takes as
its one command-line argument the directory to output the javadoc HTML.



\T \newpage
% \W \htmldirectory{HTML}
\xname{running}
\section{Running \jrvm} \label{section:running}
% \W \htmldirectory{HTML/running}
This section describes how to run a Jikes\TMweb{}{} RVM  image built
from the previous section. 

%%%
% This section uses the EBNF commands defined in userguide.tex
%%%%%%%%%%%%%%%%%%%%%%%%
\subsection{Running \jrvm}

\cindex[jikes command]{\texttt{jikes} command}%
\cindex[rvm script]{\texttt{rvm} script}%
Jikes\TMweb{} RVM executes Java virtual machine byte code instructions from {\tt .class} files.  
It does {\em not} compile 
Java\TMweb{} source code. Therefore, you must compile all Java source
files into byte code using your favorite Java compiler.
Our favorite Java compiler is the IBM\Rweb{} Jikes compiler. 

For example, to run class {\tt foo} with source code in file {\tt foo.java}:
\begin{verbatim}
% jikes foo.java
% rvm foo 
\end{verbatim}

The general syntax is
\begin{example}
\tt{}   rvm \MZeroOrMore{rvm options\ldots} class \MZeroOrMore{args\ldots}
\end{example}

\index{command-line options}%
You may choose from a myriad of options for the {\tt rvm} command-line.  
Options fall into two categories: {\em standard} and {\em
non-standard}.  Non-standard options are preceded by {\bf ``{\tt -X:}''}.

%%%%%%%%%%%%%%%%%%%%%%%%%%%%%%%
\subsubsection{Standard Command-Line Options}

We currently support a subset of the JDK 1.4 standard options.  Below
is a list of all options and their descriptions.  Unless otherwise noted each
option is supported in Jikes RVM.\@
\begin{description}
\item[{\tt \Mlbr{} -cp \Mor{} -classpath \Mrbr{} \Mmeta{directories and
zip/jar files separated by \Mlitch{:}}}]
set search path for application classes and resources

\item[{\tt -D\Mmeta{name}=\Mmeta{value}}] set a system property

\item[{\tt -verbose:\Mlsq{} class \Mor{} gc \Mor{} jni \Mrsq}]
enable verbose output

\item[{\tt -version}] print current VM version and terminate the run

\item[{\tt -showversion}] print current VM version and continue running

\item[{\tt -fullversion}] like ``{\tt -version}'', but with more information

\item[{\tt -?} or {\tt -help}] print help message

\item[{\tt -X}] print help on non-standard options

\item[{\tt -jar}] execute a jar file 

\end{description}

%%%%%%%%%%%%%%%%%%%%%%%%%%%%%%%%%%%%%%%%%%%%%%%%%
\subsubsection{Non-standard Command-Line Options}

The non-standard options are

\begin{description}
\item[{\tt -X}]
immediately print usage information on nonstandard options

\item[{\tt -X:verbose}]
during run, print out additional information for GC and hardware trap handling

\item[{\tt -X:verboseBoot=\Mmeta{number}}]
print out additional information while VM is booting, using verbosity
level \Mmeta{number}.

\item[{\tt -Xms\Mmeta{number}\Mlsq{}\Mmeta{unit}\Mrsq{}}]
set the initial heap size to \Mmeta{number}\Mmeta{unit} bytes.
 \link{The section on MMTk Command-Line Options \texonly{(see
     appendix~\Ref, page~\Pageref{})} explains the syntax for
     expressing memory sizes}{section:mmtkoptions}%
.

\item[{\tt -Xmx\Mmeta{number}\Mlsq{}\Mmeta{unit}\Mrsq{}}]
set the maximum heap size to \Mmeta{number}\Mmeta{unit} bytes.


\item[{\tt -X:sysLogfile=\Mmeta{filename}}]
redirects messages that would go to standard error to \Mmeta{filename} instead.

\item[{\tt -X:i=\Mmeta{filename}}]
read Jikes RVM's \emph{boot image} from \Mmeta{filename}

\item[{\tt -X:vm\Mlsq{}:help\Mrsq{}}]
immediately print options supported by the core virtual machine

\item[{\tt -X:vm:\Mmeta{option}}]
pass \Mmeta{option} to the core virtual machine

\item[{\tt -X:gc\Mlsq{}:help\Mrsq}]
print options supported by the memory management system

\item[{\tt -X:gc:\Mmeta{option}}]
pass \Mmeta{option} to the memory management system

\item[{\tt -X:aos\Mlsq{}:help\Mrsq{}}]
print options supported by adaptive optimization system when in an
adaptive configuration

\item[{\tt -X:aos:\Mmeta{option}}]
pass \Mmeta{option} to the adaptive optimization system when in an adaptive configuration

\item[{\tt -X:irc\Mlsq{}:help\Mrsq{}}]
print options supported by the initial runtime compiler

\item[{\tt -X:irc:\Mmeta{option}}]
pass \Mmeta{option} to the initial runtime compiler

\item[{\tt -X:recomp\Mlsq{}:help\Mrsq{}}]
print options supported by the compilers used for recompilation

\item[{\tt -X:recomp:\Mmeta{option}}]
pass \Mmeta{option} to the compilers used for recompilation

\item[{\tt -X:base\Mlsq{}:help\Mrsq{}}]
print the options supported by the baseline compiler

\item[{\tt -X:base:\Mmeta{option}}]
pass \Mmeta{option} to the baseline compiler

\item[{\tt -X:quick\Mlsq{}:help\Mrsq{}}]
print the options supported by the quick compiler

\item[{\tt -X:quick:\Mmeta{option}}]
pass \Mmeta{option} to the quick compiler

\item[{\tt -X:opt\Mlsq{}:help\Mrsq{}}]
print the options supported by the optimizing compiler

\item[{\tt -X:opt:\Mmeta{option}}]
pass \Mmeta{option} to the optimizing compiler

\item[{\tt -X:prof:\Mmeta{option}}]
pass \Mmeta{option} to the profiling subsystem

\item[{\tt -X:vmClasses=\Mmeta{path}}]
load classes from \Mmeta{path}.  \Mmeta{path} is the file name or
file names of one or more {\tt .jar} files.  File names in
\Mmeta{path} are separated by colons (\Mlitch{:}).

\item[{\tt -X:cpuAffinity=\Mmeta{int}}]
\Mmeta{int} is the number of the physical CPU to which the first virtual processor will be bound.

\item[{\tt -X:processors=\Mlsq{} \Mmeta{int} \Mor{} \Mlitch{all} \Mrsq{}}]
number of processors to use on a multiprocessor

\end{description}

\link{Later in this user's guide \texonly{(see appendix~\Ref, page~\Pageref{})} there
are more details on command-line options}{appendix:nonadaptive:cmdline}, including the list of options supported by the baseline
compiler, optimizing compiler, and adaptive optimization system. 

\subsection{Regression Tests}
\label{sec:regression}
\remark{TODO. Write this section}


% LocalWords:  args JDK cp gc jni VM showversion fullversion verboseBoot Xms vm
% LocalWords:  Xmx sysLogfile RVM's aos irc recomp vmClasses cpuAffinity


\T \newpage
% \W \htmldirectory{HTML}
\xname{eclipse}
\section{Eclipse on Jikes RVM} \label{section:eclipse}
\newcommand{\eclipseURL}{http://www.eclipse.org}
\newcommand{\antURL}{http://ant.apache.org}

\index{Eclipse IDE, running under Jikes RVM}
Since version 2.2.1, Jikes RVM purports to run
\xlink{Eclipse}{\eclipseURL}.  This is really a technology preview and
for that reason we start with some caveats, the most obvious being
that production-mode use of Eclipse with Jikes RVM
is not recommended at this time.  Furthermore:
\begin{itemize}
\item We strongly recommend you run a
recent---7.3 or newer---version of RedHat Linux for Intel IA32.
\item Use either *Adaptive* or BaseBase*  Jikes RVM builds when trying
to run Eclipse.  The *Opt* images will attempt to optimize all
executed methods, which will result in bad performance.
\item Use only the *SemiSpace builds of Jikes RVM.  We have seen
various limitations of the other collectors, such as severe
fragmentation and implementation-specific resource limits.
\item Use only the Eclipse builds for Linux and GTK.  There are builds
for Linux and Motif, but we never test these and work on Eclipse seems
to focus on the GTK version.
\item We strongly recommend Eclipse 2.1, currently the latest release
from \xlink{{\tt \eclipseURL}}{\eclipseURL}; this is what we use and
test at Watson.  Eclipse 2.0.x releases may also work.  Do not try to
use Eclipse 1.x versions.
\item The Eclipse debugging support for Java programs---which relies
on talking JDWP to the VM---will not work because
Jikes RVM has yet to implemented this protocol.
\item Running Eclipse requires that you build Jikes RVM with
RVM\_FOR\_SINGLE\_VIRTUAL\_PROCESSOR equal to 0. The
select-interception mechanism (which is required to run Eclipse on
Jikes RVM) currently does not work if this variable is set to 1 (see
defect 3583).
\end{itemize}

Given these caveats, Eclipse has been running on Jikes RVM at
Watson in various configurations since the summer of 2002, and
more-or-less all of the standard Eclipse functionality appears to work
(The single biggest exception is debugging support, as mentioned
above).  We have successfully developed and run Java programs, checked
projects out of CVS repositories, used the refactoring support, run
the Web-based help system, applied updates to Eclipse, and so on.
Furthermore, performance is generally not too bad, at least when
adaptive builds of Jikes RVM, such as FastAdaptiveSemiSpace, 
are used.  Most importantly, we very much
want users to try running Eclipse on Jikes RVM: we eagerly
solicit feedback regarding bugs and would be especially grateful for
any contributions that enhance the usability of Eclipse on Jikes RVM.  

Setting up Eclipse and Jikes RVM is relatively straightforward; you
have to install Jikes RVM and Eclipse themselves first, and, after
that, there are just a few steps:
\begin{enumerate}
\index{Ant, the Java-based build tool}
\item Install Ant---the Java-based build tool.  You can get this from
\xlink{{\tt \antURL}}{\antURL}. 
\item Add two variables to your Jikes RVM configuration file:
 \begin{description}
  \index{ANT\_CMD}
 \item[ANT\_CMD] is the full pathname of the executable Ant command
  \index{ECLIPSE\_INSTALL\_DIR}
 \item[ECLIPSE\_INSTALL\_DIR] is the root of your Eclipse install
 \end{description}
\item In a fresh build directory, run ./jbuild.plugin to install
support for Eclipse to use Jikes RVM
\end{enumerate}

\index{rvmeclipse}
 Once you have done this, you will be able to run Eclipse on Jikes RVM
using the {\tt RVM\_ROOT/rvm/bin/rvmeclipse} command.  This wrapper
calls the normal Eclipse command, instructing it to use Jikes RVM to
run Eclipse.  You must have RVM\_ROOT and RVM\_BUILD set whenever you
invoke {\tt rvmeclipse}.


\T \newpage
% \W \htmldirectory{HTML}
\xname{gcspy}
\section{Running Jikes RVM with GCSpy} \label{section:gcspy}
\newcommand{\rvmRoot} {\texttt{\$RVM\_ROOT}}

\xname{gcspy_framework}
\subsection{The GCspy Heap Visualisation Framework}

\newcommand{\GCspyOOPSLATwoPaperURL}{http://www.cs.kent.ac.uk/pubs/2002/1426/}
GCspy is a visualisation framework that allows developers to observe
the behaviour of the heap and related data structures.  For details of
the GCspy model, see \xlink{\textit{GCspy: An adaptable heap visualisation framework}
by Tony Printezis and Richard Jones, OOPSLA'02}{\GCspyOOPSLATwoPaperURL}\texonly{, also available at \texttt{\GCspyOOPSLATwoPaperURL}}.   The framework
comprises two components that communicate across a socket: a
\emph{client} and a \emph{server} incorporated
into the virtual machine of the system being visualised.
The client is usually a visualiser (written in Java) but the framework
also provides other tools (for example, to store traces in a compressed file).
The GCspy server implementation for JikesRVM was contributed by Richard
Jones of the University of Kent. 

GCspy is designed to be independent of the target system.  Instead, it
requires the GC developer to describe their system in terms of four
GCspy abstractions, \emph{spaces, streams, tiles} and \emph{events}.
This description is
transmitted to the visualiser when it connects to the server. 

A \textit{space} is an abstraction of a component of the system; it may
represent a memory region, a free-list, a remembered-set or whatever.
Each space is divided into a number of blocks which are represented by
the visualiser as \emph{tiles}.  Each space will have a number of attributes
--- \emph{streams} --- such as the amount of space used, the number of
objects it contains, the length of a free-list and so on. 

In order to instrument a \jrvm{} collector with GCspy:
\begin{enumerate}
   \item Provide a \texttt{startGCspyServer} method in that
   collector's plan.  That method initialises the GCspy server with the port on which to communicate and a list of event names, instantiates drivers for each space, and then starts the server.
   \item  Gather data from each space for the tiles of each stream (e.g.\ before, during and after each collection).
   \item  Provide a driver for each space. 
\end{enumerate} 

\textit{Space drivers} handle communication between collectors and the GCspy
infrastructure by mapping information collected by the memory manager
to the space's streams.  A typical space driver will:

\begin{itemize}
   \item Create a GCspy \textit{space}.
   \item Create a \textit{stream} for each attribute of the space.
   \item Update the tile statistics as the memory manager passes it information. 
   \item Send the tile data along with any summary or control information to the visualiser.
\end{itemize}

The \jrvm{} SSGCspy plan gives an example of how to instrument a
collector.  It provides GCspy spaces, streams and drivers for the
semi-spaces, the immortal space and the large object space, and also
illustrates how performance may be traded for the gathering of more
detailed information.

\xname{gcspy_install}
\subsection{Installation of GCspy with \jrvm{}}


\subsubsection{System Requirements}

The GCspy C server code needs a pthread (created in
\texttt{gcspyStartserver()} in \texttt{sys.C}) in order to run.
So, GCspy will only work on a system where you've build Jikes RVM with
\texttt{RVM\_FOR\_SINGLE\_VIRTUAL\_PROCESSOR} set to \texttt{0}.
\texttt{jconfigure} will warn you if you try to configure such a build
(this parameter is discussed
  \link{elsewhere in this guide}[
  (SubSubSection~\ref{single-virtual-processor-subsubsection},
  item~\Ref, on page~\Pageref)]{single-virtual-processor-item}.)

In your \texttt{\$RVM\_ROOT/rvm/i686-pc-linux-gnu\textit{.mine}} file,
set \texttt{GCSPY\_ROOT} to where you will place the GCspy libraries,
e.g.\ \texttt{export GCSPY\_ROOT="\$HOME/gcspy1.0"}. 

\paragraph{Downloads}

\begin{enumerate}

\newcommand{\GCspyURL}{http://www.cs.kent.ac.uk/projects/gc/gcspy/}
   \item Download the GCspy files from \xlink{\texttt{\GCspyURL}}{\GCspyURL}
   You want the C infrastructure (not the C++ one) and the Java visualiser. 
   The C infrastructure provides a GCspy server to \jrvm{}.

\newcommand{\JAIURL}{http://java.sun.com/products/java-media/jai}
  \item Download the Java Advanced Imaging (JAI) API from
  \xlink{\texttt{\JAIURL}}{\JAIURL}.  You want the download named
  ``Linux CLASSPATH Install''.   You will pull down a file named
  \texttt{jai-1\_1\_2-lib-linux-i586.tar.gz} 

  \item  Unpack the GCspy and JAI sources. 
\begin{verbatim}
    $ tar xzf gcspy1_0.tar.gz
    $ tar xzf jai-1_1_2-lib-linux-i586.tar.gz
\end{verbatim}
  You should then have directories named  
  {\tt gcspy1.0} and
  {\tt jai-1\_1\_2}.  


\end{enumerate}

\paragraph{Building GCspy itself}

\newcommand{\gcspyroot}{\$GCSPY\_ROOT}
   Edit the file \gcspyroot/src/java/GNUmakefile and set the
   value of \texttt{JAI\_ROOT} to where you have placed {\tt jai-1\_1\_2}.
   
   Copy the JAI \texttt{.jar} files into your JDK's \texttt{ext}
   directory:
\begin{verbatim}
    $ cp jai-1_1_2/lib/*.jar $JAVA_HOME/jre/lib/ext/
\end{verbatim}

   Make the GCspy visualiser and server.
   \textbf{Note:} you will need Java 5 (or later) to compile and run the
   GCspy 1.0 visualiser.
\begin{verbatim}
    $ cd $GCSPY_ROOT/src/
    $ make install c java
\end{verbatim}

\paragraph{Building \jrvm{} to use GCspy}

\begin{itemize}
  \item Build an image, using the a GCspy configuration, such as
    \texttt{BaseBaseSemiSpaceGCspy}: 
\begin{verbatim}
    $ jconfigure BaseBaseSemiSpaceGCspy
    $ cd $RVM_BUILD
    $ ./jbuild
\end{verbatim}
\end{itemize}

\paragraph{Running \jrvm{} with GCspy}

\begin{itemize}
   \item  Next, start \jrvm{}, first adding the GCspy server library to your {\tt LD\_LIBRARY\_PATH}:
\begin{verbatim}
    $ export LD_LIBRARY_PATH=$LD_LIBRARY_PATH:$GCSPY_ROOT/src/c/lib
    $ rvm -Xms20m -X:gc:gcspyPort=3000 -X:gc:gcspyWait=true  &
\end{verbatim}

   \item Then, start the GCspy visualiser:

\begin{verbatim}
    $ cd $GCSPY_CLASSES
    $ java gcspy.Main -server localhost 3000
\end{verbatim}

      and click the ``Connect'' button in the bottom right-hand corner
      of the visualiser.  (Alternatively, you can give the server and
      port arguments in the Connect dialogue box.)  

%      The visualiser itself has not been tested on \jrvm{}.
\end{itemize}

\xname{gcspy_arguments}
\subsection{Command line arguments}

Additional GCspy-related arguments to the \texttt{rvm} command:

\begin{itemize}

\item {\tt -X:gc:gcspyPort=\Mmeta{port}} \\
    The number of the port on which to connect to the visualiser.  The
    default is port \texttt{0}, which signifies no connection. 

\item {\tt -X:gc:gcspyWait=\Mlbr{} \mbox{\texttt{true}} \Mor{} \mbox{\texttt{false}} \Mrbr} \\
    Whether \jrvm{} should wait for a visualiser to connect.

\item {\tt -X:gc:gcspyTilesize=\Mmeta{size}} \\
    How many KB are represented by one tile.  The default value is 128.

\end{itemize}

\subsection{Writing GCspy drivers}

To instrument a new collector with GCspy, you will probably want to subclass your 
collector and to write new drivers for it. 
The following sections explain the modifications you need to make
and how to write a driver. You may use \texttt{org.mmtk.plan.semispace.gcspy} and its
drivers as an example.


The recommended way to instrument a \jrvm{} collector with GCspy is to
create a \texttt{gcspy} subdirectory in the directory of the collector being instrumented,
e.g.\ \texttt{MMTk/src/org/mmtk/plan/semispace/gcspy}. 
In that directory, we need 5 classes:
\begin{itemize}
\item \texttt{SSGCspy},
\item \texttt{SSGCspyCollector},
\item \texttt{SSGCspyConstraints} 
\item \texttt{SSGCspyMutator} and
\item \texttt{SSGCspyTraceLocal}.
\end{itemize}

\texttt{SSGCspy} is the plan for the instrumented collector. It is a 
subclass of \texttt{SS}.

\texttt{SSGCspyConstraints} extends \texttt{SSConstraints} to provide methods
\texttt{boolean needsLinearScan()} and \texttt{boolean withGCspy()}, both
of which return true.

\texttt{SSGCspyTraceLocal} extends \texttt{SSTraceLocal} to override methods
\texttt{traceObject} and 
\texttt{willNotMove}
to ensure that tracing 
deals properly with GCspy objects: the GCspyTraceLocal file will be similar for 
any instrumented collector.

The instrumented collector, \texttt{SSGCspyCollector}, extends \texttt{SSCollector}.
It needs to override \texttt{collectionPhase}.

Similarly, \texttt{SSGCspyMutator} extends \texttt{SSMutator} and must also override its 
parent's methods
\texttt{collectionPhase}, to
allow the allocators to collect data;
and its \texttt{alloc} and \texttt{postAlloc} methods to allocate GCspy objects in GCspy's
heap space.


   
\subsubsection{The Plan}

\texttt{SSGCspy.startGCspyServer} is called immediately before the ``main'' method is loaded and run. 
It initialises the GCspy server with the port on which to communicate,
adds event names, 
instantiates a driver for each space, and then starts the server,
forcing the VM to wait for a GCspy to connect if necessary.
This method has the following responsibilities.

\begin{enumerate}
\item Initialise the GCspy server: 
\begin{verbatim}
      server.init(name, portNumber, verbose);
\end{verbatim}

\item Add each event to the \texttt{ServerInterpreter} (`server' for short) 
\begin{verbatim}
      server.addEvent(eventID, eventName);
\end{verbatim}

\item Set some general information about the server 
      (e.g. name of the collector, build, etc) 
\begin{verbatim}
      server.setGeneralInfo(info);
\end{verbatim}

\item Create new drivers for each component to be visualised 
\begin{verbatim}
      myDriver = new MyDriver(server, args...);
\end{verbatim}
      Drivers extend \texttt{AbstractDriver} and register their space with the 
      \texttt{ServerInterpreter}. In addition to the server, drivers will take 
      as arguments the name of the space, the MMTk space, the tilesize, and
      whether this space is to be the main space in the visualiser.
\end{enumerate}

\subsubsection{The Collector and Mutator}

Instrumenters  will typically want to add data collection points 
before, during and after a collection by overriding \texttt{collectionPhase} 
in \texttt{SSGCspyCollector} and \texttt{SSGCspyMutator}.

\texttt{SSGCspyCollector} deals with the data in the semi-spaces
that has been allocated there (copied) by the collector. It only
does any real work at the end of the collector's last tracing
phase, \texttt{FORWARD\_FINALIZABLE}.

\texttt{SSGCspyMutator} is more complex: as well as gathering data
for objects that it allocated in From-space at the start of
the \texttt{PREPARE\_MUTATOR} phase, it also deals with the 
immortal and large object spaces.

At a collection point, the collector or mutator will typically
\begin{enumerate}
\item Return if the GCspy port number is 0 (as no client can be connected).

\item Check whether the server is connected at this event. If so, the 
   compensation timer (which discounts the time taken by GCspy to
   gather the data) should be started before gathering data and
   stopped after it.

\item After gathering the data, have each driver  call its
   \texttt{transmit} method.

\item \texttt{SSGCspyCollector} does \emph{not} call the GCspy
   server's \texttt{safepoint} method, 
   as the collector phase is usually followed by a mutator
   phase.
   Instead, \texttt{safepoint} can be called by 
   \texttt{SSGCspyMutator} to indicate that this
   is a point at which the server can pause, play one
   event, etc. 
\end{enumerate}

Gathering data will vary from MMTk space to space. It will typically be
necessary to resize a space before gathering data.
For a space, 

\begin{enumerate}
\item We may need to reset the GCspy driver's data 
   depending on the collection phase.

\item We will pass the driver as a call-back to the allocator.
The allocator will typically ask the driver to set the range of addresses
from which we want to gather data, using the driver's \texttt{setRange} method.
The allocator should then iterates through its MMTk space, passing a reference to each object
found to the driver's scan method.
\end{enumerate}


\subsubsection{The Driver}

GCspy space drivers extend \texttt{AbstractDriver}. This class creates a new GCspy 
\texttt{ServerSpace}
and initialises the control values for each tile in the space. 
\emph{Control} values
indicate whether a tile is \emph{used}, \emph{unused}, a \emph{background}, a 
\emph{separator} or a \emph{link}.
The constructor for a typical space driver will:

\begin{enumerate}
\item Create a GCspy \texttt{Stream} for each attribute of a space.

\item Initialise the tile statistics in each stream.
\end{enumerate}

\noindent
Some drivers may also create a \texttt{LinearScan} object to handle call-backs 
from the VM as it sweeps the heap (see above).

The chief roles of a driver are to accumulate tile statistics, and to transmit
the summary and control data and the data for all of their streams. Their
data gathering interface is the \texttt{scan} method (to which an object 
reference or address is passed). 

When the collector or mutator has finished gathering data,
it calls the \texttt{transmit} of the driver for each space that needs to send 
its data.
Streams may send values of types byte, short or int, implemented through classes
\texttt{ByteStream}, \texttt{ShortStream} or \texttt{IntStream}. 
A driver's \texttt{transmit} method will typically:

\begin{enumerate}
\item Determine whether a GCspy client is connected and interested in
      this event, e.g. 
\begin{verbatim}
      server.isConnected(event)
\end{verbatim}
\item Setup the summaries for each stream, e.g. 
\begin{verbatim}
      stream.setSummary(values...);
\end{verbatim}

\item Setup the control information for each tile. e.g. 
\begin{verbatim}
      controlValues(CONTROL_USED, start, numBlocks);
      controlValues(CONTROL_UNUSED, end, remainingBlocks);
\end{verbatim}

\item Set up the space information, e.g. 
\begin{verbatim}
      setSpace(info);
\end{verbatim}

\item Send the data for all streams, e.g. 
\begin{verbatim}
      send(event, numTiles);
\end{verbatim}
      Note that \texttt{AbstractDriver.send} takes care of sending the 
      information for all streams (including control data).
\end{enumerate}


\subsubsection{Subspaces}

\texttt{Subspace} provides a useful abstraction of a contiguous region of a heap,
recording its start and end address, the index of its first block, the size of 
blocks in this space and the number of blocks in the region. In particular,
\texttt{Subspace} provides methods to:
\begin{itemize}
\item Determine whether an address falls within a subspace;
\item Determine the block index of the address;
\item Calculate how much space remains in a block after a given address;
\end{itemize}





\T \newpage
\xname{componentOverview}
\section{Overview of \jrvm{} Structure}
This subsection briefly describes the overall structure of \jrvm.
Details of the various subsystems are provided in subsequent
sections.  

\subsection{Major Components of \jrvm}


\jrvm{} can be divided into the following major components:
\begin{description}
\item[Core runtime] (thread scheduler, class loader, library support,
verifier, etc.) This component is responsible for 
managing all the underlying data
structures required to execute applications and interfacing with
libraries.

\item[Compilers] (baseline, optimizing, JNI) This component is
responsible for generating executable code from bytecodes.

\item[Memory managers] This component is responsible for the
allocation and collection of objects during the execution of an
application. 

\item[Adaptive optimization system] This component is responsible
for profiling an executing application
and judiciously using the optimizing compiler to
improve its performance.
\end{description}

More details of each of these components are provided in the following sections.

\subsection{Package Structure}
The 2.2.0 release introduced packages into the system.  Prior releases
did not use packages explicitly, which resulted in all classes being in
the unnamed package.  

There are currently eight packages in \jrvm. All classes are in
one of these packages
\begin{description}
\item[\texttt{com.ibm.JikesRVM}] Classes for the core runtime, except for library
support.  This package also contains other classes that are not
included in one of the other packages, such as the baseline and JNI
compilers. 

\item[\texttt{com.ibm.JikesRVM.adaptive}] Classes for the adaptive optimization system

\item[\texttt{com.ibm.JikesRVM.classloader}] Implementation of classloaders
and associated data structures including the VM representation of classes,
methods, etc. 

\item[\texttt{\MMpackage{}\-.JMTk}] Classes in the newer JMTk
(Java\TMweb{} Memory Management Toolkit) collection of memory managers

\item[\texttt{\MMpackage\-.\vmInterface{}}] Classes related to
memory management that deal with the interface to the VM

\item[\texttt{com.ibm.JikesRVM.opt}] Classes related to the optimizing
compiler, except for IR-related classes

\item[\texttt{com.ibm.JikesRVM.opt.ir}] Classes related to the IR
(intermediate representation) of the optimizing compiler

\item[\texttt{com.ibm.JikesRVM.OSR}] Classes related to On-Stack-Replacement. 

\end{description}

The distributed directory structure
does not follow the Java convention that the source file directory tree
match the package structure, {\it i.e.},
there is no {\tt com/ibm/JikesRVM} directory anywhere under \texttt{\$RVM\_ROOT/rvm}.  Instead, the source directory structure follows a more logical 
structure.  The boot image builder
copies the source files from the \texttt{\$RVM\_ROOT/rvm} tree into a build
directory.  The scripts that perform this copy create the directory
structure required by Java semantics and place classes in appropriate
directories.

This approach avoids the need to change the directory structure as the
package structure evolves to become more fine-grained. If we
eventually arrive at a more fine-grained package structure (so that
each package contains a reasonably small number of classes) we may
convert the source tree to a package-oriented structure to facilitate
using standard Java IDE's to edit Jikes RVM source code.


\T \newpage
\xname{vmdetails}
\section{Runtime Implementation Details}
This section provides some information on various
implementation details for the Jikes RVM runtime system.

\subsection{Object Model} \label{sssec:objects}
\index{object model}
An {\em object model} dictates how to represent objects in storage;
the best object model will maximize efficiency of frequent language
operations while minimizing storage overhead. Jikes RVM's object model
is defined by \xlink{{\tt VM\_ObjectModel.java}}{\VMObjectModelURL}.

Values in the Java\trademark programming language are either {\em
primitive} (e.g. {\tt int}, {\tt double}, etc.)  or they are {\em
references} (that is, pointers) to objects.  Objects are either {\em
arrays} having elements or {\em scalar objects} having fields.  RVM's
object model is governed by the following criteria:
\begin{itemize}
\item{}
field and array accesses should be as fast as possible,
\item{}
null-pointer checks should be performed by the hardware if possible, 
\item{}
virtual method dispatch should be fast, and 
\item{}
other (less frequent) Java operations should not be prohibitively slow.
\item{}
per object storage overhead should be as small as possible
\end{itemize}

\index{field access}
\index{array access}
Assuming the reference to an object is in a register, the object's
fields can be accessed at a fixed displacement in a single
instruction.  To facilitate array access, the reference to an array
points to the first (zeroth) element of an array and the remaining
elements are laid out in ascending order.  The number of elements in
an array, its {\em length}, is kept just before its first
element. Elements of the array can be easily accessed via base +
scaled index addressing.

\index{NullPointerException}
The Java programming language requires that an attempt to access an
object through a {\tt null} 
object reference generate a NullPointerException.  In the RVM, references
are machine addresses, and {\tt null} is represented by address $0$.
The AIX\AIXTMFootnote operating system permits loads from low memory,
but accesses 
to very high memory (at small {\em negative} offsets from a null
pointer) normally cause hardware interrupts.\footnote{In AIX, it is at
least theoretically possible for another process to cause a shared
system library to get loaded into very high memory.  This remote
possibility is not a concern in a research project, but would need to
be addressed by a commercial JVM.  It would be sufficient to forbid
read and write access to the last page of addressable memory.
(Accesses to some of the fields of objects bigger than a page could be
checked explicitly without having a major impact on performance.)}
Thus, attempts to index off a null array reference are trapped by
the hardware, because array accesses require loading the array length
which is $-4$ bytes off the array reference.  A hardware null-pointer
check for field accesses is effected by locating fields at negative
offsets from the object reference.

\JavaTMFooter

\AIXTMFooter 

\subsection{Object headers} \label{sssec:headers}
\index{object header}
A two-word object header is associated with each object.  This header
supports virtual method dispatch, dynamic type checking, memory
management, synchronization, and hashing.  It is located $12$ bytes
below the value of a reference to the object.  (This leaves room for
the length field in case the object is an array, see
figure~\ref{fig:objects}.)

\index{locking}
\index{hashing}
One word of the header is a {\em status} word.  The status word is
divided into three {\em bit-fields}.  The first bit-field is used for
locking.  The second bit-field holds
the default hash value of hashed objects.  The third bit-field is used
by the memory management subsystem.  (The size of these bit-fields is
determined by build-time constants.)

\index{TIB}
\index{superclass}
\index{interfaces}
\index{virtual methods}
The other word of an object header is a reference to the {\em Type
Information Block} (TIB) for the object's class. This structure describes
the object's class including its superclass and the interfaces it implements
as well as has pointers to the class' virtual methods.

\subsection{Methods and Fields}\label{sssec:methods}
\index{methods}
\index{JTOC}
\index{TIB}
A compiled method body is an array of machine instructions (stored as
{\tt int}s). 
While the pointers for static fields and methods are stored in the 
{\em JikesRVM Table of Contents} (JTOC),
pointers for instance fields and virtual methods are stored in the class's TIB.
Consequently the dispatch mechanism is different for static and virtual 
methods.

\paragraph{The JTOC}
\index{JTOC}
\begin{figure}[htb]
\begin{gif}{jtoc}
\vbox{
\hbox{\psfig{file=jtoc.ps,height=3.5in}}
}\hfil
\end{gif}
\caption{The RVM Table Of Contents and other objects.}
\label{fig:jtoc}
\end{figure}
\index{literals}
\index{constants}
\index{dynamic type checking}
All of RVM's global data structures are stored in the JTOC. 
Literals, numeric
constants and references to String constants, are also stored there.
To enable fast common-case dynamic type checking, the JTOC also
contains references to the TIB for each class in the system.  
Since these 
structures can have many types and the JTOC is declared to be an array of 
{\tt int}s  
RVM uses a descriptor array, co-indexed with the JTOC, 
to identify the entries containing references.
A reference to the JTOC is maintained in a dedicated machine register 
(the JTOC register).
The JTOC
is depicted in figure~\ref{fig:jtoc}.  

\paragraph{Virtual Methods}
\index{virtual methods}
A TIB contains pointers to the compiled method 
bodies (executable code) for the virtual methods of its class. 
Thus, the TIB serves as RVM's virtual method table.
A
virtual method dispatch entails loading the TIB pointer at a fixed
offset off the object reference, loading the address of the method
body at a given offset off the TIB pointer, moving this address to the
PowerPC\PowerPCTMFootnote ``link-register'', and executing a
branch-and-link --- four 
instructions.

\paragraph{Static Fields and Methods} 
\index{static methods}
Static fields and methods are stored in the JTOC. Static method dispatch is 
simpler than virtual dispatch requiring only that the offset of method in the 
JTOC be read to find the address of the method. 

\paragraph{Lazy Method Compilation}
\index{lazy method compilation}
\index{deferred compilation}
\index{lazy method invocation stub}
The slots in the TIB or the JTOC may be filled in with 
a pointer to the compiled code for the method itself or if lazy method 
compilation is enabled and the method has not yet been compiled 
it may be filled in with
a pointer to the compiled code of the {\em lazy method invocation stub}.
If the lazy method invocation stub is invoked its action is to compile the 
method, substitute a pointer to the compiled code of the method in the slot in
the TIB or the JTOC from which it was invoked and then 
cause execution to jump to the start of the compiled method. 

\paragraph{Interface Methods}
\index{interface methods}
\index{IMT}
\index{conflict resolution stub}
Regardless of whether or not a method is overridden in a class that inherits it
virtual method dispatch is still very simple since the method body will be at
the same offset in the TIB in its defining class and in every class that 
inherits from it. 
However, where the method is an interface method, 
that is where it is invoked through an {\tt invoke\_interface} call rather than
an {\tt invoke\_virtual call}, its offset is not the same for every class that 
implements its interface and dispatch is more difficult.
The simplest, and least efficient way, of locating an interface method 
is to search all the virtual method entries in the TIB until a match is found.
Another way uses an {\em Interface Method Table} (IMT) which is much like the 
TIB. Any method that could be an interface method has a fixed offset into the 
IMT just as with the TIB. However, unlike in the TIB, two different methods may
share the same offset into the IMT. In this case, a {\em conflict resolution
stub} is inserted in the IMT. Conflict resolution stubs are
custom-generated machine code sequences that test the value of a
hidden parameter to dispatch to the desired interface method.

\PowerPCTMFooter

\subsection{VM Conventions}

%% footnotes not allowed in section headings, so we specialize
\htmlonly{\subsubsection{AIX\trademark VM Conventions}} \label{aix-conventions}
\texonly{\subsubsection{AIX\trademark VM Conventions}} \label{aix-conventions}

\index{stack conventions}
\index{register conventions}
\index{calling conventions}

This section describes register, stack, and calling conventions that apply to 
RVM on PowerPC\PowerPCTMFootnote.

Stackframe layout and calling conventions may evolve as our understanding
of the RVM's performance improves.  Where possible API's should be used
to protect code against such changes.  In particular, we may move to
the AIX conventions at a later date.  Where code differs from the AIX
conventions, it should be marked with a comment to that effect containing
the string "AIX".

\noindent{\bf Register conventions}

Registers (general purpose, gp, and floating point, fp) can be roughly
categorized into four types:

\begin{description}
\item [Scratch]
     Needed for method prologue/epilogue.  Can be used by compiler between
     calls.

\item[Dedicated]
     Reserved registers with known contents:
\begin{description}
\item [JTOC - JikesRVM Table Of Contents]
        Globally accessible data: constants, static fields and methods.

\item [FP - Frame Pointer]
        Current stack frame (thread specific).

\item [TI - Thread (locking) Id]
        Used to set (and test) the locking field of light weight object
        locks.  Can also be shifted to get the index of an object
        representing the current thread in into a global array.

\item [PR - Processor register]
        An object representing the current virtual processor (the one
        executing on the CPU containing these registers).  A field in
        this object contains a reference to the object representing
        the VM\_Thread being executed.
\end{description}

\item [volatile ("caller save", or "parameter")]
     Like scratch registers these can be used by the compiler as
     temporaries, but they are not preserved across calls.  (Volatile
     registers differ from scratch registers in that volatiles
     can be used to pass parameters and result(s) to and from
     methods.)

\item [Nonvolatile ("callee save", or "preserved")]
     These can be used (and are preserved across calls), but they must be
     saved on method entry and restored at method exit.  Highest numbered
     registers are to be used first.  (At least initially, nonvolatile
     registers will not be used to pass parameters.)

\item[Condition Register's 4-bit fields]
\begin{description}
\item    [CR0 - CR1] scratch

\item    [CR2 - TSCR] dedicated (thread switching, bit 8 TSCRB) (this
     convention is being phased out, and in fact is not being used in RVM 2.0)

\item    [CR3 - CR7] scratch
\end{description}
\end{description}


\noindent{\bf Stack conventions}

Stacks grow from high memory to low memory.
The layout of the stackframe appears in a block comment in
\xlink{{\tt \$RVM\_ROOT/rvm/src/vm/arch/powerpc/VM\_StackframeLayoutConstants.java}}
{\PPCStackframeLayoutURL}.

\noindent{\bf Calling Conventions}

\begin{description}
\item[Parameters]

    All parameters (that fit) are passed in VOLATILE registers.  Object
    reference and int parameters (or results) consume one GP register; long
    parameters, two gp registers (low-order half in the first);  float and
    double parameters, one fp registers.  Parameters are 
    assigned to registers
    starting with the lowest volatile register through the highest volatile
    register to the highest nonvolatile of the required kind (gp or fp).

    Any additional parameters are passed on the stack in an parameter spill
    area of the caller's stack frame.  The first spilled parameter occupies
    the lowest memory slot.  Slots are filled in the order that parameters
    are spilled.

    An int, or object reference, result is returned in the first volatile
    gp register; a float or double result is returned in the first volatile
    fp register; a long result is returned in the first two volatile gp
    registers (low-order half in the first);

\item [Method prologue responsibilities] (some of these can be omitted for leaf
  methods):

\begin{enumerate}
\item Save the caller's next instruction pointer (callee's return address,
       from the Link Register).

\item Save any nonvolatile floating-point registers used by callee.

\item Save any nonvolatile general-purpose registers used by callee.

\item Store and update the frame pointer FP.

\item Store callee's compiled method ID (eventually, this may not be needed).

\item Check to see if the Java\trademark thread must yield the VM\_Processor
(and yield if threadswitch was requested). 
\end{enumerate}

\item [Method epilogue responsibilities]

\begin{enumerate}
\item Restore FP to point to caller's stack frame.

\item Restore any nonvolatile general-purpose registers used by callee.

\item Restore any nonvolatile floating-point registers used by callee.

\item Branch to the return address in caller.
\end{enumerate}
\end{description}

\subsubsection{Linux/IA32 VM Conventions} \label{lintel-conventions}
\index{stack conventions}
\index{register conventions}
\index{calling conventions}

This section describes register, stack, and calling conventions that
apply to RVM on Linux/IA32.  {\em Linux/IA32 conventions are still
changing; be sure to check the relevant files for the most accurate
information!}

\noindent{\bf Register conventions}

\begin{description}
\item [EAX]
    First GPR parameter register, first GPR result value (high-order part
    of a long result), otherwise volatile (caller-save).

\item[ECX]
    Scratch.

\item[EDX]
    Second GPR parameter register, second GPR result value (low-order part
    of a long result), otherwise volatile (caller-save).

\item[EBX]
    Nonvolatile.

\item[ESP]
    Stack pointer.

\item[EBP]
    Nonvolatile.

\item[ESI]
    Processor register, reference to the VM\_Processor object for the current
    virtual processor.

\item[EDI]
    Nonvolatile.  (used to hold JTOC in baseline compiled code)

\end{description}

\noindent{\bf Stack conventions}

Stacks grow from high memory to low memory.
The layout of the stackframe appears in a block comment in
\xlink{{\tt
\$RVM\_ROOT/rvm/src/vm/arch/intel/VM\_StackframeLayoutConstants.java}}
{\LintelStackframeLayoutURL}.

\noindent{\bf Calling Conventions}

\begin{description}
\item[At the beginning of callee's prologue]
The first two areas of the callee's stackframe (see above) have been
     established.  ESP points to caller's return address.
     Parameters from caller to callee are as mandated by 
\xlink{{\tt
\$RVM\_ROOT/rvm/src/vm/arch/intel/VM\_RegisterConstants.java}}
{\LintelRegisterConstantsURL}.
\item[After callee's epilogue]
     Callee's stackframe has been removed.  ESP points to the word above where
     callee's frame was.  The framePointer field
     of the VM\_Processor object pointed to by ESI points to A's
     frame.  If B returns a floating-point result, this is at
     the top of the fp register stack.  If B returns a long, the
     low-order word is in EAX and the high-order word is in EDX.
     Otherwise, if B has a result, it is in EAX.

\end{description}

\JavaTMFooter

\AIXTMFooter

\PowerPCTMFooter

\subsection{Class Loading} \label{sssec:classLoading}
\index{class loading}

RVM implements the Java\trademark programming language's dynamic class
loading. While a class is being loaded it 
can be in one of five states. These are
\begin{description}
\item[vacant] a forward reference exists to the class but loading has not yet 
begun.
\item[loaded] the class's bytecode file has been read and parsed successfully.
\item[resolved] the superclass of this class has been loaded and resolved and
the offsets (whether in the object itself, the JTOC, or the class's TIB) of its 
fields and methods have been calculated.
\item[instantiated] the superclass has been instantiated and pointers to the
compiled methods have been inserted into the JTOC(for static methods) and the
TIB (for virtual methods).
\item[initializing] the superclass has been initialized and the class
initializer is being run.
\item[initialized] the superclass has been initialized and the class
initializer has been run.
\end{description}

The class passes through these states in the following fashion.

\paragraph{Vacant}
The 
\xlink{{\tt VM\_Class}}{\VMClassURL} 
object for this class has been created and registered. 
A class can be in this state if a reference to the class exists in a constant
pool of some other class.

\paragraph{Loaded} 
\index{constant pool}
In this state the class file has been read and parsed.  The constant pool has 
been constructed. The declared methods and fields of the class have been loaded.
Loading a method or field consists of reading its modifiers and attributes.

\paragraph{Resolved}
In this state the superclass of this class has been loaded and resolved. 
A list of the virtual methods and instance fields of this class, including the 
methods and fields
inherited from its superclass has been constructed and the offsets for the 
instance fields have been calculated.  
Space has been allocated in the JTOC for all static fields of the class and for
static method pointers and the appropriate offsets calculated.
The TIB has been initialized and offsets for the virtual methods have been
calculated.

\paragraph{Instantiated}
In this state the superclass of this state has been instantiated. 
The slots in the TIB are filled in with pointers to the compiled code for the 
virtual methods. 
The slots in the JTOC are filled in with pointers to the compiled code for the 
static methods.

\paragraph{Initializing} 
\index{class initializer}
In this state the superclass has been initialized. The class
initializer is being run. 

\paragraph{Initialized} 
\index{class initializer}
In this state the superclass has been initialized. The class initializer has 
been run. 

\JavaTMFooter

\subsection{Thread System}\label{sec:threads}

This section provides some explanation of how Java\trademark threads are
scheduled and synchronized by the RVM.

\index{threads}
\index{scheduling}
\index{locking}

All Java threads (application threads, garbage collector threads, {\em
etc.})  derive from 
\xlink{{\tt VM\_Thread}}{\VMThreadURL}.  
These threads are multiplexed by
one or more virtual processors (see 
\xlink{{\tt VM\_Processor}}{\VMProcessorURL}
).  Normally, the
number of RVM virtual processors to use is a command line argument
(e.g. {\tt -X:processors=4}) Generally, there should be one RVM
virtual processor for each CPU on an SMP.  Additional virtual
processors may be created to handle threads executing non Java code
through the Java JNI.  Multiple virtual processors require a working
pThread library, each virtual processor being bound to a pThread.  It
is possible to build a system that only uses one virtual processor by
setting the preprocessor directive {\tt
JVM\_WITH\_SINGLE\_VIRTUAL\_PROCESSOR} to 1.  This may give a minor
performance benefit on uniprocessors.

Threads that are not executing are either placed on thead queues
(deriving from 
\xlink{{\tt VM\_AbstractThreadQueue}}{\VMAbstractThreadQueueURL}
) or are proxied (see below).
Thread queues are either global or (virtual) processor local.  The
latter do not require synchronized access but global queues do.
Unfortunately, we did not see how to use Java monitors to provide
this synchronization.  (In part, because it is needed to implement
monitors, see below.)  Instead this low-level synchronization is
provided by 
\xlink{{\tt VM\_ProcessorLock}}{\VMProcessorLockURL}s.

Transferring execution from one thread (A) to another (B) is a complex
operation negotiated by the {\tt yield} and {\tt morph} methods of
VM\_Thread and the {\tt dispatch} method of VM\_Processor.  {\tt
yield} places A on an indicated queue (releasing the lock on the
queue, if it is global).  {\tt morph} uses 
\xlink{{\tt VM\_Magic}}{\VMMagicURL} 
to capture the
state of the running thread and transfers control to {\tt dispatch}.
At this point, the virtual processor is executing in {\em phantom
mode}, it is using A's stack, but not in a way that will be visible to
A when it next gets executed.  {\tt dispatch} removes B from a queue
of executable threads and, using more VM\_Magic, transfers control
to B's stack.  (To B, it looks as if {\em its} call to {\tt dispatch}
has just returned.)  To prevent a different processor from dispatching
A while it is still executing in phantom mode, {\tt yield} sets the
{\tt beingDispatched} field of A, which is only reset by the magic
that transfers control to B.

Currently, RVM has no priority mechanism, that is, all threads run
at the same priority.

Similarly, it has only the most rudimentary load balancing mechanism.
Each virtual processor has a (local) idle queue.  Normally, the
processor's \xlink{{\tt VM\_IdleThread}}{\VMIdleThreadURL} inhabits
this queue.  When no other thread is executable, this thread is
executed.  This thread requests work by setting the static {\tt
idleProcessor} field to its virtual processor.  (When {\tt dispatch}
is looking for a runnable thread it checks that this field is null.
If not, and it has an spare runnable thread, it places the spare
thread on the idle processor's transfer queue.)  The idle thread then
spins for a short period of time waiting for work to materialize.  If
it does, the idle thread returns to its processor's idle queue and
the processor resumes execution of the newly transferred thread.
Otherwise, the idle thread surrenders the remainder of its virtual
processor's time slice back to the operating system.

More sophisticated priority and load-balancing mechanisms are in
order.

RVM uses a light-weight locking scheme to implement Java monitors (see
\xlink{{\tt VM\_Lock}}{\VMLockURL}).  Twenty bits of the status word of 
the object header are used for locking.  If the top bit is set, the
bottom nineteen are an index into an array of heavy-weight locks.
Otherwise, if the object is locked, these bits contain the id of the
thread that holds the lock and a count of how many times it is held.
If a thread tries to lock an object locked with a light-weight lock by
another thread, it can spin, yield, or inflate the lock.  Spinning is
probably a bad idea.  The number of times to yield before inflating is
a matter open for investigation (as are a number of locking
issues, see {\tt VM\_Lock}).  Heavy-weight locks contain an {\tt
enteringQueue} for threads trying to acquire the lock.

A similar mechanism is used to implement Java wait/notification
semantics.  Heavy-weight locks contain a {\tt waitingQueue} for
threads blocked at a Java {\tt wait}.  When a {\tt notify} is
received, a thread is taken from this queue and transferred to a ready
queue.  Priority {\tt wakeupQueue}s are used to implement Java sleep
semantics.  Logically, Java timed-wait semantics entail placing a
thread on both a {\tt waitingQueue} and a {\tt wakeupQueue}.  However, our
implementation only allows a thread to be on one thread queue at
a time.  To accommodate timed-waits, both {\tt waitingQueue}s and
{\tt wakeupQueue}s are queues of {\em proxies} rather than threads.
A \xlink{{\tt VM\_Proxy}}{\VMProxyURL} can represent the same thread
on more than one proxy queue.

\JavaTMFooter

\subsection{VM Callbacks}\label{sssec:callbacks}

\index{callbacks}

The RVM provides callbacks for many runtime events of interest to the RVM
programmer, such as classloading, VM bootimage creation, and VM exit.  The
callbacks allow arbitrary code to be executed on any of the supported events.

The callbacks are accessed through the nested interfaces defined in the 
\xlink{{\tt VM\_Callbacks}}{\VMCallbacksURL} 
class.  There is one interface per event type.  To be notified
of an event, register an instance of a class that implements the corresponding
interface with {\tt VM\_Callbacks} by calling the corresponding {\tt add...()}
method.  For example, to be notified when a class is instantiated (see section
\ref{sssec:classLoading}), first implement the {\tt
VM\_Callbacks.ClassInstantiatedMonitor} interface, and then call {\tt
VM\_Callbacks.addClassInstantiatedMonitor()} with an instance of your class.
When any class is instantiated, the {\tt notifyClassInstantiated} method in
your instance will be invoked.

The RVM currently supports callbacks for the following events.
\begin{itemize}
\item {\tt ClassLoaded}, which happens when a class is {\em loaded}.
\item {\tt ClassResolved}, which happens when a class is {\em resolved}.
\item {\tt ClassInstantiated}, which happens when a class is {\em
instantiated}.
\item {\tt ClassInitialized}, which happens when a class is {\em initialized}.
\item {\tt MethodOverride}, which happens when a method in a newly loaded class
overrides a method in an existing class.
\item {\tt ForName}, which happens when java.lang.Class.forName() is invoked.
\item {\tt BootImageWriting}, which happens when boot image writing is started.
\item {\tt Exit}, which happens when the RVM is about to exit.
\end{itemize}
The appropriate interface names can be obtained by appending ``Monitor'' to the
event names (e.g. the interface to implement for the {\tt MethodOverride} event
is {\tt VM\_Callbacks.MethodOverrideMonitor}).  Likewise, the method to
register the callback is ``add'', followed by the name of the interface (e.g.
the register method for the above interface is {\tt
VM\_Callbacks.addMethodOverrideMonitor()}).

NOTE: there is currently no {\tt Startup} event.  It is in the process of being
implemented.

Since the events for which callbacks are available are internal to the RVM,
there are naturally some limitations on the behavior of the callback code.  For
example, as soon as the exit callback is invoked, all threads are considered
daemon threads (i.e. the VM will not wait for any new threads created in the
callbacks to complete before exiting).  Thus, if the exit callback creates any
threads, it has to {\tt join()} with them before returning.  These limitations
may also produce some unexpected behavior.  For example, while there is an
elementary safeguard on any classloading callback that prevents recursive
invocation (i.e. if the callback code itself causes classloading), there is no
such safeguard across events, so, if there are callbacks registered for both
{\tt ClassLoaded} and {\tt ClassInstantiated} events, and the {\tt
ClassInstantiated} callback code causes dynamic class loading, the {\tt
ClassLoaded} callback will be invoked for the new class, but not the {\tt
ClassInstantiated} callback.

Examples of callback use can be seen in the {\tt VM\_Controller} class in the
adaptive system and most {\tt VM\_Allocator} classes.



\subsection{Support for Soot-style Annotations}\label{sssec:annotations}
Jikes RVM optionally supports reading and using Soot-style class file
annotations.  Such annotations can specify that a null check or bounds
check is redundant for a particular byte code instruction.  These
annotations are produced by the {\bf Soot} class file optimizer
available from
\xlink{\SOOTURL}{\SOOTURL}.  

When the annotations command-line option is true, class file
annotations are processed by \xlink{{\tt
VM\_Method.java}}{\VMMethodURL} and stored internally in Jikes RVM.
During compilation (either baseline or optimizing), the compiler 
queries the {\tt VM\_Method} class for a particular bytecode to
determine if the generation of a null or bounds check can be
suppressed. 

Processing Soot-style annotations is not enabled, by default.  To
enable this support specify {\tt ``annotations=true''} to the
appropriate compiler.  For example, use {\tt -X:irc:annotations=true}
for non-adaptive images and {\tt -X:aos:irc:annotations=true} for
adaptive images.





\T \newpage
\xname{basedetails}
\section{Baseline Compiler Implementation Details}
\label{section:basedetails}
Describe baseline compiler implementation details.  TODO -- dave.



\T \newpage
\xname{quick_compiler}
\section{Quick Compiler}
\label{section:quick}
Release 2.3.4 marks the first public appearance of the ``Quick''
compiler for the PowerPC architecture targets.  

\xname{quick_compiler_goals}
\subsection{Goals}

The Quick compiler is supposed to be rather smarter than the Base
compiler about register use without being nearly as complex as the Opt
compiler. 

Rather smarter means that stack and local variables are kept in
registers if possible, not in memory. 

Not nearly as complex means, for instance, that the Quick compiler
code generator works in a single pass, with no intermediate
representations. Right now the only additional optimizations are to
remember if stack or local variable values are already in a temporary
or work register and hence don't need to be loaded, and to postpone
local variable stores until it becomes obvious that the value really
needs to be written. 

One purpose of the Quick compiler is to explore how close to the
performance of the Opt compiler we can come by using just the most
beneficial optimizations. Also, we hope to use it as a starting point
for experimenting with other optimizations without having to deal with
the implementation details of the Opt compiler and the complex
interactions among the many optimizations performed by the Opt
compiler. 

\xname{quick_compiler_why_use_it}
\subsection{Why use it?}

Baseline compiled code has horrible performance on modern (Power4,
Power5) PowerPC implementations.  The very frequent store/load to the
same memory location (a.k.a.\ the expression stack) causes a large number
of mis-speculations and replays. 

As of this writing (December 22, 2004), we are discussing moving to
using the quick compiler (in its simplest, fastest form) as a more or
less complete replacement for the base compiler on PowerPC.  Our guess
is that this is a more attractive option than injecting the quick
compiler as another level available to AOS.  It would be ok to rely on
the baseline compiler for some fringe stuff (DynamicBridge,
JNIFunctions, complex magics), but we'd like to see all normal Java
going to the quick compiler for initial compilation.

\xname{quick_compiler_status}
\subsection{Status}

As of this writing, the quick compiler is not part of the default
configurations (\textbf{prototype}, \textbf{prototype-opt},
\textbf{development}, and \textbf{production}).  It's been fairly
stable, but part of the gctest regression continues to fail, for
instance. 



\T \newpage
\xname{optdetails}
\section{Optimizing Compiler Implementation Details}
\label{section:optdetails}
This section provides some information on various
implementation details for the RVM optimizing compiler.

%%%%%%%%%%%%%%%%%%%%
\subsection{Options}

\index{command-line options}
\index{OPT\_Options class}
The command-line options to the optimizing compiler are
stored as fields in an object of type {\tt OPT\_Options}.
The RVM build process generates the {\tt OPT\_Options.java} 
file automatically from a template.  

\index{BooleanOptions.dat}
\index{ValueOptions.dat}
To add or modify the command-line options in {\tt OPT\_Options.java},
you must modify either {\tt BooleanOptions.dat} or 
{\tt ValueOptions.dat}.  You should describe your desired
command-line option in a format described below.
Your option will be generated the next time you build the
system.

%%%%%%%%%%%%%%%%%%%%%%%%%%%%%%%%%%
\subsubsection{BooleanOptions.dat}
\index{BooleanOptions.dat}

The {\tt BooleanOptions.dat} file defines boolean options for
the optimizing compiler.  Each command-line option is
described by a two-line record, and each record is separated
by a blank line.  Long lines can be partitioned using ``$\backslash$''.
{\bf NOTE:} blank lines {\em are} important!
Lines starting with ``\#'' are ignored.

The first line must have the following format
\begin{quote}
\begin{verbatim}
FULL_NAME OPT_LEVEL DEFAULT_VALUE {SHORT_NAME}
\end{verbatim}
\end{quote}
where
\begin{itemize}
\item {\tt FULL\_NAME} gives the name of the boolean field in {\tt OPT\_Options.java}
\item {\tt OPT\_LEVEL} gives the minimum optimization level that automatically sets this field true
\item {\tt DEFAULT\_VALUE} of {\tt true} or {\tt false}
\item {\tt SHORT\_NAME} is an optional field which defines a mnemonic by which the command-line processor recognizes this option.
\end{itemize}

The second line of each record must be a short textual description of
the semantics of the option.  This description will be printed
by {\tt -X:irc:help} (see Section~\ref{appendix:nonadaptive:cmdline}).

For example, the two line record in {\tt BooleanOptions.dat}
that defines the option of whether
to perform local scalar replacement is
\begin{verbatim}
LOCAL_SCALAR_REPLACEMENT 1 true local_sr
Perform local scalar replacement
\end{verbatim}

%%%%%%%%%%%%%%%%%%%%%%%%%%%%%%%%
\subsubsection{ValueOptions.dat}
\index{ValueOptions.dat}

The {\tt ValueOptions.dat} file defines non-boolean options for
the optimizing compiler.  Each command-line option is
described by a three-line record, and each record is separated
by a blank line.  As with {\tt BooleanOptions.dat},
long lines can be broken by using ``$\backslash$'' and
blank lines are once again significant.
Lines starting with ``\#'' are ignored.

The first line must have the following format
\begin{quote}
\begin{verbatim}
TAG FULL_NAME TYPE DEFAULT_VALUE {SHORT_NAME}
\end{verbatim}
\end{quote}
where
\begin{itemize}
\item {\tt TAG} is 'E' for an Enumeration type, and 'V' for a value type.  Further instructions for Enumeration types appear below.
\item {\tt FULL\_NAME} gives the name of the field in {\tt OPT\_Options.java}
\item {\tt TYPE} is one of 'byte', 'int', or 'String', and gives the primitive datatype for the value in OPT\_Options.java
\item {\tt DEFAULT\_VALUE} is the default value for the option
\item {\tt SHORT\_NAME} is an optional field which defines a mnemonic by which the command-line processor recognizes this option.
\end{itemize}

The second line of each record must be a short textual description of
the semantics of the option.  This description will be printed
by {\tt -X:irc:help} (see Section~\ref{appendix:nonadaptive:cmdline}).

The third line of each record is used for enumeration options, and must
be left blank for other options.

For example, the three-line record in {\tt ValueOptions.dat}
that defines the maximum inlining depth when using static inlining
heuristics is
\begin{verbatim}
V IC_MAX_INLINE_DEPTH int 5
Static inlining heuristic: Upper bound on depth of inlining
<blank line>
\end{verbatim}

Enumeration options provide a mechanism to define an option in terms of 
a small fixed set of choices.  For an enumeration option, the third line
of the record should contain a specification for each value that the
enumeration can take.  Each such specification must have the following
format:
\begin{verbatim}
"ITEM_NAME QUERY_NAME CMD_NAME"
\end{verbatim}
where
\begin{itemize}
\item {\tt ITEM\_NAME} gives the name of the enumeration value in {\tt OPT\_Options.java}
\item {\tt QUERY\_NAME} gives the name of an accessor function which returns {\tt true} iff the enumeration takes the value {\tt ITEM\_NAME}.
\item {\tt CMD\_NAME} is the name to pass on the command-line to set the enumeration to this value.
\end{itemize}
The quotes are important, and the specifications should be
space-separated.

For example, RVM supports a choice of three options for floating-point
optimization rules.  The three-line record describing these options is:
\begin{verbatim}
E FP_MODE byte FP_STRICT
Selection of strictness level for floating point computations
"FP_STRICT strictFP strict" \
"FP_ALLOW_FMA allowFMA allow_fma" \
"FP_LOOSE allowAssocFP allow_assoc"
\end{verbatim}
Notice how the third line was broken up by using ``$\backslash$''.

So, by default, RVM uses the {\em strict} floating-point semantics.  To use
the option that allows fused multiply-add instructions, 
specify {\tt -X:irc:allow\_fma} on the command-line.
Given an {\tt OPT\_Options} object called {\tt options}, your code can
query if fma is allowed by testing {\tt options.allowFMA()}.

%%%%%%%%%%%%%%%%%%%%%%%%%%%%%%%
\subsection{Method Compilation}
\label{sec:optdriver}
\index{compilation}
\index{optimizations}
\index{IR}
\index{HIR}
\index{LIR}
\index{MIR}
The fundamental unit for optimization in RVM is a single method. 
The optimization of a method consists of a series of 
compiler phases performed on the method. These 
phases transform the  
IR (intermediate representation) from bytecodes through 
HIR (high-level intermediate representation), 
LIR (low-level intermediate representation), and 
MIR (machine intermediate representation) and finally into machine code. 
Various optimizing transformations are performed at each level of IR.

\index{OPT\_CompilationPlan class}
\index{VM\_Method class}
\index{OPT\_OptimizationPlanElement class}
An object of the class {\tt OPT\_CompilationPlan} contains all the  
information necessary to generate machine code for a method. Two
important component of this class are the {\tt VM\_Method} to be
compiled and the
array of {\tt OPT\_OptimizationPlanElement}s to perform the compiling.
When the {\tt execute} method  of this class is called, machine code
is generated for the method as described by the 
{\tt OPT\_OptimizationPlanElement}s.

\index{OPT\_OptimizationPlanner class}
Another important class is {\tt OPT\_OptimizationPlanner}.  This class
contains a static field, called {\tt masterPlan}, which contains all
possible {\tt OPT\_OptimizationPlanElement}s.
The structure of the master plan is 
a tree. Any element may either be an atomic element (a leaf of the 
tree), or an aggregate element (an internal node of the tree).
The master plan has the following general structure:

\begin{itemize}
\item elements which convert bytecodes to HIR
\item elements which perform optimization transformations on the HIR
   \begin{itemize}
   \item elements which perform optimization transformations using SSA form
   \end{itemize}
\item elements which convert HIR to LIR
\item elements which perform optimization transformations on the LIR
   \begin{itemize}
   \item elements which perform optimization transformations using SSA form
   \end{itemize}
\item elements which convert LIR to MIR
\item elements which perform optimization transformations on MIR 
\item elements which convert MIR to machine code
\end{itemize}


\index{optimization plan}
A specific optimization plan is constructed by including all the 
{\tt OPT\_OptimizationPlanElement}s contained in the master plan which are 
appropriate for this compilation instance. 
Whether or not an element should be part of a compilation plan is determined 
by its {\tt shouldPerform} method. For each atomic element, the values in the
{\tt OPT\_Options} object are generally used to determine whether the element
should be included in the compilation plan. Each aggregate element must be 
included when any of its component elements must be included. 

Each element must have a {\tt perform} method defined which takes the IR as
a parameter. It is expected, but not required, that the {\tt perform}
method will modify the IR. 
The perform method of an aggregate element will invoke the 
perform methods of its elements.

\index{OPT\_CompilerPhase class}
Each atomic element is an object of the final class 
{\tt OPT\_OptimizationPlanAtomicElement}. The main work of this class
is performed by its {\em phase}, an object of type {\tt OPT\_CompilerPhase}. The
{\tt OPT\_CompilerPhase} class is not final; each phase overrides this class,
in particular it overrides the {\tt perform} method, which is invoked by its 
enclosing element's {\tt perform} method. All the state associated with 
the element
is contained in the {\tt OPT\_CompilerPhase}; no
state is in the element.

Every optimization plan consists of a selection of elements from the master 
plan;
thus though two optimization plans will be associated with different methods 
they
will share the same element objects. Clearly, it is not desirable that any state
associated with a particular compilation phase should be shared between two
different methods. In order to prevent this , the {\tt perform}
method of an atomic element creates a new instance of its phase immediately 
before calling the phase's {\tt perform} method. In the case where the phase
contains no state the {\tt newExecution} method of 
{\tt OPT\_CompilerPhase} can be overridden to return the phase itself rather 
than a clone of the phase.

%%%%%%%%%%%%%%%%%%%%%%%%%
\subsection{IR Operators}
\index{IR}
\index{instructions}
\index{operators}

The optimizing compiler intermediate representation (IR) includes a list
of instructions.  Each instruction includes an operator and zero or
more operands.

\index{OPT\_Operators class}
\index{OperatorList.dat}
The IR operators are defined by the class {\tt OPT\_Operators}, which in
turn is automatically generated from a template by a driver.  The input to the
driver are two files, both called {\tt OperatorList.dat}.  One input
file resides in {\tt vm/optimizingCompiler/IR} and defines machine-independent
operators.  The other resides in {\tt vm/arch/\{arch\}/optimizingCompiler/IR}
and defines machine-dependent operators, where \{arch\} is the
specific architecture of interest, such as powerPC.

Each operator in {\tt OperatorList.dat} is defined by a five-line record,
consisting of:
\begin{itemize}
\item {\tt SYMBOL}: a static symbol to identify the operator
\item {\tt INSTRUCTION\_FORMAT}: the instruction format class that accepts this operator.  See Section~\ref{iformats} for more information.
\item {\tt TRAITS}: a set of characteristics of the operator, composed with a bit-wise or ($|$) operator.  See {\tt OPT\_Operator.java} for a list of valid traits.
\item {\tt IMPLDEFS}: set of registers implicitly defined by this operator; usually applies only to machine-dependent operators
\item {\tt IMPLUSES}: set of registers implicitly used by this operator; usually applies only to machine-dependent operators
\end{itemize}

For example, the entry in {\tt OperatorList.dat} that defines the integer
addition operator is
\begin{verbatim}
INT_ADD
Binary
none
<blank line>
<blank line>
\end{verbatim}

The operator for a conditional branch based on values of two references is
defined by
\begin{verbatim}
REF_IFCOMP
IntIfCmp
branch | conditional
<blank line>
<blank line>
\end{verbatim}

Additionally,  the machine-specific {\tt OperatorList.dat} file contains 
another line of information for use by the assembler.  See the file for details.

%%%%%%%%%%%%%%%%%%%%%%%%%%%%%%%%
\subsection{Instruction Formats}\label{iformats}
\index{instructions}
\index{instructionFormats.java}

Every IR instruction fits one of the pre-defined {\em Instruction Formats}.
The package {\tt instructionFormats.java} defines roughly 75 architecture-independent
instruction formats.  For each instruction format, the package includes a class
that defines a set of static methods by which optimizing compiler
code can access an instruction of that format.

For example, {\tt INT\_MOVE} instructions conform to the {\tt Move}
instruction format.  The following code fragment shows code that uses the
{\tt OPT\_Operators} interface and the {\tt Move} instruction format:
\begin{verbatim}
import instructionFormats.*;
class X {
  void foo(OPT_Instruction s) {
    if (Move.conforms(s)) {     // if this instruction fits the Move format
      OPT_RegisterOperand r1 = Move.getResult(s);
      OPT_Operand r2 = Move.getVal(s);
      System.out.println("Found a move instruction: " + r1 + " := " + r2);
    } else {
      System.out.println(s + " is not a MOVE");
    }
  }
}
\end{verbatim}

This example shows just a subset of the access functions defined for the
Move format.  Other static access functions can set each operand 
(in this case, {\tt Result} and {\tt Val}), query each operand for
nullness, clear operands, create Move instructions, mutate other
instructions into Move instructions, and check the index of a particular
operand field in the instruction.  See the javadoc reference for a complete
description of the API.

\index{InstructionFormatList.dat}
Each fixed-length instruction format is defined in the text file 
{\tt \$RVM\_ROOT/rvm/src/vm/optimizingCompiler/IR/InstructionFormatList.dat}.
Each record in this file has four lines:
\begin{itemize}
\item {\tt NAME}: the name of the instruction format
\item {\tt SIZES}: the number of operands defined, defined and used, and used 
\item {\tt SIG}: a description of each operand, each description given
by
\begin{itemize}
\item {\tt D/DU/U}: Is this operand a def, use, or both?
\item {\tt NAME}: the unique name to identify the operand
\item {\tt TYPE}: the type of the operand (a subclass of {\tt OPT\_Operand}
\item {\tt [opt]}: is this operand optional?
\end{itemize}
\item {\tt VARSIG}: a description of repeating operands, used for
variable-length instructions.
\end{itemize}

So for example, the record that defines the {\tt Move} instruction format
is
\begin{verbatim}
Move
1 0 1
"D Result OPT_RegisterOperand" "U Val OPT_Operand"
<blank line>
\end{verbatim}

This specifies that the {\tt Move} format has two operands, one def and one
use.  The def is called {\tt Result} and must be of
type {\tt OPT\_RegisterOperand}.
The use is called {\tt Val} and must be of type {\tt OPT\_Operand}.

A few instruction formats have variable number of operands.  The
format for these records is given at the top of {\tt InstructionFormatList.dat}.
For example, the record for the variable-length {\tt Call} instruction
format is: 
\begin{verbatim}
Call
1 0 3 1 U 4
"D Result OPT_RegisterOperand" \
"U Address OPT_Operand" "U Method OPT_MethodOperand" "U Guard OPT_Operand opt"
"Param OPT_Operand"
\end{verbatim}
This record defines the {\tt Call} instruction format.  The second line
indicates that this format always has at least 4 operands (1 def and 3 uses),
plus a variable number of uses of one other type.  The trailing
4 on line 2 tells the template generator to generate special constructors
for cases of having 1, 2, 3, or 4 of the extra operands.
Finally, the record names the {\tt Call} instruction operands and
constrains the types.  The final line specifies the name and
types of the variable-numbered operands.  In this case, a {\tt Call}
instruction has a variable number of (use) operands called {\tt Param}.
Client code can access the {\tt i}th parameter operand of a {\tt Call}
instruction {\tt s} by calling {\tt Call.getParam(s,i)}.

A number of instruction formats share operands of 
the same semantic meaning and name.  For convenience in accessing
like instruction formats, the template generator supports four
common operand access types:
\begin{itemize}
\item {\tt ResultCarrier}: provides access to an operand of type {\tt OPT\_RegisterOperand} named {\tt Result}.
\item {\tt GuardResultCarrier}: provides access to an operand of type {\tt OPT\_RegisterOperand} named {\tt GuardResult}.
\item {\tt LocationCarrier}: provides access to an operand of type {\tt OPT\_LocationOperand} named {\tt Location}.
\item {\tt GuardCarrier}: provides access to an operand of type {\tt OPT\_Operand} named {\tt Guard}.
\end{itemize}

For example, for any instruction {\tt s} that carries a {\tt Result} operand
(eg. {\tt Move}, {\tt Binary}, and {\tt Unary} formats), client code can call
{\tt ResultCarrier.conforms(s)} and {\tt ResultCarrier.getResult(s)} to access
the {\tt Result} operand.

Finally, a note on rationale.  Religious object-oriented philosophers will
cringe at the InstructionFormats.  Instead, all this functionality could
be implemented more cleanly with a hierarchy of
instruction types exploiting inheritance.  We rejected the class hierarchy 
approach due to efficiency concerns of frequent virtual method dispatch
and type checks.  Recent improvements in our dynamic type checking
algorithms may alleviate this concern.

%%%%%%%%%%%%%%%%%%%%%%%
\subsection{BURS Rules}\label{burs}
\index{BURS}
\index{instruction selection}

The optimizing compiler uses the Bottom-Up Rewrite System (BURS) for
instruction selection.  BURS is essentially a tree pattern matching
system derived from Iburg by David R.\ Hanson.   (See ``Engineering a
Simple, Efficient Code-Generator Generator'' by Fraser, Hanson, and
Proebsting, LOPLAS 1(3), Sept.\ 1992.)
The instruction selection rules for each architecture are specified in an
architecture-specific file called {\tt LIR2MIR.rules}, which resides in
{\tt vm/arch/\{arch\}/optimizingCompiler/ConvertLIR2MIR}, where \{arch\} is the
specific architecture of interest, such as powerPC.  The rules are
used in generating a parser, which transforms the IR.

Each rule in {\tt LIR2MIR.rules} is defined by a four-line record,
consisting of:
\begin{itemize}
\item {\tt PRODUCTION}: the tree pattern to be matched.  The format of each
pattern is explained below.
\item {\tt COST}: the cost of matching the pattern as opposed to skipping
it.  It is a Java\trademark\ expression that evaluates to an integer.
\item {\tt FLAGS}: specifies whether the rule actually represents a sequence
of instructions (1) or a transformation of operands (0)
\item {\tt TEMPLATE}: Java\trademark\ code to emit
\end{itemize}

Each production has a {\em non-terminal}, which denotes a value, followed
by a colon (``:''), followed by a dependence tree that produces that value.
For example, the rule resulting in memory add on the INTEL architecture is
expressed in the following way:
\begin{verbatim}
stm:    INT_STORE(INT_ADD_ACC(INT_LOAD(r,riv),riv),OTHER_OPERAND(r, riv))
ADDRESS_EQUAL(P(p), PLL(p), 17)
1
EMIT(MIR_BinaryAcc.mutate(P(p), IA32_ADD, MO_S(P(p), DW), \
                          BinaryAcc.getValue(PL(p))));
\end{verbatim}
The production in this rule represents the following tree:
\begin{verbatim}
         r     riv
          \    /
         INT_LOAD  riv
             \     /
           INT_ADD_ACC  r  riv
                    \   |  /
                   INT_STORE
\end{verbatim}
where {\tt r} is a non-terminal that represents a register or a tree
producing a register, {\tt riv} is a non-terminal that represents a register
(or a tree producing one) or an immediate value, and {\tt INT\_LOAD},
{\tt INT\_ADD\_ACC} and {\tt INT\_STORE} are operators ({\em terminals}).
{\tt OTHER\_OPERAND} is just an abstraction to make the tree binary.

There are multiple helper functions that can be used in Java\trademark\ code (both cost
expressions and generation templates).  In all code sequences the name
{\tt p} is reserved for the current tree node.  Some of the helper methods
are shortcuts for accessing properties of tree nodes:
\begin{itemize}
\item {\tt P(p)} is used to access the instruction associated with the
current (root) node,
\item {\tt PL(p)} is used to access the instruction associated with the left
child of the current (root) node (provided it exists),
\item {\tt PR(p)} is used to access the instruction associated with the
right child of the current (root) node (provided it exists),
\item similarly, {\tt PLL(p)}, {\tt PLR(p)}, {\tt PRL(p)} and {\tt PRR(p)}
are used to access the instruction associated with the
left child of the left child, right child of the left child, left child of
the right child and right child of the right child, respectively, of the
current (root) node (provided they exist).
\end{itemize}

What the above rule basically reads is the following:\\
If a tree shown above is seen, evaluate the cost expression (which, in this
case, calls a helper function to test whether the addresses in the
{\tt STORE} ({\tt P(p)}) and the {\tt LOAD} ({\tt PLL(p)}) instructions are
equal.  The function returns 17 if they are, and a special value
{\tt INFINITE} if not), and if the cost is acceptable, emit the {\tt STORE}
instruction ({\tt P(p)}) mutated in place into a machine-dependent
add-accumulate instruction ({\tt IA32\_ADD}) that adds a given value to the
contents of a given memory location.

The rules file is used to generate a file called {\tt ir.brg}, which, in
turn, is used to produce a file called {\tt OPT\_BURS\_STATE.java}.

For more information on helper functions look at
{\tt \$RVM\_ROOT/rvm/src/vm/arch/\{arch\}/optimizingCompiler/ConvertLIR2MIR/OPT\_BURS\_Helpers.java}.
For more information on the BURS algorithm see
{\tt \$RVM\_ROOT/rvm/src/vm/optimizingCompiler/ConvertLIR2MIR/OPT\_BURS.java}.

%%%%%%%%%%%%%%%%%%%%%%%
\subsection{"Magic" Methods}\label{magic}
\index{magic methods}
\index{VM\_Magic}
\index{semantic inlining}
Certain methods, known as "magic" methods and declared as static methods of 
{\tt VM\_Magic}, are treated differently by the 
compiler. Because these methods access raw memory (such as portions of 
object headers) or registers, or are operating system calls they cannot be
implemented in Java\trademark\ code. Instead, for each of these methods, the Java\trademark\ 
instructions to generate the code is stored in 
{\tt OPT\_GenerateMagic} (to generate HIR) and {\tt VM\_MagicCompiler}
(to generate assembly code)\footnote{The optimizing compiler always uses the set
of instructions that generate HIR; the instructions that generate assembly code
are only invoked by the baseline compiler.}.

When a call site
is being compiled, and it is determined that the call target is one of these
magic methods, control is transferred to the list of instructions which will
generate the HIR for that method. The HIR for the magic method is always 
inlined into the caller method.

Magic macro methods are so called not because they are compiled as explained 
above, but because they are methods that consist mostly of calls to magic 
methods. Magic macro methods are compiled in the normal fashion, but their
magic methods are compiled as explained above. All magic macro methods are
inlined by the compiler.

\subsection{Adaptive Optimization System}\label{aos}

For a comprehensive discussion of the design and implementation of the
adaptive optimization system, see our 
\xlink{2000 OOPSLA paper}{\OOPSLAPaperURL}. 

For details on individual classes, see the source files under 
{\tt \$RVM\_ROOT/rvm/src/vm/adaptive}.


\T \newpage
\xname{MMTk}
\section{Memory Management Details}
\label{section:MMTk}
%      $Id$    

This section provides information on the implementation of JMTk, the
memory management component of Jikes\TMweb{} RVM's runtime system.

From Jikes RVM 2.2.0 onward, JMTk (Java Memory management
Toolkit) became the default memory management system for Jikes RVM.\@
JMTk is designed to be a portable toolkit, with Jikes
RVM-specific code factored out as far as possible.

\subsection{Directory Structure and Packages} \label{sssec:directories}

JMTk classes are contained in the
\texttt{\$RVM\_\-ROOT/rvm/src/vm/memoryManagers/JMTk} directory.

In keeping with the goal of modularity and portability, as far as
possible Jikes-RVM-specific code is factored out.  Such VM-specific
code resides in a separate package,
(\xlink{\texttt{\MMpackage\-.\vmInterface}}{\vmInterfacePackageURL}),
and can be found in the \texttt{\vmInterface} sub-directory.  All other
code is part of the
\xlink{\texttt{\MMpackage\-.JMTk}}{\JMTkPackageURL}
package.

The \texttt{plan} sub-directory contains classes that define
\emph{memory management plans}.  A plan specifies a particular
configuration of JMTk components which together define the memory
management strategy for a particular build of Jikes RVM.\@  The
\texttt{policy} sub-directory contains classes implementing various
memory management policies (such as \xlink{mark-sweep
    collection}{\MarkSweepLocalURL}, \xlink{free-list
    allocation}{\SegregatedFreeListURL}, \xlink{bump-pointer
    allocation}{\BumpPointerURL}, etc.).  The \texttt{utility}
sub-directory contains classes implementing generic utilities (such as
\xlink{load-balancing parallel dequeues}{\SharedDequeURL},
\xlink{sequential store buffers}{\LocalSSBURL}, etc.).

\subsection{Choosing a Garbage Collector} \label{ssec:choosinggc}

Depending on your purposes, you may choose to build Jikes RVM with one
of the following plans:
\begin{itemize}
\item \texttt{SemiSpace} (copying),
\item \texttt{MarkSweep} (non-copying),
\item \texttt{CopyMS} (non-generational copy/mark-sweep hybrid)
\item \texttt{GenCopy} (classic copying generational),
\item \texttt{GenMS} (generational with mark-sweep mature space), or
\item \texttt{RefCount} (a reference counting collector with non-concurrent cycle collection)
\item \texttt{NoGC} (allocation only, no garbage collection)
\item \texttt{GCTrace} (copying collector used for heap trace generation)
\end{itemize}
The relative performance of these collectors is highly dependent on
the application. 

\texttt{NoGC} is provided for pedagogical and
experimental purposes only; it is not suitable for general use.
\texttt{GCTrace} is provided for generation of GC traces and is not 
useful for collector comparisons.
GC researchers will of course want to experiment with other GC choices.
The \texttt{RefCount} collector is currently slightly less stable than
the others and is not quite ready for heavy usage.

All of the memory managers (except \texttt{NoGC} and \texttt{RefCount}) 
support finalization and weak, soft, and phantom references.  They are all parallel 
and load-balancing.  A collection can proceed even when some threads are
executing in native code. When a collection starts, threads in native
code are blocked from returning to Java code for the duration of that
collection.

\subsection{Plans}%
        \label{sssec:plans}%
        \index{garbage collection}%
        \index{stop-the-world garbage collection}

\IndexttClass{BasePlan}%
All plans inherit from \xlink{\texttt{BasePlan}}{\BumpPointerURL},
and all plans in this release are ``stop-the-world'' collectors, so
they all inherit from
\IndexttClass{StopTheWorldGC}%
\xlink{\texttt{StopTheWorldGC}}{\StopTheWorldGCURL}, which implements
basic stop the world GC functionality.  The two generational
collectors both inherit from
\IndexttClass{Generational}%
\xlink{\texttt{Generational}}{\GenerationalURL}, which includes
common nursery and write barrier implementations.

All JMTk plans support parallel allocation and collection (with the
exception of \IndexTexttt{RefCount}, which does not perform parallel
collection).  To minimize synchronization overheads, unsynchronized
\emph{thread-local} actions are distinguished from \emph{global}
actions, which must only be performed by a single thread.  Global
state is held in each plan's class variables, while instance variables
reflect thread-local state, each \texttt{Plan} instance bound to a
\xlink{\IndexTexttt{VM\_Processor}}{\VMProcessorURL} instance.

The basic functions of each plan include:
\begin{itemize}
\item Identifying a virtual memory layout (using
  \xlink{\texttt{VMResource}}{\VMResourceURL} to, for example, bind a
  semi-space or the nursery to a particular address range).  It is
  important to note that only the \emph{virtual memory} layout is
  partitioned statically.  Actual \emph{memory usage} is dynamically
  spread among the various spaces (such as nursery, mature, metadata,
  etc.), the only constraint being that the total memory usage remain
  within the memory available.
\item Providing allocation by binding suitable allocators to different
  \texttt{VMResource}s.
\item Invoking collection when necessary through the use of a
  \emph{polling} mechanism.
\item Applying the appropriate collection policies to objects
  encountered during the collection process (objects may be subject to
  different collection regimens depending on where they reside in
  memory).
\item Implementing read and write barriers if necessary.
\end{itemize}

It is not difficult to add your own memory management plan (allocator
and collector) to JMTk, especially if it uses the same ``stop the
world'' parallel collection strategy used by all the collectors in
this release.  A good way to start is to compare some of the different
plans and understand the significance of the differences.

The basic steps are:

\cindex[jconfigure script]{\texttt{jconfigure} script}%
\IndexttClass{Plan}%
\begin{enumerate}
\item Create a new directory within the \texttt{plan} subdirectory, such as
  ``\texttt{NewGC}''.
\item Add a new configuration in \texttt{\$RVM\_\-ROOT/rvm/config/build}
  which includes your new directory in the build.  Name it
  appropriately, such as ``\texttt{BaseBaseNewGC}''.
\item Modify \texttt{\$RVM\_\-ROOT/rvm/bin/jconfigure} to handle the new
  collector sub-directory.
\item Copy the contents of some existing plan directory into your new
  directory, and modify the files (\texttt{Plan.java} and
  \texttt{Header.java}), choosing a starting point that has similar
  properties, such as copying or non-copying, generational or
  non-generational.
\end{enumerate}

\subsection{Policy} \label{sssec:policy}

The \texttt{policy} sub-directory contains implementations of key
memory management policy choices, such as \xlink{bump-pointer
    allocation}{\BumpPointerURL} and \xlink{free-list
    allocation}{\MarkSweepLocalURL}, \xlink{copying
    collection}{\CopyURL} and \xlink{mark-sweep
  collection}{\MarkSweepSpaceURL}.

\subsection{Utility} \label{sssec:utility}

The \texttt{utility} sub-directory contains basic utilities and
mechanisms, including:
\begin{itemize}
\item \xlink{Load-balancing shared parallel
    dequeues}{\SharedDequeURL}, which provide load balancing double
  headed queue management for \xlink{address
    dequeues}{\AddressDequeURL} (used by the GC work queue), and
  \xlink{sequential store buffers}{\LocalSSBURL} (used by some write
  barriers as a remembering mechanism).
\item \xlink{Memory resources}{\MemoryResourceURL}, which are the
  mechanism for space \emph{accounting}.  Memory resources are used
  for accounting for all space, including space used by meta data
  (such as queues, etc.).
\item \xlink{Virtual Memory Resources}{\VMResourceURL} (VM
  Resources), which are mechanism for virtual memory \emph{mapping}.
  VM resources are used to associate regions of virtual memory with
  particular policies or needs (such as the nursery of a generational
  collector, or a region of memory where meta data resides, etc), and
  allow multiple allocators to consume each such space. There are a
  number of different VM resources, for \xlink{monotonic
      allocation}{\MonotoneVMResourceURL} (used by bump pointer
  allocators), \xlink{free list allocation}{\FreeListVMResourceURL}
  (used by free list allocators), etc.
\item Low level tools for \xlink{\texttt{mmap}ping}{\LazyMmapperURL}
  memory on demand, and for \xlink{allocating raw
      memory}{\RawPageAllocatorURL} (for use by meta data, for example).
\end{itemize}

\subsection{\texttt{VMInterface}} \label{sssec:vminterface}

The \texttt{vmInterface} sub-directory provides the interface between
Jikes RVM and JMTk.  The primary interface is provided by
\xlink{\texttt{VM\-In\-ter\-face\-.java}}{\VMInterfaceURL}.  Key
VM-dependent mechanisms include:

\begin{itemize}
\item \xlink{Object}{\ScanObjectURL},
  \xlink{statics}{\ScanStaticsURL}, and
  \xlink{thread}{\ScanThreadURL} scanning,
\item \xlink{GC map iteration}{\VMGCMapIteratorURL}, and
\item GC \xlink{initiation}{\VMCollectorThreadURL} and \xlink{synchronization}{\VMHandshakeURL}.
\end{itemize}

\subsection{\texttt{Generating GC Traces}} \label{sssec:gctrace}
Builds using the \texttt{GCTrace} plan produce accurate heap traces of
the running program.  These builds output records for every object
creation (both objects in the boot image and heap allocations),
reference update, and the last time each object is reachable.
Generated traces are output as normal garbage collection information
and can be captured similarly.

To record the exact state of the heap, the trace format's object
creation record differentiates objects created in the boot image,
objects allocated into the immortal region of the heap, and the
remaining object allocations.  To enable accurate garbage collection
simulations, last reachable times are recorded for all these objects.

Last reachable times for each object are recorded at a granularity
measured in terms of bytes allocated.  The earliest an object can be
reclaimed is at the next ``perfectly accurate'' allocation following
this time.  To compute these times relatively quickly, JMTk uses the
Merlin lifetime analysis algorithm.  For more information about trace
granularity or the Merlin algorithm, see the SIGMETRICS 2002 paper by 
Hertz, Blackburn, McKinley, Moss, and Stefanovic.

Trace generation defaults to using the largest possible granularity,
but finer traces can be specified by specifying the \textbf{logarithm}
of the desired granularity with the \texttt{traceRate} command-line
option.  Because garbage collection cannot occur at every allocation
(i.e., before the Jikes RVM has finished booting), traces include
``weak'' allocations no matter the granularity specified.  At the
finest granularity, these ``weak'' allocations record only when
garbage collection is prohibited and objects cannot therefore be
reclaimed.



\T \newpage
\xname{aosdetails}
\section{Adaptive Optimization System}
\label{section:aosdetails}
For a comprehensive discussion of the design and implementation of the
adaptive optimization system, see our 
\xlink{2000 OOPSLA paper}{\OOPSLAPaperURL}. 

For details on individual classes, see the source files under 
{\tt \$RVM\_ROOT/rvm/src/vm/adaptive}.

In summary, version 2.0 of the Jikes\trademark RVM includes basic AOS
functionality to identify and recompile program hot spots,
context-insensitive online profile-directed inlining, and basic
support for dynamic instrumentation. 

\JikesTMFooter



\T \newpage
\xname{magic}
\section{Magic}
This section provides information on ``magic'' which is an escape
hatch that Jikes\TMweb{} RVM provides to implement functionality that
is not possible using the pure Java\TMweb{} programming language.  For
example, the Jikes RVM garbage collectors and runtime system must, on
occasion, access memory or perform unsafe casts.  Users are {\it
strongly} discouraged from using magic in their code except where
absolutely necessary.

We are currently in the midst of a major rework of the ``syntax'' for
magic. The general trend is that we are replacing static methods of
{\tt VM\_Magic} with virtual methods defined on classes such as {\tt
VM\_Address}. This results in more natural (and less verbose) usages
in code that is manipulating raw memory. As of version 2.3.4, this
transition is complete in MMTk, but only partially done in the rest of
Jikes RVM. Completing this transision is one of the work items for the
3.0 release of Jikes RVM. 

There are currently three types of magical operations which are
described in the remainder of this section.  
%
The first is a collection
of magical methods that are static methods of the class {\tt
VM\_Magic}\IndexttClass{VM\_Magic}.  
%
The second is the magical classes 
{\tt VM\_Address}\IndexttClass{VM\_Address}, 
{\tt VM\_Word}\IndexttClass{VM\_Word}, 
{\tt VM\_Offset}\IndexttClass{VM\_Offset}, and 
{\tt VM\_Extent}\IndexttClass{VM\_Extent} used in parts of the 
runtime and garbage collector.
%
The third is various mechanisms to declare code {\em uninterruptible}.

\subsection{\texttt{VM\_Magic}}%
\index{magic methods}%
\index{semantic inlining}%
Certain methods in the class \xlink{\texttt{VM\_Magic}}{\VMMagicURL} are
treated differently by the compiler. Because these methods access raw
memory or other machine state, perform unsafe casts, or are operating
system calls, they cannot be implemented in Java code.  A
Jikes\TMweb{} RVM implementor must be {\em extremely careful} when
writing code that uses {\tt VM\_Magic} to circumvent the Java type
system.  The use of \xlink{{\tt VM\_Magic.objectAsAddress} to perform various
forms of pointer arithmetic}{\VMMagicURL\#objectAsAddress(java.lang.Object)} is especially hazardous, since it can
result in pointers being ``lost'' during garbage collection.  All such
uses of magic must either occur in \link{uninterruptible methods}[
(see below, Section~\Ref, page~\Pageref)]{uninterruptible_code}
or be guarded by calls to \xlink{{\tt VM.disableGC}}{\VMURL\#disableGC()} and \xlink{{\tt VM.enableGC}}{\VMURL\#enableGC()}.
The optimizing compiler performs aggressive inlining and code motion, so
not explictly marking such dangerous regions in one of these two
manners will lead to disaster.

Since magic is inexpressible in the Java programming language , it is
unsurprising that the bodies of {\tt VM\_Magic} methods are undefined.
Instead, for each of these methods, the Java instructions to generate
the code is stored in
\xlink{{\tt OPT\_GenerateMagic}}{\OPTGenerateMagicURL} and 
\xlink{{\tt OPT\_GenerateMachineSpecificMagic}}{\OPTGenerateMachineSpecificMagicURL} (to generate HIR) and 
\xlink{{\tt VM\_Compiler}}{\VMCompilerURL} (to generate assembly code)\footnote{The optimizing
compiler always uses the set of instructions that generate HIR; the
instructions that generate assembly code are only invoked by the
baseline compiler.}.  Whenever the compiler encounters a call to one of these
magic methods, it inlines appropriate code for the magic method into the caller method.

\subsection{VM\_Address}\IndexttClass{VM\_Address}
The type \xlink{\texttt{VM\_Address}}{\JikesRVMJavadocURL/VM\_Address.html} is used to represent a machine-dependent
address type.  In the past, the base type {\tt int} was used to
represent addresses but this approach had several shortcomings.
First, the lack of abstraction makes porting nightmarish.  Equally
important is that Java type {\tt int} is signed whereas address are
more appropriately considered unsigned.  The difference is problematic
since an unsigned comparison on {\tt int} is inexpressible in the Java
programming language.

To overcome these problems, instances of \xlink{{\tt VM\_Address}}{\JikesRVMJavadocURL/VM\_Address.html} are used to
represent addresses.  The class supports the expected well-typed
methods like adding an integer offset to an address to obtain another
address, computing the difference of two addresses, and comparing
addresses.  Other operations that make sense on {\tt int} but not on
addresses are excluded like multiplication of addresses.  Two methods
deserve special attention: converting an address into an integer and
the inverse.  These methods should be avoided where possible.

Without special intervention, using a Java object to represent an
address would be at best abysmally inefficient.  Instead, when the
Jikes compiler encounters creation of an address object, it will
return the primitive value that represents an address for that
platform.  Currently, the address type maps to a 32-bit unsigned
integer.  Since an address is not really an object, the following must
be kept in mind:

\begin{enumerate}
\item{} Do not pass a \xlink{{\tt VM\_Address}}{\JikesRVMJavadocURL/VM\_Address.html}  instance where an {\tt Object}
is expected. This will type-check, but it is {\em not} what you want.  A
corollary is to avoid overloading a method where the two overloaded
versions of the method can only be distinguished by operating on an
{\tt Object}  versus a {\tt VM\_Address}. 
\item{} Do not synchronize on a {\tt VM\_Address} instance.
\item{} Due to a current shortcoming in the way {\tt VM\_Address} works, do not make an
array of {VM\_Address} values.  Instead make a
\xlink{{\tt VM\_AddressArray}}{\JikesRVMJavadocURL/VM\_AddressArray.html}.
  Creating {VM\_AddressArray[]} works as expected and is allowed.
\end{enumerate}


\subsection{What are the Semantics of Uninterruptible Code?\label{uninterruptible_code}}%
\index{uninterruptible}%
\index{interruptible}%
\Indextt{VM\_Uninterruptible}%
\Indextt{VM\_PragmaInterruptible}%
\Indextt{VM\_PragmaUninterruptible}%
\Indextt{VM\_PragmaLogicallyUninterruptible}%
\index{yield point}%

Declaring a method uninterruptible enables a Jikes RVM developer to
prevent the Jikes RVM compilers from inserting ``hidden'' thread
switch points in the compiled code for the method.  As a result, the
code can be written assuming that it cannot involuntarily ``lose
control'' while executing due to a timer-driven thread switch. In
particular, neither yield points nor stack overflow
checks will be generated for uninterruptible methods. 

When writing uninterruptible code, the programmer is restricted to a
subset of the Java language.  The following are the restrictions on
uninterruptible code.
\begin{enumerate}
\item{} Because a stack overflow check represents a potential yield
point (if GC is triggered when the stack is grown), stack overflow
checks are omitted from the prologues of uninterruptible code.  As a
result, all uninterruptible code must be able to execute in the
stack space available to them when the first uninterruptible method on
the call stack is invoked.  This is typically about 8K for
uninterruptible regions called from mutator code.  The collector
threads must preallocate enough stack space, since all collector code
is uninterruptible. As a result, using recursive methods in the GC
subsystem is a bad idea.
\item{} Since no yield points are inserted in uninterruptible code,
there will be no timer-driven thread switches while executing it.  So,
if possible, one should avoid ``long running'' uninterruptible methods
outside of the GC subsystem.

\item{} Certain bytecodes are forbidden in uninterruptible code,
because Jikes RVM cannot implement them in a manner that ensures
uninterruptibility. The forbidden bytecodes are: {\instruction aastore};
{\instruction invokeinterface}; {\instruction new}; {\instruction
newarray}; {\instruction anewarray}; {\instruction athrow};
{\instruction checkcast} and
{\instruction instanceof} unless the LHS type is a final class;
{\instruction monitorenter},
{\instruction monitorexit}, {\instruction multianewarray}. 
\item{} Uninterruptible code cannot cause class loading and thus must
not contain unresolved {\instruction getstatic}, {\instruction
putstatic}, {\instruction getfield}, {\instruction putfield},
{\instruction invokevirtual}, or {\instruction invokestatic} bytecodes. 
\item{} Uninterruptible code cannot contain calls to interruptible
code. As a consequence, it is illegal to override an uninterruptible
virtual method with an interruptible method.
\item{} Uninterruptible methods cannot be synchronized. 
\end{enumerate}

We have augmented the baseline compiler to print a warning message
when one of these restrictions is violated.  Because there are still a
small number of violations in Jikes RVM, this checking is not enabled
by default, but can be enabled by setting \xlink{{\tt
VM\_Configuration.VerifyUnint}}{\JikesRVMJavadocURL/VM\_Configuration.html#VerifyUnint} to  \texttt{true}. 
If uninterruptible code were to raise a runtime exception such
as NullPointerException, ArrayIndexOutOfBoundsException, or
ClassCastException, then it could be interrupted.  We assume that such
conditions are a programming error and do not flag bytecodes that
might result in one of these exceptions being raised as a violation of
uninterruptibility. Checking for a particular method can be disabled
by having the method throw the exception \xlink{{\tt
VM\_PragmaLogicallyUninterruptible}}{\JikesRVMJavadocURL/VM\_PragmaLogicallyUninterruptible.html}.  This should be done with extreme
care, but in a few cases is necessary to avoid spurious warning
messages. 

The following rules determine whether or not a method is
uninterruptible.
\begin{enumerate}
\item{} All class initializers are interruptible, since they
can only be invoked during class loading.
\item{} All object constructors are interruptible, since they an
only be invoked as part of the implementation of the new bytecode.
\item{} If a method throws the exception \xlink{{\tt
VM\_PragmaInterruptible}}{\JikesRVMJavadocURL/VM\_PragmaInterruptible.html} then it is interruptible.
\item{} If none of the above rules apply and a method throws the
exception 
\xlink{{\tt VM\_\-Prag\-ma\-Un\-in\-ter\-rup\-ti\-ble}}{\JikesRVMJavadocURL/VM\_PragmaUninterruptible.html}, then it is uninterruptible.
\item{} If none of the above rules apply and the declaring class of
the method directly implements the interface {\tt VM\_\-Un\-in\-ter\-rup\-ti\-ble}
then it is uninterruptible.
\end{enumerate}
Whether to use {\tt VM\_Uninterruptible} or the {\tt
VM\_PragmaUninterruptible} is a matter of taste and mainly depends on
the ratio of interruptible to uninterruptible methods in a class.  If
most methods of the class should be uninterruptible, then implementing
{\tt VM\_Uninterruptible} is preferred. 



\T \newpage
\xname{preprocessor}
\section{Java Preprocessor}
\label{section:preprocessor}
As of this writing, about twelve percent (128 out of 1045) of the Java
source code in Jikes RVM contains preprocessor constructs.  The
preprocessor syntax has not been documented until this writing (July,
2003).  Here is the help message the preprocessor now displays; this
is a placeholder until someone writes more attractive documentation
for it.

\subsection{Usage}

\begin{verbatim}
Usage: preprocessModifiedFiles [--help]
        [ --disable-modification-exit-status ] [--trace]
        [ -D<name>[ =1 | =0 | =<string-value> ] ]... 
        [ -- ] <output directory> [ <input file> ]...
   Preprocess source files that are new or have changed.

   The timestamp of each input file is compared with that
   of the corresponding file in the <output directory>.  If the
   output file doesn't exist, or is older than the input file,
   then the input file is copied to the <output directory>, 
   with preprocessing.

   Invocation parameters:
      - zero or more preprocessor directives of the form "-D<name>=1", of the
        equivalent shorthand form "-D<name>", of the form "-D<name>=0",
         and/or of the form "-D<name>=<string-value>".
      - name of directory to receive output files
      - names of zero or more input files
      - other flags

   Process exit status means:
           0 - no files changed
           1 - some files changed
       other - trouble

   With --disable-modification-exit-status, the process will
   exit wtih status 0 even when some files changed.  Under
   --disable-modification-exit-status, non-zero exit status
   always means trouble.


   --trace  The preprocessor prints a '.' for each file that did
          not need to be changed and a '+' for each file that needed
          preprocessing again.

   --verbose, -v  The preprocessor prints a message for each file
         examined, and prints a summary at the end 

   --help, -h  Show this long help message and exit with status 0.

   -D<name>=0 is a no-op; equivalent to never defining <name>.

   -D<name>=1 and -D<name> are equivalent.

   -D<name>=<any-string-value-but-0-or-1> will define a constant that is
       usable in a //-#value dirctive.

   The following preprocessor directives are recognized
   in source files.  They must be the first non-whitespace characters
   on a line of input.

      //-#if    <name>
            It is not an error for <name> to be undefined.  Only checks
            whether <name> is defined.

            "//-#if" also supports the constructs '!' (invert the sense of 
            the next test), '&&', and '||'.  '!' binds more tightly 
            than '&&' and '||' do.   '&&' and '||' are at the same precedence.
            The preprocessor does not support parentheses in //-#if constructs
            If you don't mix '&&' and '||' in the same line, you'll be OK.

      //-#elif  <name>
            Takes the same arguments that //-#if does. 

      //-#else  <optional-comment>

      //-#endif <optional-comment>

      //-#value <preprocessor-symbol>
            <-preprocessor-symbol> is the name of a constant defined on the
            command line with -D; it will be replaced with the defined value.

            It is an error for <preprocessor-symbol> not to be defined.

            It is an error for <preprocessor-symbol> to have been defined with
            -D<name>=1 or with -D<name>

           (This is an odd restriction, but is the way the code was written
           when I found it.  You're free to rewrite it if you want it to act
           just like the C preprocessor does.)

     There is no equivalent to the C preprocessor's "#define" construct;
     all constants are defined on the command line with "-D".
\end{verbatim}

\subsection{Where It Is}

The source code for the preprocessor is in {\tt
rvm/src/tools/preprocessor}.  The Jikes RVM build process installs it
as RVM\_BUILD{\tt /jbuild.prep}.  The preprocessing is done automatically
during the build process.




\T \newpage
\xname{jni}
\section{JNI Implementation Details}
\label{section:jni}
Describe implementation of JNI system.


\T \newpage
\xname{libraries}
\section{Class Libraries}
\label{section:libraries}
 As of version 2.2.1, Jikes\JikesTMFootnote\ RVM has adopted the
\xlink{GNU Classpath}{http://www.classpath.org} libraries.  GNU
Classpath is an ongoing project working towards a complete set of
Java\JavaTMFootnote\ libraries, and rough guides to what portions of
the libraries exist can be found on the GNU Classpath web site.  Jikes
RVM runs with unmodified versions of GNU Classpath straight from its
ftp site, and the Jikes RVM configuration process checks out
appropriate versions of the library as needed.  Thus, the integration
can be seemless from a user's perspective.

 The automatic, seamless use of GNU Classpath is appropriate for
nearly all users of Jikes RVM, and is strongly recommended.  There are
two other configurations that work in this release, but both are
subject to change without notice in CVS head (including possible
deletion). 

\begin{description}
\item[Manual GNU Classpath] If you set the CLASSPATH\_SRC variable in
your Jikes RVM configuration file, the build process will use the
libraries that are located there.  The setup must mimic the one
generated automatically, which means that
CLASSPATH\_SRC/`config.guess` must cointain a build for your platform
and CLASSPATH\_SRC/classpath must be a version of GNU Classpath from
CVS.  This option is primarily for people who want to work on the
Jikes RVM integration with GNU Classpath; it will be maintained in
some form in the future, although the mechanics may change
arbitrarily.

\item[Hybrid OTI-GNU Classpath] This is what was standard in
the last release: a few core packages come from our old OTI-derived
libraries and the rest is from GNU Classpath.  To use these libraries
henceforth, you must give the RVM\_WITH\_GNU\_CLASSPATH=0 option to
jconfigure.  This configuration is significantly less functional than
the GNU Classpath-only version, and is provided in this release for
migration purposes only.  {\em This configuration is deprecated and
will be deleted in the future; you are STRONGLY DISCOURAGED from using
it for ANY purpose save that of migrating code that currently depends
upon it.}
\end{description}

 Some library classes require special VM support.  This is provided by
{\tt com.ibm.JikesRVM.librarySupport} and the classes found in
{\tt rvm/src/vm/OTILibrarySupport} and {\tt
rvm/src/vm/GNUClasspathSupport}.  With the impending demise of the
hybrid-OTI libraries, we anticipate that this code will be
restructured in a forthcoming release.

\JavaTMFooter
\JikesTMFooter


\T \newpage
\xname{debugging}
\section{Debugging Support}
\label{section:debugging}
Previous releases of Jikes\TMweb{} RVM included a home grown
debugger called {\bf jdp}.  The \texttt{jdp} debugger is no longer included in
\jrvm{}.\@  This is because, although the \texttt{jdp} debugger provided some
useful functionality for debugging Jikes RVM itself, it was always a
little unstable and was not really intended to support source level
debugging for programs running on top of Jikes RVM.\@  Prior to the
2.2.0 release, the Jikes RVM core team decided to deprecate \texttt{jdp}
because we believe that the cost of maintaining it outweighs its
usefulness.


The future plan for debugging Jikes RVM consists of two main pieces of
work:

\label{JDWP}
\index{JDWP}
\begin{itemize}
\item Implement \xlink{\textbf{JDWP}}{\JDWPURL} support in Jikes RVM.\@ If Jikes RVM
implemented the standard \xlink{Java Debugging Wire Protocol
(JDWP)}{\JDWPURL}, then a 
number of debugger front ends could be used with Jikes RVM.\@  This
would support most debugging tasks, but would obviously not work if
Jikes RVM itself was crashing or corrupted.

\item Low level debugging of Jikes RVM itself would be performed using
the GNU {\bf gdb} debugger.  Work could be done to either or both of
{\tt gdb} and Jikes RVM to make this more pleasant.  For example,
Jikes RVM could be extended to generate basic {\tt stabs} information
for the boot image.
\end{itemize}

Contributions from the community along either of these fronts would be
greatly appreciated.  Some prototyping of \xlink{JDWP}{\JDWPURL} support for Jikes RVM
was done at IBM during the summer of 2002; we can make this code
available if anyone is interested in completing it.




\T \newpage
\xname{profiling}
\section{Using \jrvm{} to Profile an Application}
This section contains information on several ways in which
Jikes\TMweb{} RVM can be used to profile an application and the VM
itself.  The first section describes how Jikes RVM supports
platform-specific hardware performance monitors (HPM). Currently HPM
support is only available for
PowerPC\TMweb/AIX\TMweb{} version 5, but can be
used in any configuration of Jikes RVM on that platform.  The next two
sections describe how adaptive configurations of Jikes RVM can be used
to gather profile data.  This support is available on all platforms,
but only in adaptive configurations.

%%%%%%%%%%%%%%%%%%%%%%%%%%%%%%%%%%%%%%%%%%%%%%%%%%%%%%%%%%%
\subsection{Using Hardware Performance Monitors}

Jikes RVM can be built to enable access to the PowerPC hardware
performance monitors (HPMs) using the {\tt bos.pmapi} interface 
included in AIX version 5. To build Jikes RVM to use
HPMs, you must define \varName{RVM\_HPM\_DIR} to be the directory containing
the pmapi code (typically {\tt /usr/pmapi}) and \varName{RVM\_WITH\_HPM} to be 1.
See the config file {\tt \$RVM\_\-ROOT/rvm/config/powerpc-ibm-aix5.1.MC4U} for
an example.

After a valid configuration is built, the HPM are accessed through 
command line options. 
The command line options that are available are enumerated with the 
{\tt help} option prefixed with {\tt -X:hpm:}.  This option
generates the following output:
%
\begin{verbatim}
Boolean Options (-X:hpm:<option>=true or -X:hpm:<option>=false) default is false
 Option       Description
 trace        trace HPM counter values at each thread switch.
 processor    print name of processor and number of counters.
 listAll      list all events associated with each counter.
 listSelected list selected events for each counter.

Value Options (-X:hpm:<option>=<value>)
 Option        Type    Description
 eventN        int     specify event for counter N where 1<=N<=UB \
                                and UB is processor specific
 filename      String  prefix for file names.  Concatenate \
                                virtual processor number.
 mode          int     specify mode: 1=GROUP, 2=PROCESS, 4=KERNEL, \
                                8=USER, 16=COUNT, 32=PROCTREE
 trace_verbose int     write events for this PID to the console.  \
                                Used on a multiprocessor.
 verbose       int     print more information.
\end{verbatim}

At least one HPM event command line option must be set for HPM counter values to be gathered.  
When HPM counter values are gathered, aggregate values for each event 
are generated for each virtual processor, and each Java\TMweb{} thread that executes.
For example, the aggregate HPM counter values for virtual processor 1
when Jikes RVM is run on the PowerPC\TMweb{} POWER4 architecture with group 23 
specified generated the following output:

\begin{verbatim}
Dump HPM counter values for virtual processors
 Virtual Processor: 1
0: REAL_TIME           :4,398,503,873
1: PM_LSU_SRQ_S0_VALID :2,244,438,656
2: PM_LSU_SRQ_S0_ALLOC :71,532,205
3: PM_LSU0_BUSY        :2,470,379,577
4: PM_LSU1_BUSY        :1,882,422,375
5: PM_LSU_LRQ_S0_VALID :3,894,273,486
6: PM_LSU_LRQ_S0_ALLOC :153,366,828
7: PM_INST_CMPL        :13,890,418,177
\end{verbatim}

The {\tt -X:hpm:trace=true} command line option generates a trace file for each 
virtual processor. 
(The number of virtual processors is specified by the {\tt -X:processors} command line option.)
A trace file contains a trace record every time a Java thread switch occurs on the 
corresponding virtual processor.
The record contains the HPM counter values at the thread switch in addition to 
the identification of the Java thread.

The Jikes RVM callback mechanism allows the application's soure code 
to make callbacks that when executed will place a marker record in the trace file.
When the trace file is processed, the marker records allow the
programmer to focus on portions of the execution.
Jikes RVM uses this mechanism to open the trace files
when the it starts executing and close the files when it exits.
The AppStart, AppComplete, AppRunStart, and AppRunComplete monitors
are supported for application use.  
Support for other callbacks can be easily added following the current design.

A trace file reader is available in the {\tt \$RVM\_\-ROOT/rvm/src/tools/HPM\_\-trace\-Fi\-le\-Rea\-der}
subdirectory.  Type ``{\tt make}'' on the command line to generate the class files and then 
type ``{\tt java Trace\-Fi\-le\-Rea\-der}'' to determine what command line options are available.


%%%%%%%%%%%%%%%%%%%%%%%%%%%%%%%%%%%%%%%%%%%%%%%%%%%%%%%%%%%
\subsection{Profiling An Application}
One component of the adaptive optimization system is a low-overhead
time-based sampling mechanism.  This information can be used to drive
recompilation decisions
\T~\cite{jalapeno-adaptive-00}.
It can also be used to produce an aggregate
profile of the execution of an application.  
Here's how.

\begin{enumerate}
\item Create an adaptive configuration.  For the most accurate profile
\link{use the {\tt production} configuration.}[  See Section~\Ref, page~\Pageref.]{section:installation}
\begin{verbatim}
% jconfigure production
% cd $RVM_BUILD
% jbuild
\end{verbatim}

\item Run the application using the opt compiler as the runtime compiler and
instructing Jikes RVM to gather profile data.
\begin{verbatim}
% rvm -X:aos:enable_recompilation=false 
      -X:aos:initial_compiler=opt 
      -X:aos:gather_profile_data=true <classfile>
\end{verbatim}
\end{enumerate}

%%%%%%%%%%%%%%%%%%%%%%%%%%%%%%%%%%%%%%%%
\subsection{Instrumented Event Counters}
\label{counting_events}
This section describes how the Jikes RVM optimizing compiler can be
used to insert counters in the optimized code to count the frequency
of specific events.  Infrastructure for counting events is in place
that hides many of the implementation details of the counters, so that
(hopefully) adding new code to count events should be easy.  All of
the instrumentation phases described below require an adaptive boot
image (any one should work).  Most of the code regarding
instrumentation lives in {\tt
\$RVM\_\-ROOT/rvm/src/vm/adaptive/runtimeMeasurements/instrumentation} and {\tt
adaptive/recompilation/instrumentation}.

\link{Below}[, Section~\Ref{} (page~\Pageref)]{existing_phases}\link{ describes existing instrumentation
phases and how to run them}{existing_phases}; \link{immediately
following}[, Section~\Ref{} (page~\Pageref)]{adding_phases}
\link{describes the details of how a new phase can be added}{adding_phases}.

To instrument all dynamically compiled code, use the following command
line arguments to force all dynamically compiled methods to be
compiled by the optimizing compiler: {\tt
-X:aos:en\-a\-ble\_re\-com\-pi\-la\-tion=false -X:aos:i\-ni\-tial\_com\-pi\-ler=opt}

\subsubsection{Existing Instrumentation Phases}
\label{existing_phases}
There are several existing instrumentation phases that can be enabled
by giving the adaptive optimization system command line
arguments. These counters are {\em not} synchronized (as discussed \link*{later}[in
Section~\Ref, page~\Pageref]{adding_phases}), so they should not be considered
precise.
\begin{enumerate}
\item {\bf Method Invocation Counters} 

Inserts a counter in each opt compiled method prologue.  Prints
counters to stderr at end. Enabled by the command line argument.
{\tt -X:aos:insert\_method\_counters\_opt=true}.

\item {\bf Yieldpoint Counters}  

Inserts a counter after each yieldpoint instruction.  Maintains a
separate counter for backedge and prologue yieldpoints. Enabled by 
{\tt -X:aos:insert\_yieldpoint\_counters=true}.

\item {\bf Instruction Counters}  

Inserts a counters on each instruction.  A separate count is
maintained for each opcode, and results are dumped to stderr at end of
run. The results look something like:

\begin{verbatim}
Printing Instruction Counters:
------------------------------
109.0 call
0.0 int_ifcmp
30415.0 getfield
20039.0 getstatic
63.0 putfield
20013.0 putstatic
Total: 302933
\end{verbatim}

This is useful for debugging or assessing the effectiveness
of an optimization because you can see a dynamic execution count, rather
than relying on timing.  

NOTE: Currently the counters are inserted at the end of HIR, so the
counts {\em will} capture the effect of HIR optimizations, and will
{\em not} capture optimization that occurs in LIR or later.  

\item {\bf Debugging Counters}  

This flag does not produce observable behavior by itself, but is
designed to allow debugging counters to be inserted easily in
opt-compiler to help debugging of opt-compiler transformations.
If you would like to know the dynamic frequency of a particular
event, simply turn on this flag, and you can easily count dynamic
frequencies of events by calling the method
\xlink{{\tt VM\_AOS\-Da\-ta\-base.de\-bug\-ging\-Coun\-ter\-Da\-ta.%
get\-Coun\-ter\-In\-struc\-tion\-For\-E\-vent(\-String e\-vent\-Name);}}{\VMAOSDatabaseURL}.  This method
returns an 
\xlink{{\tt OPT\_\-In\-struc\-tion}}{\OPTInstructionURL} 
that can be inserted into the
code.  The instruction will increment a counter associated with
the String name ``eventName'', and the counter will be printed at the
end of execution.

For an example, see 
\xlink{{\tt OPT\_\-In\-li\-ner.java}}{\OPTInlinerURL}.  
Look
for the code guarded by the flag {\tt COUNT\_\-FAILED\_\-ME\-THOD\_\-GUARDS}.
 
Enabled by {\tt -X:aos:insert\_debugging\_counters=true}.

\end{enumerate}

%%%%%%%%%%%%%%%%%%%%%%%%%%%%%%%%%%%%%%%%%%%%%%%%%%
\subsubsection{Writing new instrumentation phases}
\label{adding_phases}
This subsection describes the event counting infrastructure.  It is
not a step-by-step for writing new phases, but instead is a
description of the main ideas of the counter infrastructure.
This description, in combination with the above examples, should be
enough to allow new users to write new instrumentation phases.

\paragraph{Counter Managers:}  Counters are created and inserted into
the code using the 
\xlink{{\tt
OPT\_\-In\-stru\-men\-ted\-Event\-Coun\-ter\-Ma\-na\-ger}}{\OPTInstrumentedEventCounterManagerURL} 
interface.
The purpose of the counter manager interface is to abstract away the
implementation details of the counters, making instrumentation
phases simpler and allowing the counter implementation to be changed
easily (new counter managers can be used without changing any of the
instrumentation phases).  Currently there exists only one counter
manager, 
\xlink{{\tt VM\_\-Counter\-Array\-Manager}}{\VMCounterArrayManagerURL}
, which implements unsynchronized
counters.
When instrumentation options
are turned on in the adaptive system, 
\xlink{{\tt VM\_\-In\-stru\-men\-ta\-tion.boot()}}{\VMInstrumentationURL}
creates an instance of a \xlink{{\tt VM\_\-Counter\-Array\-Manager}}{\VMCounterArrayManagerURL}.

\paragraph{Managed Data:} The class 
\xlink{{\tt VM\_ManagedCounterData}}{\VMManagedCounterDataURL} 
is used to
keep track of counter data that is managed using a counter
manager. This purpose of the data object is to maintain the mapping
between the counters themselves (which are indexed by number) and the
events that they represent.  For example, 
\xlink{{\tt VM\_StringEventCounterData}}{\VMStringEventCounterDataURL} 
is used record the fact that counter \#1
maps to the event named ``FooBar''.  

\ignore{ {\tt VM\_InstrumentedControlFlowEdgeData} is used during edge counting to
record the fact that counter \#1 maps to the ``fallthrough'' edge of
the branch instruction at bytecode offset \#5 at inline context
FooBar.  }

Depending on what you are counting, there may be one data object for
the whole program (such as 
\xlink{{\tt VM\-\_\-Yield\-point\-Coun\-ter\-Da\-ta}}{\VMYieldpointCounterDataURL} and
\xlink{{\tt
VM\_MethodInvocationCounterData}}{\VMMethodInvocationCounterDataURL}), 
or one per method.  There is also a
generic data object called 
\xlink{{\tt VM\_StringEventCounterData}}{\VMStringEventCounterDataURL} 
that
allows events to be give string names (see Debugging Counters above).

\paragraph{Instrumentation Phases:}  The instrumentation itself is
inserted by a compiler phase.  (see
\xlink{{\tt
OPT\_InsertInstructionCounters.java}}{\OPTInsertInstructionCountersURL},
\xlink{{\tt
OPT\_InsertYieldpointCounters.java}}{\OPTInsertYieldpointCountersURL},
\xlink{{\tt
OPT\_\-In\-sert\-Me\-thod\-In\-vo\-ca\-tion\-Counter.java}}{\OPTInsertMethodInvocationCounterURL} 
).  The instrumentation phase
inserts high level ``count event'' instructions (which are obtained by
asking the counter manager) into the code.  It also updates the
instrumented counter to remember which counters correspond to which
events.

\paragraph{Lower Instrumentation Phase:}  This 
\xlink{phase}{\OPTCompilerPhaseURL}
converts the high level ``count event'' instruction into the actual
counter code by using the counter manager.  It currently occurs at the
end of LIR, so instrumentation can not be inserted using this
mechanism after LIR.\@  This phase does not need to be modified if you
add a new phase, except that the {\tt shouldPerform()} method needs to
have your instrumentation listed, so this phase is run when your
instrumentation is turned on.

%%%%%%%%%%%%%%%%%%%%%%%%%%%%%%%%%%%%%%%%
\subsection{Instrumentation Sampling Framework}
\label{instrumentation_sampling}

Jikes RVM contains an implementation of the {\bf \tt Full-Duplication
Sampling Framework} as described in the PLDI'01 paper ``A Framework
for Reducing the Cost of Instrumented Code'' by Arnold and Ryder.
Arnold's Ph.D. thesis contains an expanded description and contains
some implementation details not in the PLDI version.


\subsubsection{What it does}

The sampling framework allows instrumentation that was inserted into
the code to be sampled.  When the instrumentation is not being
executed, the code runs at close to full speed.  This is achieved by
creating a second version of the code within the method and placing
checks that transfer control into the instrumented version.  Currently
the checks use a counter to determine when control should transfer
into the instrumented code.
For full details see the papers mentioned above. 

\T \pagebreak[4]
\subsubsection{How to use it}
\T \nopagebreak
Example usage: 
\T \nopagebreak
\begin{verbatim}
rvm -X:aos:enable_recompilation=false
    -X:aos:initial_compiler=opt
    -X:aos:insert_instruction_counters=true
    -X:opt:instrumentation_sampling=true
    -X:aos:counter_based_sample_interval=10 MyProgram
\end{verbatim}

This will \link{insert instruction counters instrumentation}[ (as described in
Section~\Ref, page~\Pageref)]{counting_events} and enable the instrumentation sampling
framework with a sample interval of at 10; thus, the instrumented code
will be executed roughtly one tenth of the time.   

Relevant optimizing compiler options:


\begin{itemize}

\item INSTRUMENTATION\_SAMPLING -1 false\\
Turn on the instrumentation sampling transformation when compiling the
method.

\item PROCESSOR\_SPECIFIC\_COUNTER -1 true\\
Should there be one CBS counter per processor for SMP performance?  

\item REMOVE\_YP\_FROM\_CHECKING -1 false\\
Should yieldpoints be removed from the checking code?  This helps
lower the overhead of the sampling framework.   As long as the
sample interval sample interval is finite a yieldpoint is guaranteed
to be executed eventually thus the JVM will execute correctly.  

\item NO\_DUPLICATION -1 false \\
Perform the ``no-duplication'' version of the framework where the code
is not actually duplicated.   Essentially this simply conditionalizes
all instrumentation operations.

\item DEBUG\_INSTRU\_SAMPLING -1 false \\
Enable debugging statements for instrumentation sampling

\item DEBUG\_INSTRU\_SAMPLING\_DETAIL -1 false\\
Enable detailed debugging statements for instrumentation sampling
\end{itemize}

Adaptive system options:

\begin{itemize}
\item COUNTER\_BASED\_SAMPLE\_INTERVAL int 1000\\
What is the sample interval for counter-based sampling
\end{itemize}





\T \newpage
\xname{performance}
\section{Experimental Guidelines}
This section provides some tips on collecting performance numbers with
RVM.

\index{boot image}
\index{configurations}
\subsection{Which boot image should I use?}

To make a long story short, use the
\begin{itemize}
\item {\tt FastSemispace} configuration for performance runs that invoke the optimizing compiler on every method, and
\item {\tt FastAdaptiveSemispace} configuration for performance runs that use the adaptive compilation system.
\end{itemize}

These two configurations share the following characteristics:

\begin{itemize}
\item The code placed in the boot image is optimized.
\item The optimizing compiler and associated support are in
the boot image.  Other
configurations with this characteristic begin with the prefix {\tt Full}.
If you do not use a {\tt Fast<...>} or {\tt Full<...>} configuration, 
the optimizing
compiler loads at runtime, and the optimizing compiler itself will be
baseline compiled and run slowly.
\item Both configurations set the final static boolean
{\tt VM.VerifyAssertions = false}.  This flag avoids expensive assertion
checking at runtime.
\index{garbage collection}
\item Both configurations use the non-generational, copying (semi-space) 
collector.  This collector has the fastest allocation sequence.  For
applications with a large working set (eg SPECjbb2000), you may get
better performance from the MarkSweep collector than the Semispace
collector.  Naturally, GC research will need to build configurations
with other collectors.
\end{itemize}

\subsection{What commmand-line arguments should I use?}

For best performance we recommend the following:

\begin{itemize}
\item {\tt -processors all}: By default, RVM uses only one processor.  Setting this option tells the runtime system to utilize all available processors.
\item {\tt -X:irc:O2}: For non-adaptive configurations, this command-line option tells the optimizing compiler to use our highest level of optimization.
\item Set the heap and large heap sizes generously.  We typically set the heap size to at least half the physical memory on a machine.
\item Use a dedicated machine with no other users.  The RVM thread and synchronization implementation do not play well with others.
\end{itemize}

\subsection{RVM is really slow! What am I doing wrong?}


Perhaps you are not seeing stellar RVM performance.  If RVM as
described above is not competitive with the IBM AIX\AIXTMFootnote or
Linux/IA32 product DK, we recommend you test your installation with
the SPECjvm98 benchmarks.  We expect RVM performance to be competitive
with the IBM DK 1.3.0 on the SPECjvm98 benchmarks.

Of course, SPECjvm98 does not guarantee that RVM runs all codes
well.  We have also tested various flavors of pBOB and the Volano
benchmarks, and usually see superior or competitive performance.

The IA32 port is somewhat less mature than the PPC port, and does not
deliver competitive performance on some codes.  In particular, IA32
floating-point performance is mediocre.

Some classes of codes will not run fast on RVM.  Known issues include:
\begin{itemize}
\item RVM start-up is slow compared to the IBM product JVM.
\item Remember that the non-adaptive configurations (eg. Fast) opt-compile
{\em every} method the first time it executes.  With aggressive optimization
levels, opt-compiling will severely slow down the first execution of
each method.  For many benchmarks, it is possible to test the quality
of generated code by either running for several iterations and ignoring
the first, or by building a warm-up period into the code.  The SPEC benchmarks
already use these strategies.  The adaptive configuration does not
have this problem; however, we cannot stipulate that the adaptive
system will compete with the product on short-running codes of a few seconds.
\item We expect RVM to perform well on codes with many threads, such as
VolanoMark.  However, if you have a code with many threads, each using
JNI, RVM performance will suffer due to factors in the design of
the current thread system.
\index{on-stack replacement}
\item RVM does {\em not} yet support on-stack replacement for
optimizing methods.  The adaptive system will not optimize a single
invocation of a long-running 
method.
\index{quasi-preemption}
\item Performance on tight loops may suffer.  The RVM thread system
relies on quasi-preemption; the optimizing compiler inserts a thread-switch
test on every back edge.  This will hurt tight loops, including many
simple microbenchmarks.  We should someday alleviate this problem by
strip-mining and hoisting the yield point out of hot loops.
\item The thread system currently uses a spinning idle thread. If a RVM
virtual processor (ie., pthread) has no work to do, it spins chewing up
cpu cycles.  Thus, RVM will only perform well if there is no other activity on the machine.
\item The load balancing in the system is naive and unfair.  This can hurt some styles of codes, including bulk-synchronous parallel programs.
\item The adaptive system may not perform well on SMPs; this may be due to bad
interaction with the thread load balancer.
\end{itemize}

The RVM developers wish to ensure that RVM delivers competitive performance.
If you can isolate reproducible performance problems, please let us
know. 

\AIXTMFooter

\T \newpage
\xname{coding_style_java}
\section{Java Coding Style Guidelines and Conventions}
\label{section:javacodingstyle}
This section describes a set of coding style guidelines that we
recommend for all code added to the RVM  system.  

Regrettably, much code in the current system does not follow any consistent
coding style.  This unfortunate residue of the system's evolution 
make editing sometimes unpleasant, and prevent javadoc from formatting comments
in
many files.  To alleviate this problem, we present this style guide 
(which consists of a small tweak of the style guide advanced by Sun) 
for new code. 

\index{javadoc}
Most code in the optimizing compiler has been formatted to at least obey 
the indentation rule, 80 columns, and javadoc comments.  Most other RVM code
has not been so formatted. We ask that all new code introduced
into the system, at least for the optimizing compiler, follow the 
guidelines unless compelling factors dictate otherwise.  

\subsection {Coding style description}

The RVM coding style guidelines are defined as a reference to an Sun
Microsystems ``Code Conventions for the Java\trademark\ Programming Language'',
with a few exceptions listed below.  The Sun coding
conventions can be found at 
\xlink{{\tt \SunCodeConventionURL}} {\SunCodeConventionURL} in HTML,
postscript, and PDF.  Most of the style guide is intuitive; 
however, please read through the document or look at its sample code.

We have adopted one revision to the Sun code conventions:
\begin{enumerate}
\index{indenting}
\item {\bf Two space indenting} The Sun coding convention suggests 4
space indenting, however with 80 column lines and 4 space indenting,
there is very little room left for code.  Thus, we recommend using 2
spaces indenting.

\end{enumerate}

\subsection {Javadoc requirements}
\index{javadoc}

All files should contain descriptive comments
in Javadoc form (
\xlink{{\tt \JavadocURL}} {\JavadocURL}
) so
that documentation can be generated automatically.  Of course,
additional non-javadoc source code comments should appear as
appropriate.
For javadoc, at a minimum,

\begin{enumerate}
\item All classes and methods should have a block comment describing
them
\item All methods contain a short description of their arguments
(using {\tt @param}), the return value (using {\tt @return}) and the
exceptions they may throw (using {\tt @throws}).
\item Each class should include {\tt @see} and {\tt @link} 
references as appropriate.
\end{enumerate}

\subsection {Useful tools/hints}
\index{editing source code}
\index{vi}
\index{emacs}

This section describes helpful hints for conforming with the style
guide.  Below are suggestions on how to setup the two most common
editors, emacs and vi. 
%\remark{If we find a pretty-print code processor, we
%can describe it in this section.}

\subsubsection{emacs} 

The following tells {\tt emacs} to indent 2 spaces:
\begin{verbatim}
;; You have to do it in this complicated way because of the
;; strange way the cc-mode initializes the value of `c-basic-offset'.
(add-hook 'c-mode-hook (lambda () (setq c-basic-offset 2)))
(add-hook 'java-mode-hook (lambda () (setq c-basic-offset 2)))
\end{verbatim}
If you want {\tt emacs} to truncate long lines instead of wrapping them, add
the following to your c/java mode hook:
\begin{verbatim}
(setq truncate-lines 't)
\end{verbatim}

\subsubsection{vi}\label{options:vi/vim}

If you are more comfortable with {\tt vi}, it is recommended that you
use a {\tt vi} clone called {\tt vim} 
(\xlink{{\tt \VimURL}}{\VimURL}).  It
contains all of {\tt vi}'s commands and is fully backward compatible,
but is much more configurable than {\tt vi}.  Hints for {\tt vi}
diehards who absolutely refuse to use {\tt vim} are provided at the end
of this subsection (\ref{options:vi}).

\paragraph{vim}\label{options:vim}

Add the following to your {\tt .vimrc} for formatting:
\begin{verbatim}
set shiftwidth=2           " for indenting and shifting
set expandtab              " to replace tab characters by spaces
set smarttab               " to allow the use of <Tab> for indenting
set formatoptions-=t2croq  " reset formatting
set formatoptions+=croq    " format comments
set textwidth=0            " don't wrap text
set wrapmargin=0           " ditto
" Java mode setup
augroup java
   autocmd!
   autocmd BufEnter *.java set cindent
   autocmd BufEnter *.java set cinoptions=>s,e0,n0,f0,{0,}0,^0,:s,=s,ps,ts,c3,+s,(0,u0,)20,*30,gs,hs
   autocmd BufEnter *.java set cinwords=if,else,while,do,for,switch,static,new
   autocmd BufLeave *.java set nocindent
augroup END
\end{verbatim}
If you want {\tt vim} to truncate long lines instead of wrapping them, add
the following to your {\tt .vimrc}:
\begin{verbatim}
set formatoption+=t " to allow autowrap text
set textwidth=74    " to allow autowrap text at 74th column
\end{verbatim}

\paragraph{vi}\label{options:vi}

Standard {\tt vi} options that would approximate Java formatting are:
\begin{verbatim}
set shiftwidth=2  " for indenting and shifting
set autoindent    " automatically indent new lines to the start of previous
\end{verbatim}
and the approximation for wrapping long lines is
\begin{verbatim}
set wrapmargin=6  " to allow autowrap text at 74th column
\end{verbatim}

% LocalWords:  Javadoc param



\T \newpage
\xname{coding_style_other}
\section{Coding Style Guidelines for other Programming Languages}
\label{section:codingstyle-nonjava}
There are several types of source code in the Jikes RVM system.  They
include:


\subsection{Java} 
($\approx 1052$ source files as of this writing, November 21, 2003)  This is
the vast majority of the code.  The names of Java source files always end in
`{\tt .java}''.   We \link{have already discussed Java coding style}[ in
section~\Ref{} (page~\Pageref)]{section:javacodingstyle}. 

We have found it necessary to preprocess the Java code; \link{the
preprocessor is discussed in the \SectionName{Java Preprocessor} section}[ 
  (Section~\Ref), on page~\Pageref]{section:preprocessor}.


\subsection{C}

($\approx 27$ source files)  The file names always
end in ``{\tt .c}''.  We are committed to GNU C.\@  One of the core team
members uses the latest stable GNU C release (3.3.2 as of this writing), but many users
are still running GNU C 2.95 (which came out in July 29, 1999).  So if
you use constructs that are not accepted by GNU C 2.95, then please surround them with
appropriate preprocessor conditionals.  It is OK to use GCC 2.95;
that's the oldest version we'll support.

Our C style is straight Kernighan and Ritchie, with four-space
indenting.  

\subsection{C++} 

($\approx 15$ source files)  The file names always end in
``{\tt .C}''.   The material discussed for C applies here as well.
Some older C++ source files don't meet the coding standard yet.

\subsection{Bourne-Again Shell (Bash)}

We use the GNU Bourne-Again Shell (Bash) exclusively in our shell
scripts ($\approx 13$ source files).  Our build system is driven by
one massive Bash script, {\tt rvm/bin/jconfigure}, which in turn
generates other scripts in the \varName{RVM\_BUILD} directory.

\subsubsection{Indenting}

We use four-space indenting for Bash.  One source file contains an
older indentation style.
We limit lines to 80 columns unless necessary for syntactic reasons to
do otherwise. 

\subsubsection{Declaring functions}

We use the {\tt function} keyword to declare functions:
\begin{verbatim}
function emitCopier () {
\end{verbatim}
rather than eliding it:
\begin{verbatim}
emitcopier () {
\end{verbatim}

\subsubsection{Exit in case of Build Error}

An important consideration for the build process is that if trouble
happens while buiding a part of the system, the build should abort
rather than continuing on.  We have encountered several problems with this,
where the build process continued on despite trouble building the GNU
Classpath library, and users then got an RVM that did not work.  

The Bash source code used in the build system is protected with 
{\tt set~-e}, which causes the shell to immediately exit in case of
trouble, and (on Bash versions that support it) with a {\tt trap}
against the {\tt ERR} pesudo-signal.  You can see an example of this
in the function {\tt emit\_\-enable\_\-exit\_\-on\_\-err\-or} in {\tt
rvm/\-bin/\-j\-con\-fi\-gure}.  

\paragraph{Subshells}

{\tt -e} and {\tt ERR} do not apply to commands executed within
subshells.  So, if you want to execute Make in the directory
{\it directory-name}, it's more robust to use:
\begin{example}
\tt{}make -C {\it directory-name} {\it{}make-target}
\end{example}
instead of:
\begin{example}
\tt{}( cd {\it directory-name} && make {\it{}make-target})
\end{example}
If you want to use a subshell in your build-system code, then try:
\begin{example}
\tt{}({\it subshell-commands }) || false
\end{example}
which will make the right thing happen, or:
\begin{example}
\tt{}if ! ({\it subshell-commands }); then 
    echo >&2 "\$ME: Something bad happened."
    exit 1
fi

\end{example}

\paragraph{{\bf \tt unset} fails on nonexistent variables}
The construct ``{\tt unset} \(\mbox{\textit{variable}}_1\)\ldots'' will
fail if any \(\mbox{\textit{variable}}_i\) is not already set.
If this could happen, you should guard the construct as in this
example:
\begin{example}
\texttt{unset \textit{variablename} || :}
\end{example} 
where the shell command ``\texttt{:}'' (colon) always exits with true
status.



\subsubsection{Miscellany}

\subsection{GNU Makefiles} 

We are committed to GNU Make.  All new makefiles should be named {\tt
GNUmakefile}, \link{just as discussed in the Sun\Rheadingweb{} style guide}[
  mentioned in section~\Ref, on
  page~\Pageref]{section:javacodingstyle}.  We are committed to
pure Bash, so Makefile writers can safely use Bash constructs.




\T \newpage
\xname{faq}
\section{FAQ}
\begin{center}  
{\bf Jikes RVM Frequently Asked Questions}
\end{center}

\subsection{General}

For most general Jikes\TMweb RVM questions and answers, see
\xlink*{\texttt{\QandAURL}}[our question and answer page]{\QandAURL}.

\subsubsection{What is Jikes RVM?} 

The short answer:
The Jikes Research Virtual Machine (Jikes RVM) is a software project
designed to provide the academic and research communities with a
flexible testbed that makes it possible to quickly prototype new
virtual machine technologies and experiment with different design
choices.  Jikes RVM executes Java\TMweb\ programs useful for research on
fundamental virtual machine design issues.
It runs on AIX\TMweb{}/PowerPC\TMweb{}, Mac OS X/PowerPC,
Linux\Rweb{}/PowerPC, and Linux/IA-32, 
and exhibits industry-strength performance for many benchmark programs
on these platforms.  Jikes RVM includes the latest VM
technologies for dynamic compilation, adaptive optimization, garbage
collection, thread scheduling, and synchronization.

\subsubsection{Who is using Jikes RVM}

\xlink{We maintain a list of current Jikes RVM researchers}[ available at
{\tt \RVMUsersURL}]{\RVMUsersURL}.  If you would like to 
be added to the web page, please let us know.

\subsubsection{Can I use Jikes RVM when teaching a class?}

Yes, this is fine under the
\xlink{Common Public License}{\CPLURL}.  In fact,  professors
have already done at both the graduate and undergraduate levels.
\xlink{Teaching material and links to courses taught using Jikes RVM are
available}[ at {\tt \RVMTeachingResourcesURL}]{\RVMTeachingResourcesURL}.

\subsubsection{Whom can I contact with questions?}

\xlink{Use the {\tt jikesrvm-researchers} mailing list}[, available at {\tt \RVMDownloadURL}]{\RVMDownloadURL}.

\subsubsection{Which mailing list(s) should I subscribe to?}

We currently have the following four mailing lists:

\begin{description}
\item[jikesrvm-researchers]
    General discussion of Jikes RVM design, implementation, issues, and
    plans.
\item[jikesrvm-regression]
   Automatic mail messages and subsequent discussion regarding nightly
   regression runs.
\item[jikesrvm-announce]
  Infrequent announcements and news items.
\item[jikesrvm-core]
  Discussion of day-to-day development and design among Jikes RVM  core team
  members.
\end{description}


\subsubsection{How can I contribute to \jrvm{}?}
\xlink{Bug reports or feature requests can be submitted directly}[ at
{\tt \RVMBugURL}]{\RVMBugURL}.  
Information about contributing bug fixes or extensions to the system
can be found at
\xlink{{\tt \RVMContribURL}}{\RVMContribURL}.   

\subsection{Getting \jrvm{} and Documentation}

\subsubsection{How do I get \jrvm{}?}

You can download the \jrvm{} source from DeveloperWorks at \xlink{{\tt
\RVMDownloadURL}}{\RVMDownloadURL}.  The \jrvm{} source is also available
through a public \xlink{CVS server}{\RVMCVSURL}.

\subsubsection{Is there a list of known bugs?}

\xlink{See the bug tracking system on DeveloperWorks}[ at 
{\tt \RVMBugURL}]{\RVMBugURL}.

The bug tracking system lists {\em defects}, representing bugs in the system, and
{\em feature requests}, which are TODO items to improve the system.

\subsubsection{Is there documentation on-line?}

Yes.  \xlink{See the \jrvm{} Home Page}[at
{\tt \RVMHomeURL}]{\RVMHomeURL}.

\subsubsection{Can I get the Quicksilver Quasi-Static System?}

No. This project is no longer active or supported.

\subsubsection{Can I get DejaVu?}

No. This project is not supported. 

\subsubsection{What happened to {\tt jdp}?}

\link{The {\tt jdp} debugger is no longer maintained.}[See
Section~\Ref for more details.]{section:debugging}

\subsection{Building \jrvm}


\subsubsection{What's the right value to use for
RVM\_FOR\_SINGLE\_VIRTUAL\_PROCESSOR?} 
\label{singleProcessorQuestion}

Jikes RVM can be built to support either $m$-to-$n$ or $m$-to-$1$
threading.  In $m$-to-$n$ threading
(RVM\_FOR\_SINGLE\_VIRTUAL\_PROCESSOR equals 0), the virtual machine
multiplexes $m$ Java {\tt Thread}s
onto $n$ virtual processors (operating system pthreads).
In $m$-to-$1$ threading (RVM\_FOR\_SINGLE\_VIRTUAL\_PROCESSOR equals 1),
all $m$ Java threads run on a single virtual processor.  Jikes RVM 
can only exploit multiple CPUs on a hardware SMP machine if 
built with RVM\_FOR\_SINGLE\_VIRTUAL\_PROCESSOR equal to 0.

Unfortunately, $m$-to-$n$ threading is not supported on all platforms due to
mismatches between Jikes RVM  and the host pthread implementation.  
The key issue is that because Jikes RVM multiplexes Java threads
(each with their own stack) onto operating system threads, the
C libraries it is linked with must not use the value of a thread's
stackpointer/framepointer to access pthread-local storage.  The
preferred values for RVM\_FOR\_SINGLE\_VIRTUAL\_PROCESSOR are as
follows: 
\begin{itemize}
\item {\tt AIX/PowerPC}: 0 (SMP is supported)
\item {\tt OSX/PowerPC}:  1 (SMP is not supported)
\item {\tt Linux/PowerPC}:  1 (SMP is not supported)
\item {\tt Linux 2.2/IA32}: 1 (SMP is not supported)
\item {\tt Linux 2.4/IA32}: 0 (SMP is supported), but with the
additional restriction that you must link with a version of glibc that
was compiled to use the GS segment register to access pthread-specific
state.  Depending on your Linux distribution, this may not be the
default for glibc. See a discussion in the archives for
jikesrvm-researchers from November of 2001. If your version of glibc
is not compiled this way, Jikes RVM  will usually fail by 'hanging' when
it is run.
\end{itemize}                

\subsubsection{Which version of {\tt jikes} should I use?}
At Watson, we're currently using \xlink{{\tt jikes}}{\jikesURL} v1.18
to compile the RVM source on Linux\Rweb{} and AIX\TMweb. Previously we
were using version 1.13. Version 1.14 through 1.17 do not entirely
work with Jikes RVM.  For versions of Jikes RVM before 2.2.0, you must
use jikes 1.13.

\subsubsection{Has anybody thought about incremental boot image writing?}

Incremental boot image building is not a trivial problem.  One big
issue is: if we change the implementation of one class in the boot image,
what other parts of the VM image must be invalidated?  Which
methods must be recompiled to reflect the new implementation?  We have no
mechanism in place to trace these kinds of dependencies.  There are other
examples, too.  In summary: incremental boot image writing would be nice,
but it's not easy to support, and it hasn't been at the top of our
priorities.

\subsubsection{Can I monitor progress during {\tt jbuild}?}

Yes.  Use {\tt jbuild -trace} to see detailed progress.  Use {\tt
jbuild -help} to see a list of sub-options you can give to the {\tt
-trace} flag.

\subsubsection{How can I include my own classes in the boot image?}

The {\tt jconfigure} script defines which classes go in the boot image, by
spitting out the file {\tt \$RVM\_BUILD/RVM.primordials}.  By default, any
class, with the {\tt VM\_} prefix, in a defined directory set, goes in the
boot image.

You may choose to add more classes to the primordial list.  One way to do
this is to edit {\tt jconfigure}; look at the function {\tt
emitImageLinker}.  You will see that the script already puts certain other
non-VM classes in the primordial list (eg. {\tt java.lang.Object}).

\subsection{Runtime implementation}

\subsubsection{Does \jrvm{} have an interpreter?}

No.  RVM relies on two compilers, and compiles all methods to native code.

\subsubsection{Does \jrvm{} support JNI?}

Most JNI functionality is supported. A few functions are not. 
There are a few functions that are only supported on AIX\TMweb{}
and some other functions that are only supported on Linux\Rweb{}/IA32.

\subsubsection{Does \jrvm{} support user-defined class loaders?}
Yes, with the caveat that we do not verify linking constraints.

\subsubsection{Does \jrvm{} support the Java\TMheadingweb{} security model?} 

We support it to the extent that the 
\xlink{GNU Classpath}{\classpathURL}  
libraries support it.

\subsubsection{Does \jrvm{} support serialization?}

We support it to the extent that the 
\xlink{GNU Classpath libraries}{\classpathURL}  
support it. We have noticed that sometimes it doesn't
work exactly the same as in the Sun\Rweb{} class library.

\subsubsection{Does \jrvm{} enforce the Java Memory Model?}

No. Depending on the architecture, various features of the memory model
are not implemented according to the current spec, to the best of our
understanding.

Known issues include:
\begin{itemize}
\item on PowerPC\TMweb{}, the system does not enforce
sequential consistency for volatile variables
\item the system does not enforce atomicity of memory accesses for
doubleword values
\item by default, the optimizing compiler does not respect the "reads
kill" property.  However, there is a command-line option to enforce the
property, which constrains the optimizations.
\end{itemize}

\subsubsection{How do \jrvm{}'s threads, Posix threads, and kernel
threads relate to each other?}

RVM implements an $m$-to-$n$ threading model, where $m$ is the number of 
Java threads and $n$ is the number of Posix threads (ie., pthreads).  RVM
does not know or care whether the Posix threads are implemented as kernel
threads or user-level level threads.  You can specify $n$, the number of
Posix threads to use, on the command line with {\tt -X:processors=n}.
You should normally set $n$ to be the number of physical processors on
your machine.  

In the source code, a 
\xlink{{\tt VM\_Thread}}{\VMThreadURL} 
is the base class for each Java
thread, and a 
\xlink{{\tt VM\_Processor}}{\VMProcessorURL} 
is the base class representing each
Posix thread.  

\link{On some platforms, only $m$-to-1 threading is supported}[.  See 
question~\Ref]{singleProcessorQuestion}.

\subsubsection{What is the list of operations that may cause a GC?}

Any operation that allocates memory or causes memory to be allocated may
force a GC.  Some cases to look out for include:
\begin{itemize}
\item any instruction that throws an exception,
\item any call that may cause a stack overflow,
\item any monitorenter on a contended lock,
\item string concatenation, and
\item any thread-switch point may allow another thread to force GC.
\end{itemize}

\subsubsection{How can I implement a new GC algorithm?}

See the memory management section of the userguide.

\subsubsection{How does Jikes RVM enter native code?}

See the discussion in \link{the JNI section of the userguide}[ (\Ref)]{section:jni}. 

\subsubsection{What happens to thread switching while a thread is
executing native code?}
See the discussion in \link{the JNI section}[(~\Ref)]{section:jni}. 

\subsubsection{How do the various locking and synchronization mechanisms
relate to each other?}

There are at least six ways to enforce mutual exclusion in the
RVM runtime.  For normal library code and most VM code, monitorenter and
monitorexit should suffice.  The lower-level primitives provide 
building blocks for implementing monitorenter and exit. Some VM systems,
such as thread scheduling and GC, resort to lower-level primitives for
situations where normal Java object locking is inconvenient or illegal.
\begin{description}
\item[VM\_Magic.prepare and VM\_Magic.attempt]
The RVM compiler translates these 
\xlink{VM\_Magic}{\VMMagicURL} 
calls into low-level
hardware-supported atomic sequences.  These low-level primitives are the 
building blocks for all other mutual exclusion mechanisms. 

The prepare call fetches the
contents of a memory location and begins a conditional critical section.
The attempt call ends the conditional critical section, and returns true
if and only there were no intervening writes to the guarded memory
location.

On PowerPC, the compilers implement prepare and attempt using the lwarx
and stwcx instructions.  On IA32, the compilers rely on CMPXCHG with the
LOCK prefix to implement attempt; prepare is a normal load instruction.
\item[\xlink{VM\_Synchronization}{\VMSynchronizationURL}]
This class implements some useful common low-level synchronization
sequences, such as fetch-and-add and test-and-set.  The VM\_Synchronization
primitives, in turn, are implemented using VM\_Magic.prepare and attempt.
\item[
\xlink{VM\_ProcessorLock}{\VMProcessorLockURL}
]
This lock is used to enforce mutual exclusion between {\tt VM\_Processors}
(pthreads.)  It provides a non-blocking attempt to require the lock
({\tt tryLock()}) as well as a blocking spin-lock ({\tt lock()}).
\item[
\xlink{VM\_Lock}{\VMLockURL}
]
This class provides the normal synchronization operations on Java objects
between Java threads.  The implementation is a variant of Thin Locks.
\item[monitorenter and monitorexit]
Synchronized statements in Java source code are compiled to monitorenter
and exit in the Java bytecode.  The RVM compilers implement these
bytecodes by inserting calls to \xlink{{\tt VM\_Lock}}{\VMLockURL} routines; 
the optimizing
compiler inlines the common cases.
\item[\xlink{VM\_GCLocks}{\VMGCLocksURL}]
This class simply encapsulates a number of VM\_Synchronization locks 
used for various purposes by the RVM GC system.

\end{description}

\subsubsection{Does \jrvm{} conform to Sun's JDK Host Porting Interface?}

No. There is nothing in RVM that remotely resembles HPI.

\subsection{Libraries}

\subsubsection{Why don't you use the GNU Classpath libraries?}
As of version 2.2.1 we do.  

\subsubsection{Does \jrvm{} run awt?}

We believe we support it to the extent that the 
\xlink{GNU Classpath}{\classpathURL}  
libraries support it. 

\subsubsection{Can I run some standard library on \jrvm{} that is not included
in GNU Classpath?}

You can try.  Set your classpath to pick up the library you desire.

\subsection{Optimizing Compiler}

\subsubsection{How can I force all dynamically compiled methods to be
compiled with the optimizing compiler?}
Use the command line argument {\tt -X:aos:initial\_compiler=opt} to the
adaptive system.  If you want to disable profile-driven recompilation,
you also need to give the command line argument 
{\tt -X:aos:enable\_recompilation=false}. 

\subsubsection{What is a PEI?}
\index{PEI: Potentially Excepting Instruction}
PEI is our acronym for potentially excepting instruction.  This applies to
any instruction in the IR that may throw an exception.

\subsubsection{What is AOS?}
\index{AOS}
AOS stands for adaptive optimization system.

\subsubsection{Is there a difference between a GC safe point and a thread
switch point?}
\index{GC safe point}
\index{thread switch point}

Yes.  Every thread switch point is a GC safe point, but every GC safe point 
need not be a thread switch point.

\index{scheduler}
A thread switch point is an instruction where the RVM thread scheduler may
intervene and cause a different Java\TMweb{} thread (VM\_Thread) to
run on the current 
pthread (VM\_Processor), even if no exception is thrown.  
Thread switch points include yield points inserted in prologues, epilogues, 
and back edges, monitorenter and exits.

A GC safe point is any instruction where the compiler must generate a GC map, 
including every thread switch point.  In particular, every 
PEI is a GC point.

\subsubsection{How do I find the def of a register in SSA form?}

Use 
\xlink{{\tt OPT\_Register.getFirstDef()}}{\OPTRegisterURL}.

If this returns {\tt null}, then either a) the register is dead and it's definition has been eliminated, 
or b) the def-use chains are not up-to-date.

The def-use chains are not normally kept up-to-date incrementally.  To
recompute the def-use chains, call 
\xlink{{\tt OPT\_DefUse.computeDU(ir)}}{\OPTDefUseURL}. 
Most optimization passes over SSA form call this method at 
the beginning of the compiler phase.

\subsubsection{What is Heap Array SSA form?}
\index{Heap Array SSA Form}
See the \xlink{SAS 2000 paper}{\SASPaperURL}, as well as comments in
\xlink{{\tt OPT\_SSA.java}}{\OPTSSAURL}.

\subsubsection{Is ABCD included?}
\index{ABCD}
Version 2.0.0 through 2.2.0 of Jikes RVM included a derivative of the
prototype ABCD implementation used for the \xlink{PLDI 2000
paper}{\ABCDPaperURL}.  However, ABCD was {\em not} enabled by default.
The implementation was incorrect, as it checked only upper
bounds and not lower bounds. 

\subsubsection{Is escape analysis included?}

\index{escape analysis}
The interprocedural flow-sensitive escape analysis in the 
\xlink{OOSPLA 99 paper}{\EscapeAnalysisPaperURL} is not currently included.

However, the distribution includes a less-powerful flow-insensitive
escape analysis.  See  
\xlink{{\tt OPT\_SimpleEscape.java}}{\OPTSimpleEscapeURL}.

\subsubsection{How do I insert my new compiler pass in the optimizing
compiler driver?}
\index{compiler pass, inserting a new one}

See \link{the Optimizing Compiler Implementation Details section}[ (\Ref)]{sec:optdriver} of the userguide, which describes how to
add phases to class 
\xlink{\texttt{OPT\_OptimizationPlanner}}{\OPTOptimizationPlannerURL}.

\subsubsection{What if I want my pass to do inter-procedural analysis?}

\index{IPA: inter-procedural analysis}
The normal RVM does not have a convenient entrypoint for IPA.  Because the
Java programming language is
dynamic, the RVM continually compiles methods as they are
invoked.  Each method is compiled individually.

You can use the {\tt OptTestHarness} driver to define a set of classes. 
This driver program loads a set of classes or methods defined on the
command line.  You can then add an entrypoint in {\tt OptTestHarness.java}
that calls your IPA after loading all the relevant classes.

If you come up with a general mechanism for this, please consider
contributing it back.

\subsubsection{What is the OptTestHarness?}

\index{OptTestHarness}
The \link{OptTestHarness}[ (see section~\Ref)]{opttestharness} is a driver
program to run the optimizing compiler even on a BaseBase boot image.
This driver is useful for optimizing compiler development, since you
can use the driver to selectively compile individual methods with
certain options.

\subsection{Regression Tests}

\subsubsection{Where can I order or download the test programs 
  SPECjvm\Rheadingweb{}98 and SPECjbb\Rheadingweb{}2000?}

\index{SPECjvm98}
\index{SPECjbb98}
Information for ordering the SPEC\Rweb{} benchmark suites can be found at
\xlink{http://www.spec.org}{\SPECURL}.
As you'll see from the site, there is a non-trivial charge for getting the
suites, although universities do get a discount.
However, because the license is institution-wide, you may want to check with
other researchers at your institution to see if they already have a license.



\T \newpage
\xname{acknowledgements}
\section*{Acknowledgements}
\begin{quote}
{\em I'm a pepper, you're a pepper, he's a pepper, she's a pepper! Wouldn't you
like to be a pepper too?} \\
\author{--- Ancient proverb}
\end{quote}


Eternal thanks to the following friends,
who set a shining example for
all by contributing text or bug fixes to this document.  
{\em We salute you!}

\begin{itemize}
\item Steve Blackburn
\item Barbara Ryder
\item David Bacon
\item Wilson Hsieh
\item John Davis
\end{itemize}


\T \newpage
\T \bibliographystyle{abbrv}
\T \bibliography{main}

\T \newpage

\T \appendix

\xname{cmdline}
\section{Appendix A: Jikes RVM Command-Line Options}
\label{appendix:nonadaptive:cmdline}
\label{subsection:cmdline}
\index{command line arguments}

This section describes how non-standard Jikes\trademark RVM command
line options are specified. Command line options can be used to modify
the behavior of the runtime system, adaptive optimization system, the
optimizing compiler, the baseline compiler, or the memory manager.  A
command-line directive is constructed by concatenating an option with
a prefix which identifies the desired destination for that option.

All of the non-standard VM options must occur before 
the application class name and application's command-line options.

The order of non-standard command line argument is significant. Each
command line option is processed in order, as typed on the command
line. For example, using a {\tt printOptions} directive for a
subsystem will print the current value of the subsystem's options and
this will reflect only those command line arguments that have preceded
the {\tt printOptions} directive.


%%%%%%%%%%%%%%%%%%%%%%%%%%%%%%%%%%%%%%%%%%%%%%%%%%%%%%%%%%%%%%%%%%%%%%%
\subsection{Adaptive Optimization System (AOS) Command-Line Options}

To see a description of the command-line options to the AOS, use 
{\tt -X:aos:help}.  As of this writing, this command produces the
following output:

\T \begin{tiny}
\input{adaptive_options}
\T \end{tiny}

All options in an adaptive configuration are prefixed with {\tt
-X:aos}.  To pass an option to the adaptive optimization system, use
the {\tt -X:aos:} prefix.  For example, to set the logging level of
AOS to one, use the directive {\tt -X:aos:logging\_level=1}. An
adaptive configuration may conceptually have many optimizing compilers
that are available at runtime, each with its own set of option values.
We present a mechanism to address each conceptual optimizing compiler.
To pass options to the opt compiler that recompiles a method use the
{\tt -X:recomp[?]} prefix where the {\tt ?} is optional and if
specified is an integer that identifies the optimization level.  For
example, {\tt -X:recomp2:global\_bounds=true} performs global array
bounds check elimination on demand when a method is recompiled at
optimization level 2.  If no optimization level is specified, the
option applies to all optimization levels of the optimizing compiler
that recompiles methods.  For example, {\tt
-X:recomp:global\_bounds=true} performs global array bounds check
elimination on demand whenever a method is recompiled with
optimization.  In the default adaptive configurations, the initial
runtime compiler is the baseline compiler.  Options are passed to the
initial runtime compiler by prefixing each option with {\tt -X:irc:}.  

The {\tt enable\_recompilation} option determines whether or not the
adaptive system will actually adaptively recompile methods.  By
default, this is set to true. By setting this to false, Jikes RVM can
simulate a JIT-only system by compiling each method on first
invocation with a specified {\tt initial\_compiler}. For example, 
by saying {\tt -X:aos:enable\_recompilation=false
-X:aos:initial\_compiler=opt}, one can force all methods to be
optimized on first invocation. 
This has the effect of making the initial runtime compiler
be the optimizing compiler (and no recompilations will take place), so
subsequent options in the command line that are prefixed with {\tt
-X:irc:} will be passed to the optimizing compiler. Since no
recompilation will take place options prefixed with {\tt -X:recomp:}
will have no effect.  For example to optimize compile a method at
optimization level 1, use the {\tt -X:irc:O1} option. 

%%%%%%%%%%%%%%%%%%%%%%%%%%%%%%%%%%%%%%%%%%%%%%%%%%%%%%
\subsection{Baseline Compiler Command-Line Options}
\label{section:nonadaptive:baseline:options}

To see descriptions of the command-line options to the baseline
compiler, use the {\tt help} option with the prefix {\tt -X:base:} 
to generate the following output:

\T \begin{small}
% From VM_BASEOptions.template
\begin{verbatim}
-X:base[:help]			Print brief description of baseline compiler's command-line arguments
-X:base:printOptions		Print the current values of the active baseline compiler options

Boolean Options (-X:base:<option>=true or -X:base:<option>=false)
option                               Description
annotations                            Act on annotations in class files
preload_as_boot                        Apply boot options to preload_class
verbose                                Print method name at start of compilation
mc                                     Print final machine code

Value Options (-X:base<option>=<value>)
option                         Type    Description
preload_class                  String  Class to preload upon 1st OPT compilation

Selection Options (set option to one of an enumeration of possible values)

Set Options (option is a set of values)
method_to_print                Only apply print options against methods whose name contains this string
method_to_break                Invoke breakStub for jdp's benefit after compiling methods whose name contains this string
\end{verbatim}


\T \end{small}
 
Note that when the initial runtime compiler is the baseline compiler, 
the {\tt -X:irc} and the {\tt -X:base} command line prefixes are equivalent.

%%%%%%%%%%%%%%%%%%%%%%%%%%%%%%%%%%%%%%%%%%%%%%%%%%%%%%%%
\subsection{Optimizing Compiler Command-Line Options}
\label{section:nonadaptive:optimizing:options}

To see descriptions of command-line options to the optimizing compiler,
use the {\tt help} option with the {\tt -X:opt:} prefix 
to generate the following output:

\T \begin{tiny}
\input{opt_options}
\T \end{tiny}

Note that when the initial runtime compiler is the optimizing compiler, 
the {\tt -X:irc} and the {\tt -X:opt} command line prefixes are equivalent.

%%%%%%%%%%%%%%%%%%%%%%%%%%%%%%%%%%%%%%%%%%%%%%%%%%%%%%%%
\subsection{JMTk Command-Line Options}
\label{section:jmtkoptions}

To see descriptions of command-line options to JMTk,
use the {\tt help} option with the {\tt -X:gc:} prefix 
to generate the following output:

\T \begin{tiny}
\input{jmtk_options}
\T \end{tiny}

%%%%%%%%%%%%%%%%%%%%%%%%%%%%%%%%%%%%%%%%%%%%%%%%%%%%%%%%
\subsection{Adding Jikes RVM Non-Standard Command-Line Options}
This section states how non-standard Jikes\trademark RVM command-line
options can be added to Jikes RVM.  Non-standard Jikes RVM
command-line options are those options that are specific to Jikes RVM.
The format of a non-standard Jikes RVM option is {\tt -X:o=v} where
{\tt o} is the option and {\tt v} is the value it is to be set to.
Adherence to this format is important to keep command-line options
processing from becoming unwieldy.

Jikes RVM command-line options are processed in two places: 
{\tt RunBootImage.C} and {\tt VM\_CommandLineArgs.java}.  
{\tt RunBootImage.C} is called first before the boot image is loaded, and
{\tt VM\_CommandLineArgs.java} is called after.  In addition to
processing any option which does not require the Jikes RVM boot image
to be loaded (such as {\tt help} and {\tt version}), 
{\tt RunBootImage.C} processes any non-standard option that impacts either
heap size, message output, or where to find the boot image.  To allow
unrestricted order of Jikes RVM options and because command-line processing
stops at the first option that is not recognized as a Jikes RVM
option, all Jikes RVM options must be recognized by {\tt RunBootImage.C}
and passed on.

\JikesTMFooter


\T \newpage
\xname{licenses}
\section{Appendix B: Software License}
\label{appendix:licenses}
\index{licenses}%
This section discusses the licensing for Jikes\TMweb{} RVM and for
any distribution that combines \jrvm{} with its usual class
libraries,
\xlink{GNU Classpath}{\classpathURL}.

IBM does not distribute GNU Classpath, nor does IBM distribute
\jrvm{} combined with GNU Classpath, but there does not appear to be
any problem with others doing so.

One of the project's core team members intends
to distribute a \xlink{Debian\Rboth{} GNU/Linux}{\DebianURL} package that
contains a pre-built \jrvm{}, including the compiled GNU Classpath
libraries.  We encourage you to make a package or other distribution
for your favorite operating system that runs Jikes RVM.

\xname{Classpath_license}
\subsection{Class Libraries}

\index{Classpath, License}%
\index{GNU Classpath, License}%
\index{License for GNU Classpath}%
The Java class libraries that Jikes RVM uses, GNU Classpath, 
are under \xlink{the GNU General Public License (GPL)
with a special linking exception}[.  GNU Classpath's
license is available at {\tt \classpathLicenseURL}]{\classpathLicenseURL}.  

\index{Jikes RVM License}%
\index{License}%
\index{Free Software}%
\index{Debian Free Software Guidelines (DFSG)}%
\index{DFSG (Debian Free Software Guidelines)}
%
\xname{license}
\subsection{Jikes RVM License}

Jikes\TMweb{} RVM is \xlink{free software}{\DebianWhatIsFreeSoftwareURL},
distributed and freely redistributable under the Common Public License (CPL).  
This user's guide is part of \jrvm{}.
The CPL has been approved by the \xlink{Open Source Initiative}{\osiURL}
as a fully certified open source license.  The CPL also meets the
\xlink{Debian\Rweb{} Free Software Guidelines}{\DebianFreeSoftwareGuidelinesURL}.


% PFS: Someone will have to fix this link!
%There is a \xlink{CPL FAQ available
%online}[ at \texttt{\CPLFAQURL}]{\CPLFAQURL}.

\xname{CPL_text}
\subsection{Text of the Common Public License (CPL)}
\index{CPL (Common Public License)}%
\index{Common Public License (CPL)}%

\input{cpl-text}
%



\T \newpage
\xname{MMTk_contributions}
\section{Appendix C: MMTk Contributions}
\label{appendix:contributions}
This section contains guidelines on the process of
submitting a contribution to MMTk.

\index{contributions}

\subsection{MMTk Contributions}

We encourage contributions to MMTk in any of these forms: additional
components or collectors, generalization of architecture,
and instrumentation infrastructure.  Because of the sometimes significant
maintenance cost of memory managers, we expect submitted contributions
to adhere to the following guidelines:


\begin{enumerate}

\item {\bf Coding style}  
MMTk judiciously uses modularity and abstraction to lower maintenance
cost and increase code clarity.  We expect contributions, especially
large ones, to follow the same discipline.  Specifically, we highly
encourage reuse of existing components or contributing patches to
augment the functionality of existing components.  Since MMTk is
designed to be a separable module within the VM, MMTk contributions
should adhere to certain modularity constraints.  Namely,
communication with the rest of the VM should occur only through the
classes {\tt VM\_Interface} and {\tt MM\_Interface} and certain magic classes
({\tt VM\_Address}, {\tt VM\_Word}, {\tt VM\_Offset}, and {\tt VM\_Extent}).

\item {\bf Being up-to-date}  
Contributions should be with respect to CVS head or something
reasonably close at least for the affected files.

\item { \bf Stability}  
New collectors or components are expected to be stable with respect to
all GC tests on multiple platforms and on SMP's with both strong ({\it
e.g.}\ Intel\Rweb{}) and weak ({\it e.g.}\ Power) memory models.  Contributions
that are not collectors but provide new functionality such as
instrumentation should provide their own regression tests.
We expect submitted collectors to work for all regression tests.
As a submitter, a good rule of thumb is that contributors should begin
testing with the following test suites: bytecodeTests, gctest, 
SPECjvm\Rweb{}98, and SPECjbb\Rweb{}2000.  \link{See
the \SectionName{Regression Tests} section}[ (\Ref,
page~\Pageref)]{sec:regression} for more details. 

\item {\bf Timely Maintenance}  
The contributor of a significant contribution is expected to be the
advocate of the system and will fix the submitted components if and
when defects are discovered.  We expect serious failures (failure rate
> 15\%) to be fixed within a few days or a week.  Other failures (low
failure rate or non-deterministic failures) should be fixed before a
release.  The core team will strive to announce planned releases with
at least 3 weeks notice.

\item {\bf Advocacy Change}  
If the contributor does not wish to maintain the contributed
component, then the core team may or may not choose to take over
responsibility of the system.  The former is more likely if the
submitted system is general and of value to multiple Jikes users.  
In those cases, a new maintainer may volunteer.  If not and if the
stability of the system degrades, the system may eventually be
removed.

\end{enumerate}


\T \newpage
\xname{Bootstrapping_with_Kaffe}
\section{Appendix D: Bootstrapping with Kaffe}
\label{appendix:kaffe}
\newcommand{\rvmbuild}[1]{\$RVM\_BUILD\texttt{/#1}}%
\newcommand{\gft}{\texttt{GenerateFromTemplate}}%
\newcommand{\gid}{\texttt{GenerateInterfaceDeclarations}}%
\newcommand{\biw}{\texttt{BootImageWriter}}%
%
%
Starting with Jikes RVM version 2.3.2 (or the Jikes RVM 2.3.1 CVS head
as of March 17th, 2004), you can use the Kaffe open-source (GPL)
virtual machine to boot Jikes RVM under Linux, as an alternative to
the Sun-derived JDK.  This eliminates the last piece of non-free
software that used to be required to build Jikes RVM.

You can \xlink{download Kaffe}[ from {\tt \KaffeURL}]{\KaffeURL} or,
under SuSE Linux, install the {\tt kaffe} RPM.  

\subsection{Caveats}

If you want to build under Kaffe, please remember that building Jikes
RVM with the Kaffe VM is not nearly as well-tested as the Sun VM.
There are some caveats here:

\begin{itemize}

\item As of this writing (March 26, 2004), the only Kaffe version we've
  tested against is 1.1.4.  

\item So far, we have only successfully built {\tt BaseBase}{\it *}
(the {\tt prototype} configuration) images doing a pure Kaffe build.
If you want to build an image that will run Jikes RVM's optimizing JIT
compiler, you will have to go through a two-stage boot process.


\item We haven't run the regression test suite through on Kaffe yet.
  Some aspects of it appear to use features of the Sun JDK, and will
  probably need to be ported to use entirely free tools.

\end{itemize}

You will want to use one of the 
{\tt \$RVM\_\-ROOT/rvm/config/i686-pc-linux-gnu.kaffe{\it *}} files as
your starting point.  Now go ahead and \link{follow the regular
  installation directions}[.  
  See Section~\Ref, page~\Pageref]{section:installation}. 

We would very much like help with addressing any of the problems
listed in the caveats above.  We would also very much appreciate
patches for building Jikes RVM under other free Java systems.  We are
currently working on being able to use Jikes RVM itself to write the
boot image.  This does not yet work, because of namespace issues.

\subsection{Build a Jikes RVM Image with the Optimizer's Adaptive System}
\subsubsection{Template Expansion}
  There is a bug in how Kaffe 1.1.4 runs the template expander (\gft)
  when building the \jrvm{} optimizing compiler.  (We are working with
  the Kaffe developers to track down that bug.)

  The bug is discussed in 
  \xlink{\htmlonly{this}\texonly{a May, 2004} 
         message to the \texttt{jikesrvm-core} mailing list}{%
    http://www-124.ibm.com/pipermail/jikesrvm-core/2004-March/000704.html}.


Because of this bug, if you want to build an image that uses the
optimizing compiler (and in order to get good performance from Jikes
RVM, you'll have to do precisely that), you will need to use a
two-stage boot process.

First, build a {\tt prototype} image as discussed above.  Then, create
a new configuration file, this time following the example of {\tt
\$RVM\_\-ROOT/rvm/config/exp/i686-pc-linux-gnu.kaffe-with-help.augart}.    
In your new config file, set {DONOR\_RVM\_ROOT} to be the old RVM\_ROOT
(this will usually be the same as the current RVM\_ROOT), and set
DONOR\_RVM\_BUILD to be the RVM\_BUILD for the \texttt{prototype}
image you just made.

The \jrvm{} you just built will be the Host VM when you're running
\gft{}.  This is the second stage of the boot process.  Because the
Host \jrvm{} is just compiled with the baseline compiler, and because
you're expanding a lot more information, it will take much longer to
run \texttt{jbuild.expand} during this second build --- on this
author's current computer, about 120 seconds, versus less than three
seconds during the first build.

If you plan to hack on Jikes RVM, you will probably be making a lot of
builds.  You can use the newly made optimizing build as the new Donor VM.
Rename the second stage's \${RVM_BUILD} to something like
\$\{RVM_ROOT\}\texttt{/DonorRVMBuild}, and set 
DONOR\_RVM\_BUILD in your configuration file to point to
\$\{RVM_ROOT\}\texttt{/DonorRVMBuild}.  

\paragraph{Outstanding Bug} Do not use a {\tt prototype-opt} build or
a \texttt{BaseAdaptiveCopyMS} build as
your DONOR\_RVM\_BUILD.  On Saturday, March 27, 2004, a bug
was reported on the \texttt{jikesrvm-core} list that a
\texttt{prototype-opt} build (an alias for a
\texttt{BaseAdaptiveGenMS} build) crashed with a
\texttt{NullPointerException} in the garbage collector while expanding
one of the templates.  A \texttt{BaseAdaptiveCopyMS} build also
crashed.  This author recommends using a straight prototype build
until the bug is resolved.  Luckily, you don't have to expand the
templates very often.


\subsection{What does the Donor VM (Host VM) do?}

You must remember that Jikes RVM is written almost entirely in Java.
To build Jikes RVM, we need to be able to run the Jikes RVM compiler
on Jikes RVM itself.  And in order to run that compiler, we need to
have an already running Host VM.  

We use a running Host VM at three stages in the Jikes RVM
building process:

\begin{enumerate}

%
\item \rvmbuild{\IndexTexttt{jbuild.expand}} runs \gft\IndexTexttt{GenerateFromTemplate}, a Java program.
This is a macro processor with recursion, looping,
and so on.   In an x86 \jrvm{} build that includes the adaptive system and
optimizing compiler, \gft{} produces 115,900 lines of Java code, over
4 MB worth.    In an x86 \jrvm{} build that just includes the baseline
compiler, \gft{} still emits 16,416 lines of code, over 600 KB
worth.    \gft's output is in \rvmbuild{\IndexTexttt{RVM.generatedSources}}.

Kaffe 1.1.4 succeeds in this phase for a baseline compiled build, but
is slightly off for an optimizing build.  That is why, to build Jikes
RVM entirely with free software tools, you will need to run a
two-stage build the first time.  Once you have a built instance of
Jikes RVM, you can use that as your host VM for this phase in future
builds.

\item \rvmbuild{\IndexTexttt{jbuild.interfaceDeclarations}} 
launches the Java program \gid\IndexTexttt{GenerateInterfaceDeclarations}, whose product is
``\texttt{RVM.scratch/InterfaceDeclarations.h}''.
\texttt{InterfaceDeclarations.h} is a C++ header file containing the layout of
the Jikes RVM boot record.  It also contains additional declarations
that C and C++ programs may need.

Jikes RVM cannot run \gid{} at this time.  That is because \gid{}
needs to know how Jikes RVM will lay out its boot record, and the only
way to do that is for the host to load Jikes RVM's classes and run it
in a sort of zombie state, just enough to get it to lay out the
VM\_BootRecord class in memory and report back on the offsets of the
various fields in that class.  We enter that zombie state by calling
\texttt{VM.initForTool()}, which exists only for this one use. 
 If you run \gid{} under \jrvm{},
then \texttt{VM.initForTool()} reinitializes the running VM.  The
runtime system immediately forgets that it already loaded a currently
executing program, and quickly aborts with a VM internal error.


\item \rvmbuild{\IndexTexttt{jbuild.linkBooter}} runs the Java program
\IndexTexttt{BootImageWriter}.  This program loads Jikes RVM's classes
and, like \gid, half-animates the just-loaded \jrvm{} into a zombie
state --- this time, via \IndexTexttt{VM.initForBootImageWriter()}.  Once
again, a running instance of \jrvm{} does not like to become into a
zombie.

\end{enumerate}

%% \subsection{Other free VMs}

%% Right now, the other free VMs we've tried to work with as the donor VM
%% (host VM) are:

%% \begin{itemize}

%% \end{itemize}


\xname{userguide_index}
\W \section*{\indexname}\label{hlxindex}
\W \htmlprintindex
\T \printindex

 \W \xname{trademarks}
 \W \section*{Trademarks:}
\label{trademarks}

{\bf \JavaTM} and all \JavaTM{}-based trademarks and logos are trademarks or
registered trademarks of Sun Microsystems, Inc.

{\bf \LinuxR} is a registered trademark of Linus Torvalds.

{\bf \AIXTM}, {\bf \PowerPCTM}, {\bf \IBMR}, and {\bf \JikesTM} are
trademarks or 
registered trademarks of International Business Machines Corporation in the
United States, other countries, or both.

{\bf \UnixR} is a registered trademark in the United States and other
countries, exclusively licensed through X/Open Company, Ltd.

{\bf \SPECjvmR}, {\bf \SPECjbbR}, and {\bf \SPECR} are registered
trademarks of The Standard Performance Evaluation Corporation (SPEC),
a non-profit corporation. 

{\bf \RedHatTM} is a trademark or registered trademark of Red Hat, Inc.

{\bf \SuSER} is a registered trademark of SuSE Linux.

{\bf \IntelR} is a registered trademark of Intel Corporation.

Other product names mentioned herein may be the trademarks or
registered trademarks of their respective owners.



\end{document}

This subsection briefly describes the overall structure of \jrvm.
Details of the various subsystems are provided in subsequent
sections.  

\subsection{Major Components of \jrvm}


The \jrvm can be grouped into the following major components:
\begin{description}
\item [Core runtime] (thread scheduler, class loader, library support,
verifier, etc.) This component is managing all the underlying data
structures required to execute applications and interfacing with
libraries.

\item [Compilers] (baseline, optimizing, JNI) This component is
responsible for generating executable code from bytecodes

\item [Memory managers] This component is responsible for the
allocation and collection of objects during the execution of an
application. 

\item [Adaptive optimization system] This component is responsible
for profiling an executing application
and judiciously using the optimizing compiler to
improve its performance.
\end{description}

More details of each of these components are provided in the following sections.

\subsection{Package Structure}
The 2.2.0 release introduced packages into the system.  Prior releases
did not use packages expclitly, which resulted in all classes being in
the unnamed package.  

There are currently seven packages in \jrvm, all classes are in
one of these packages.
\begin{description}
\item [com.ibm.JikesRVM] Classes for the core runtime, except for library
support.  This package also contains other classes that are not
included in one of the other packages, such as the baseline and JNI
compilers. 

\item [com.ibm.JikesRVM.librarySupport] Classes for library support
\item [com.ibm.JikesRVM.adaptive] The adaptive optimization system

\item [com.ibm.JikesRVM.memoryManagers.vmInterface] Classes related to
memory management that deal with the interface to the VM

\item [com.ibm.JikesRVM.memoryManagers.JMTk] Classes in the newer JMTk
(Java Memory Management Toolkit) collection of memory managers

\item [com.ibm.JikesRVM.memoryManagers.watson] Classes in the older
Watson collection of memory managers

\item [com.ibm.JikesRVM.opt] Classes related to the optimizing
compiler, except for IR-related classes

\item [com.ibm.JikesRVM.opt.ir] Classes related to the IR
(intermediate representation) of the optimizing compiler
\end{description}

We have not followed the Java convention that the source file
directory structure be correlated with the package structure, i.e.,
there is no {\tt com/ibm/JikesRVM} directory anywhere under {\tt
rvmRoot}.  Instead we keep a more logical structure of the source
files under the rvmRoot directory.  One step of building a boot image
is copying the source files from the rvmRoot tree into a build
directory.  The scripts that perform this copy create the directory
structure required by Java semantics and place classes in appropriate
directories.

This approach precludes the need to change the directory structure
if more of the existing classes are placed into packages.






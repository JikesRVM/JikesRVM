\begin{center}
{\bf RVM Frequently Asked Questions}
\end{center}

\subsection{General}

\subsubsection{What is RVM?}

The Jikes Research Virtual Machine for Java is an open-source virtual machine
implementation, written in Java, intended to serve as infrastructure for
programming language research.

\subsubsection{Was RVM once called \jp? What's the difference?}

Yes. There is no difference.  Call a pepper by any other name, and does it
not still burn your mouth?

\subsection{Getting RVM}

\subsubsection{How do I get RVM?}

You need to download two bundles: the RVM source, and the RVM standard library
jar file.  Each of these is available for download from DeveloperWorks at
\remark{TODO! insert URL}.  
The RVM source is also available through a public CVS server 
\remark{TODO.  URL}. 

You can also download the source to the libraries \remark{TODO URL} under
a separate license. \remark{TODO URL}

\subsection{Licenses}

The RVM implementation is licensed open-source under the Common Public
License. \remark{TODO URL} There are separate, more restrictive licenses 
for the binary and 
source to the Java standard libraries for RVM.  See the DeveloperWorks
web pages for more details. \remark{TODO URL}

\subsection{Building RVM}

\subsubsection{Which jikes should I use?}
\remark{TODO URL}
At Watson, we're currently using jikes v1.13 to compile the RVM source on
both Linux and AIX.  We've had reports from users that v1.14 has problems
on Linux.  In order to build the rvmrt.jar library, we applied patch 62 to
the jikes build to fix a jikes scoping problem.


\subsubsection{Has anybody thought about incremental boot image writing?}

Incremental boot image building is not a trivial problem.  One big
issue is: if we change the implementation of one class in the boot image,
what other parts of the VM image must be invalidated?  One example: which
methods must be recompiled to reflect the new implementation?  We have no
mechanism in place to trace these kinds of dependencies.  There are other
examples, too.  In summary: incremental boot image writing would be nice,
but it's not easy to support, and it hasn't been at the top of our
priorities.

\subsection{RVM Design}

\subsubsection{Why doesn't the RVM source use packages?}

This is a historical artifact.  In the early days of the project, we did
not want to be constrained by a package structure to a particular
directory structure.  With the current build process, this is not an
issue, and we may incrementally add package structure to the
implementation over time.

\subsection{Optimizing Compiler}

\subsubsection{What is the OptTestHarness?}

The OptTestHarness is a driver program to run the optimizing compiler even
on a BaseBase boot image.  This driver is useful for optimizing compiler
development, since you can use the driver to selectively compile
individual methods with certain options.


\index{callbacks}%
%
Jikes\TMweb{} RVM provides callbacks for many runtime events of
interest to the Jikes RVM 
programmer, such as classloading, VM boot image creation, and VM exit.  The
callbacks allow arbitrary code to be executed on any of the supported events.

The callbacks are accessed through the nested interfaces defined in the 
\xlink{{\tt VM\_\-Call\-backs}}{\VMCallbacksURL} 
class.  There is one interface per event type.  To be notified
of an event, register an instance of a class that implements the corresponding
interface with {\tt VM\_Callbacks} by calling the corresponding {\tt add...()}
method.  For example, to be notified \link{when a class is
instantiated}[ (see section~\Ref, page~\Pageref{})]{sssec:classLoading}, first implement the {\tt
VM\_Callbacks.ClassInstantiatedMonitor} interface, and then call {\tt
VM\_\-Call\-backs.add\-Class\-In\-stan\-ti\-a\-ted\-Mon\-i\-tor()} with an instance of your class.
When any class is instantiated, the {\tt notifyClassInstantiated} method in
your instance will be invoked.

\xlink{Jikes RVM supports callbacks for a number of events}[; see 
{\tt VM\_Callbacks} for the list of currently
supported callbacks]{\VMCallbacksURL}.

The appropriate interface names can be obtained by appending ``\texttt{Monitor}'' to the
event names ({\it e.g.}\ the interface to implement for the {\tt MethodOverride} event
is {\tt VM\_\-Call\-backs.Me\-thod\-Ov\-er\-ride\-Mo\-ni\-tor}).  Likewise, the method to
register the callback is ``\texttt{add}'', followed by the name of the
interface ({\it e.g.}\ the register method for the above interface is {\tt
VM\_\-Call\-backs.add\-Me\-thod\-O\-ver\-ride\-Mon\-i\-tor()}).

Since the events for which callbacks are available are internal to the VM,
there are limitations on the behavior of the callback code.  For
example, as soon as the exit callback is invoked, all threads are considered
daemon threads ({\it i.e.}\ the VM will not wait for any new threads created in the
callbacks to complete before exiting).  Thus, if the exit callback creates any
threads, it has to {\tt join()} with them before returning.  These limitations
may also produce some unexpected behavior.  For example, while there is an
elementary safeguard on any classloading callback that prevents recursive
invocation ({\it i.e.}\ if the callback code itself causes classloading), there is no
such safeguard across events, so, if there are callbacks registered for both
{\tt ClassLoaded} and {\tt Class\-In\-stan\-ti\-a\-ted} events, and the {\tt
Class\-In\-stan\-ti\-a\-ted} callback code causes dynamic class loading, the {\tt
ClassLoaded} callback will be invoked for the new class, but not the {\tt
Class\-In\-stan\-ti\-a\-ted} callback.

Examples of callback use can be seen in the {\tt VM\_Controller} class in the
adaptive system and the {\tt VM\_GCStatistics} class.

\subsubsection{An Example: Modifying SPECjvm98 to Report the End of a
                  Run}\label{sssec:callback-example}

The SPECjvm\Rboth{}98 benchmark suite is configured to run one or more
benchmarks 
a particular number of times.  For example, the following runs the
{\tt compress} benchmark for 5 iterations:
\begin{example}
 rvm SpecApplication -m5 -M5 -s100 -a \_201\_compress
\end{example}
It is sometimes useful to have the VM notified when the application
has completed an iteration of the benchmark.   This can be performed
by using the {\tt VM\_Callbacks} interface.  The specifics are
specified below:
\begin{enumerate}
\item Modify {\tt spec/harness/ProgramRunner.java}:
	\begin{enumerate}

	\item add an import statement for the {\tt VM\_Callbacks} class:
        \begin{example}
        import com.ibm.JikesRVM.VM\_Callbacks;
        \end{example}

	\item before the call to {\tt runOnce} add the following:
        \begin{example}
        VM\_Callbacks.notifyAppRunStart(className, run);
        \end{example}

	\item after the call to {\tt runOnce} add the following:
        \begin{example}
        VM\_Callbacks.notifyAppRunComplete(className, run);
        \end{example}

	\end{enumerate}

\item Recompile the modified file using {\tt javac} or {\tt jikes}:
\begin{example}
javac -classpath 
   .:\$RVM\_\-BUILD/\-RVM.clas\-ses:\$RVM\_\-BUILD/\-RVM.clas\-ses/\-rvmrt.jar
   spec/\-har\-ness/\-Pro\-gram\-Run\-ner.java
\end{example}
or
\begin{example}
jbuild.tool spec/\-har\-ness/\-Pro\-gram\-Run\-ner.java
\end{example}

\item Run Jikes RVM as you normally would using the SPECjvm98 benchmarks.
\end{enumerate}

In the current system the {\tt VM\_Controller} class will gain control
when these callbacks are made and print a message into the 
AOS log\index{AOS log file}
file (by default, placed in \jrvm{}'s current working directory and called \IndexTexttt{AOSLog.txt}).


Starting with Jikes RVM version 2.3.2 (or the Jikes RVM 2.3.1 CVS head
as of March 17th, 2004), you can use the Kaffe open-source (GPL)
virtual machine to boot Jikes RVM under Linux, as an alternative to
the Sun-derived JDK.  This eliminates the last piece of non-free
software that used to be required to build Jikes RVM.

You can download Kaffe from \xlink{{\tt \KaffeURL}}{\KaffeURL} or,
under SuSE Linux, install the {\tt kaffe} RPM.  

If you want to build under Kaffe, please remember that building Jikes
RVM with the Kaffe VM is not nearly as well-tested as the Sun VM.
There are some caveats here:

\begin{itemize}

\item As of this writing (March 21, 2004), the only Kaffe version we've
tested against is 1.1.4.  

\item So far, we have only successfully built {\tt BaseBase}{\it *}
  images.  Building a \texttt{BaseAdaptive}{\it *} image makes an
  image that currently throws an exception even when running a simple
  ``Hello, World'' program.  This appears to be due to a bug in Kaffe
  1.1.4, which we are actively working to track down.

\item We haven't run the regression test suite through on Kaffe yet.
  Some aspects of it appear to use features of the Sun JDK, and will
  probably need to be ported to use entirely free tools.

\end{itemize}

You will want to use one of the 
{\tt \$RVM\_\-ROOT/rvm/config/i686-pc-linux-gnu.kaffe{\it *}} files as
your starting point.  Now go ahead and \link{follow the regular
  installation directions}[.  
  See Section~\Ref, page~\Pageref]{section:installation}. 

We would very much like help with addressing any of the problems
listed in the caveats above.  We would also very much appreciate
patches for building Jikes RVM under other free Java systems.  We are
currently working on being able to use Jikes RVM itself to write the
boot image.  This does not yet work, because of namespace issues.

